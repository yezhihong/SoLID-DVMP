\section{Proposed Measurements-Transversely Polarized Neutron DVCS}
We propose to measure spin azimuthal asymmetries in deep virtual Compton scattering on transversely polarized neutrons by using a new Solenoidal Large Intensity Device (SoLID) in Hall A. Both the scattered electron and the real photon will be measured coincidentally after the longitudinally polarized incoming electron with $E_{beam}=8.8~GeV$ or $11~GeV$ scattering off the $\mathrm{^{3}He}$ target. The beam single spin asymmetry (BSA), the target single spin asymmetry (TSA) and the double spin asymmetry (DSA) will be extracted from the data and their azimuthal dependence can give access to both real and imaginary parts of corresponding Compton Form Factors (CFFs). %, as listed in Table~\ref{table:cff_asym}. 
The transversely polarized neutron data is particularly sensitive to the GPD $E^{n}$ which is hard to be measured and least known. Combined with other GPDs, $H^{n}$, $H^{p}$ and $E^{p}$, one can perform a flavor decomposition to extract the GPDs for $u$ and $d$ quarks, and investigate the contribution of quark orbital angular momenta to the nucleon spin using the Ji's sum rule.

The measurement will be carried out with exactly the same configuration as the one used in the approved experiment, E12-10-006~\cite{e12-10-006}, which will measure the semi-inclusive deep inelastic scattering process. The SoLID is able to handle a very high luminosity (up to $10^{36} cm^{-2}s^{-1}$), and also has a large acceptance ($0.8\leq P \leq 7.5~GeV/c$, $8^{\circ}\leq \theta \leq 24^{\circ}$, and $0^{\circ}\leq \phi \leq 360^{\circ}$). Such nice features enable us to perform the measurement with high precision and extensive kinematic coverage. An introduction of the experimental setup will be given in next section.

We plan to perform a multi-dimensional binning on the new data and the results will provide strong constraints on the values of CFFs. An analysis technique~\cite{Gui03} to extract the CFFs with asymmetry data has been successfully employed in the data analysis of the 6~GeV experiments in Hall A~\cite{MunozCamacho:2006hx} and in Hall B with CLAS~\cite{PhysRevLett.100.162002}, as well as the HERMES experiments~\cite{Airapetian:2001yk, Airapetian:2012mq, Airapetian:2010ab, Airapetian:2008aa, Airapetian:2011uq, Airapetian:2006zr, Airapetian:2009aa, Airapetian:2009bi}, assuming a leading twist formalism. A new CLAS12 experiment, C12-12-010, was proposed in Hall B to perform the similar measurement but with a transversely polarized proton target (i.e., HDICE), and was conditionally approved because of an on-going effort of developing the new HDICE target.% We have been learning from colleagues in this experiment to use the CFFs fitting algorithm on our projected neutron asymmetries. A full set of fitting results will be given in the actual proposal.
