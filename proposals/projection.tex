\section{Projected Results}

To perform the simulation study and obtain the projected results, we developed
a DEMP generator, as discussed in Appendix-A, and used it to generate events
within a kinematic phase space slightly larger than the SoLID-SIDIS acceptance.
The Fermi motion of the neutron in $\mathrm{^{3}He}$, multiple scattering of
the final state particles, and energy lose due to the ionization, the radiative
effects and so on have been taken in account in this generator.  Then for every
detected article in each event, we added the acceptance profiles obtained from
the GEANT4 simulation with the SoLID-SIDIS configuration and smear the momenta
and angles of the final state particles by the detector resolutions based on
current knowledge of tracking reconstruction study. To better simulate the real
experimental conditions, we generated two sets of data with the target
polarization up and down, respectively.

\subsection{Kinematic Coverage}
\begin{figure}[!ht]
 \begin{center}
      \includegraphics[type=pdf,
        ext=.pdf,read=.pdf,width=0.45\textwidth]{./figures//Q2_x_02Hz}
  
   \caption{\footnotesize{The
kinematic coverage at 11~GeV within the SoLID acceptance. A cut $W^2>$4~GeV$^2$
was applied. }}
  \label{kin_cor}
  \end{center}
\end{figure}
\begin{figure}[!ht]
 \begin{center}
   \includegraphics[type=pdf,
     ext=.pdf,read=.pdf,width=0.75\textwidth]{./figures/dvmp_acceptance_narrow}
   \caption[The acceptance of the momenta and scattering angles for electrons,
     $\pi^{-}$ and protons]{\footnotesize{The acceptance of the momenta and
       polar angles. The top, middle and bottom plots are for electrons,
       $\pi^{-}$ and protons, respectively. Cuts of $Q^{2}>4~\mathrm{GeV^{2}}$
       and $W^2>$4~GeV$^2$ were applied.}}
  \label{p_theta}
  \end{center}
\end{figure}
The kinematic coverage in $Q^{2}$ vs. $x_{B}$ is shown in Fig.~\ref{kin_cor},
using the existing SoLID detectors to detect electrons, pions and protons at
$8^{\circ}\sim24^{\circ}$. These distributions were weighted by the DEMP
unpolarized cross sections and the SoLID acceptance profiles for electrons,
pions and protons.  A cut $W^{2}>4~GeV^{2}$ was also applied to exclude non-DIS
events.

Fig.~\ref{p_theta} shows the momentum and angular acceptance of electrons,
$\pi^{-}$ and protons which form the DEMP events and can be detected with the
SoLID detectors.  Cuts of $Q^{2}>4~\mathrm{GeV^{2}}$ and $W^2>$4~GeV$^2$ were
applied since this is the region of greatest physics interest.  The recoil
protons shown in Fig.~\ref{p_theta} have low momenta ranging from 0.3~GeV/c up
to 1.5~GeV/c and distributes in both the large- and forward-angle regions.
angle.

\subsection{Estimated Rates
\label{sec:rates}}

\begin{table}[!ht]
\centering
\begin{tabular}{|c|c|}
 \hline
  $Q^2>$1~GeV$^2$ & $Q^2>$4~GeV$^2$\\
 \hline
\multicolumn{2}{|c|}{DEMP: $\vec{n}(e,e'\pi^{-}p)$ Triple-Coincidence (Hz)}\\
 \hline
 4.95   &  0.48 \\
 \hline
\multicolumn{2}{|c|}{SIDIS: $\vec{n}(e,e'\pi^{-})X$ Double-Coincidence (Hz)}\\
 \hline
   1424.62  & 35.77   \\
 \hline
\end{tabular}
\caption[Triple-Coincidence rates for
  neutron-DEMP]{\footnotesize{Triple-Coincidence rates for DEMP events compared
    with the SIDIS rates. A cut $W^2>$4~GeV$^2$ was applied. The online production
    trigger will be the SIDIS double-coincidence trigger of which rates are
    also given.}}
\label{rate_table}
\end{table} 

Table~\ref{rate_table} lists the triple-coincidence rate of the DEMP
events. The rates were calculated with the simulated events weighted by the target
luminosity, the SoLID acceptances and unpolarized cross sections.  The ``raw''
rates are not corrected by the beam and target polarization, target dilution
and so on.  Our conservatively estimated rate is around 4.95~Hz at
$Q^{2}>$1~GeV$^{2}$, or 0.48 Hz at $Q^{2}>$4~GeV$^{2}$. For comparison, the
table also gives the SIDIS rate which will be the online production trigger
rates and is the main background of the DEMP events.

\subsection{Asymmetry Projections}
\begin{figure}[!ht]
 \begin{center}
      \includegraphics[type=pdf,
        ext=.pdf,read=.pdf,width=0.5\textwidth]{./figures/E11_Q2_t_bin_02Hz} 
    \caption[$Q^{2}$ vs. $-t$]{\footnotesize{$Q^{2}$ vs. $-t$ where the black
dashed lines specify the boundaries of 7 $-t$ bins and the black dash-dot lines
indicate the additional two $Q^{2}$ bins. }}
  \label{Q2_t_bin}
  \end{center}
\end{figure}

The proposed experiment will run in parallel with E12-10-006, which has already
been approved to run 48 days at $E_{0}$=11~GeV.  As shown in
Fig.~\ref{Q2_t_bin}, we defined 7 $-t$ bins of which the boundaries are
defined by the array:
 \begin{equation}
-t[8] = [0.0, 0.15, 0.25, 0.35, 0.45, 0.55, 0.75, 1.10]~~~~(\mathrm{in~GeV^{2}})
 \end{equation}
The number of events ($N^{\uparrow\downarrow}_{i}$) in the $i$th bin is
calculated from the total simulated events after applying cuts on important
kinematic variables, e.g. $Q^{2}>$4~GeV$^{2}$, $W>$2~GeV, 0.55$<\epsilon<$0.75
and $-t_{min}<-t<-t_{max}$. Two simulated data sets with target polarization up
and down follow exact the same cuts and binning.  As shown in
Eqn.~\ref{ncount}, each event surviving the cuts is then weighted by the
polarized cross section, together with the acceptance of the electron, pion and
proton. $N^{\uparrow\downarrow}_{i}$ is further corrected by the phase-space
factor ($PSF$) defined in the event generator, the total number of randomly
generated events ($N_{gen}$), beam-time ($T$), the target luminosity
($L=10^{36}$~cm$^{-2}$s$^{-1}$), and the overall detector efficiency
($\epsilon_{eff}$):
 \begin{equation}
     N^{\uparrow\downarrow}_{i} = \bigl(\sum_{j\in i-bin}
     \sigma^{\uparrow\downarrow}_{j}\cdot A^{e}_{j} \cdot
     A^{\pi^{-}}_{j} \cdot A^{p}_{j}\bigr) \cdot (PSF/N_{gen}) \cdot T \cdot L
     \cdot \epsilon_{eff},
     \label{ncount}
 \end{equation}
where $j$ is the $j$th event in the $i$th bin,
$\sigma^{\uparrow\downarrow}_{j}$ is the cross section of the $j$th event with
the target polarization up or down. $A^{e(\pi^{-},p)}_{j}$ is the acceptance
weight of the electron (pion, proton) in this event. The detector efficiency,
$\epsilon_{eff}$, is approximately fixed at 85\% as was used in SIDIS
proposals. $N^{\uparrow\downarrow}_{i}$ corresponds to the raw experimental
count of electrons scattering on neutrons in $\mathrm{^{3}He}$ after taking
into account the target polarization ($P\sim60\%$), the effective polarization
of neutrons ($\eta_{n}\sim0.865$), and the dilution effect from other reaction
channels when electrons scattering on $\mathrm{^{3}He}$ ($d \sim 0.9$).

In addition, we further divide each $-t$-bin into two $Q^{2}$ bins with similar
statistics, as indicated in Fig.~\ref{Q2_t_bin}.  By doing that, we are able to
examine the $Q^{2}$-dependence of the asymmetries, and also check the model
dependence of the other corrections that are directly related to the values of
$Q^{2}$.

With the numbers of simulated events in each bin for two anti-parallel target
polarizations, one is able to reconstruct the average target single-spin
asymmetry in that bin, which is identical to the experimental extracted
asymmetry:
\begin{equation}
   <A_{UT}> = \frac{1}{P\cdot\eta_{n}\cdot d}
   \frac{N^{\uparrow}-N^{\downarrow}}{N^{\uparrow}+N^{\downarrow}}.
   \label{asym_exp}
\end{equation}
The statistical error of the target single spin asymmetry ($A_{UT}$) in each
bin can be given as:
  \begin{equation}
    \delta A_{UT} = \frac{1}{P\cdot\eta_{n}\cdot d} \sqrt{\frac{1-(P\cdot
        <A_{UT}>)^{2}}{N^{\uparrow}_{i}+N^{\downarrow}_{i}}},
    \label{stat_err}
 \end{equation}
 
%Because of not performing a L/T separation in this experiment, the asymmetry
%should be corrected by another dilution factor, which is defined as:
%\begin{equation}
%  f_{L/T} =\frac{\epsilon\sigma_{L} }{\sigma_{T}+\epsilon\cdot\sigma_{L} },
%\end{equation} 
%where $\epsilon=(1+\frac{2\nu^{2}}{Q^{2}}\tan^{2}(\theta))^{-1}$. Additional
%dilution due to $\sigma_{TT}$ is assumed to be small.  A factor of $-1$ is also
%applied after comparing Eq.~\ref{eqn:asy} and Eq.~\ref{eqn:sigtarg}. Hence,
%$A_{UT} = -f_{L/T}\cdot A_{L}^{\perp,model}$.

Shown in Eqn.~\ref{eqn:six_asym} in Sec.~\ref{sec:sinphiS}, $A_{UT}$ can be
further decomposed into six asymmetries with different azimuthal angular
modulations:
   \begin{eqnarray}
        A_{UT}(\phi, \phi_{S}) &=& A_{UT}^{\sin(\phi-\phi_{S})}
        \sin(\phi-\phi_{S})+ A_{UT}^{\sin(\phi_{S})} \sin(\phi_{S}) \nonumber \\
       &+& A_{UT}^{\sin(2\phi-\phi_{S})} \sin(2\phi-\phi_{S})+
        A_{UT}^{\sin(3\phi-\phi_{S})} \sin(3\phi-\phi_{S}) \nonumber \\
      &+& A_{UT}^{\sin(\phi+\phi_{S})}
        \sin(\phi+\phi_{S})+A_{UT}^{\sin(2\phi+\phi_{S})} \sin(2\phi+\phi_{S}).
   \end{eqnarray}
In our generator, the first five different azimuthal modulations of $A_{UT}$
are predicted with a phenomenological model as discussed in Appendix-A, and the
last modulation is fixed to be zero as the model predicts a negligible
asymmetry.  As discussed in Sec. 1, two modulations,
$A_{UT}^{\sin(\phi-\phi_{S})} $ and $A_{UT}^{\sin(\phi_{S})}$, are particularly
interesting and are the main quantities this proposal aims to measure.

To demonstrate that the proposed measurement has the capability of extracting
these two asymmetries, we adopted the procedure presented in the HERMES
thesis~\cite{hermes-thesis} to extract all five asymmetries by using a unbinned
maximum likelihood (UML) method. Compared with the regular extraction methods
where the data in each -t bin are further binned into two dimensional ($\phi$,
$\phi_S$) bins, the UML method can perform much better fitting when the
statistics is limited.

The polarized cross sections with two target polarization directions are given
approximately:
\begin{eqnarray}
     \sigma_{\uparrow} &=& \sigma_{UT}(\phi, \phi_{S}) = [1 +
       \frac{|P_{T}|}{\sqrt{1-\sin^2(\theta_{q})\sin^2(\phi_{S})}} A_{UT}(\phi,
       \phi_{S})] \cdot \sigma_{UU}(\phi) \\
     \sigma_{\downarrow} &=& \sigma_{UT}(\phi, \phi_{S}+\pi) = [1
       -\frac{|P_{T}|}{\sqrt{1-\sin^2(\theta_{q})\sin^2(\phi_{S})}} A_{UT}(\phi,
       \phi_{S})]\cdot \sigma_{UU}(\phi).
\end{eqnarray}
where $|P_{T}| = P\cdot \eta_{n} \cdot d $. Hence, the probability density
function can be constructed as:
\begin{equation}
  f_{\uparrow\downarrow}(\phi, \phi_{S}; \theta_{k}) =
  \frac{1}{C_{\uparrow\downarrow}} [1 \pm
    \frac{|P_{T}|}{\sqrt{1-\sin^2(\theta_{q})\sin^2(\phi_{S})}} \sum_{k=1}^{5}
    \theta_{k} \sin(\mu\phi+\lambda\phi_{S})],
\end{equation}
where $\theta_{k}$, $k=1-5$, are the values of asymmetries that can maximize
the likelihood function.  $C_{\uparrow\downarrow}$ is a normalization constant
and is set to one as it is not important in the UML fitting. Here we have
dropped out the sixth asymmetry which is zero. The UML function can be defined
as:
\begin{equation}
	L(\theta_{k}) = L_{\uparrow}(\theta_{k})\cdot
        L_{\downarrow}(\theta_{k})=\prod_{l=1}^{N_{MC}^{\uparrow}}[f_{\uparrow}(\phi_{l},
          \phi_{S,l};\theta_{k})]^{w^{\uparrow}_{l}}\cdot
        \prod_{m=1}^{N_{MC}^{\downarrow}}[f_{\downarrow}(\phi_{m},
          \phi_{S,m};\theta_{k})]^{w^{\downarrow}_{m}}.
\end{equation}
where $w^{\uparrow}_{l} = \sigma^{\uparrow}_{l}\cdot A^{e}_{l} \cdot
A^{\pi^{-}}_{l} \cdot A^{p}_{l} \cdot PSF/N_{gen} \cdot T \cdot L \cdot
\epsilon_{eff}$, is the weight of the $l$th simulated event. It takes into
account the fact that the Monte-Carlo events are generated uniformly, and it
also includes the experimental conditions such as the acceptances of three
particles and the detector efficencies. For the real experimental data, the
weight will only takes into account the acceptance correction, detector
efficiencies correction and other experimental related corrections. From
Eq.~\ref{ncount}, one has $ N^{\uparrow}_{i} = \bigl(\sum_{l\in
  i-bin}w^{\uparrow}_{l})$. Note that $N_{MC}^{\uparrow}= \bigl(\sum_{l\in
  i-bin})$ is simply the total number of simulated events in the $i$th $-t$ bin
without any weighting. In practice, we use the TMinuite package to minimize the
following negative log-likelihood function:
\begin{equation}
  -lnL(\theta_{k}) =-ln
  L_{\uparrow}(\theta_{k})-lnL_{\downarrow}(\theta_{k})=-\sum_{l=1}^{N_{MC}^{\uparrow}}
  w^{\uparrow}_{l}\cdot lnf_{\uparrow}(\phi_{l},
  \phi_{S,l};\theta_{k})-\sum_{m=1}^{N_{MC}^{\downarrow}}
  w^{\downarrow}_{m}\cdot lnf_{\downarrow}(\phi_{m}, \phi_{S,m};\theta_{k}).
\end{equation}

\begin{figure}[!ht]
 \begin{center}
               \includegraphics[type=pdf,
        ext=.pdf,read=.pdf,width=0.65\textwidth]{./figures/bin_asym_t_fermi_02Hz}
      
      \caption{\footnotesize{Projection of $A_{UT}^{\sin(\phi-\phi_{S})}$ as a
          function of $-t$ (directly compare with Fig.~\ref{fig:hermes_aut}).}}
  \label{asym1_t}
  \end{center}
\end{figure}
\begin{figure}[!ht]
 \begin{center}
        \includegraphics[type=pdf,ext=.pdf,read=.pdf,width=0.65\textwidth]
{./figures/bin_asym_t_fermi_02Hz} 
        
        \caption{\footnotesize{Projection of $A_{UT}^{\sin(\phi-\phi_{S})}$ as a
            function of $-t$ (directly compare with
            Fig.~\ref{fig:hermes_aut}).}}
  \label{asym2_t}
  \end{center}
\end{figure}

Fig.~\ref{asym1_t} (Fig.~\ref{asym2_t} ) compare the distribution of
$A_{UT}^{\sin(\phi-\phi_{S})} $ ($A_{UT}^{\sin(\phi_{S})}$) vs. $-t$ between
values from the UML fitting and from the direct statistical averaged model
calculations.  Compared with the existing HERMES results
(Fig.~\ref{fig:hermes_aut}), the new measurement could provide more precision
data to be directly compared with theoretical predictions. Extra binning on
$Q^{2}$ enables us to study the $Q^{2}$-dependence of asymmetries as well as to
constraint some corrections during the asymmetry extraction.  The detailed
information of each bin is listed in Table~\ref{asym_bin_table}.
\begin{table}[!ht]
\centering
 \small
\begin{tabular}{|c|c|c|c|c|c|c|c|}
\hline
 \multicolumn{8}{|c|}{$Q^{2}$ bin-set\#1 } \\
\hline
      &  t-bin\#1 & t-bin\#2 & t-bin\#3 & t-bin\#4 & t-bin\#5 & t-bin\#6 & t-bin\#7 \\
  \hline
$<-t>$    &  0.13 &  0.20 & 0.30 & 0.40 & 0.50 & 0.63 & 0.91 \\
$<Q^{2}>$   &  4.11 &  4.36 & 4.73 & 5.10 & 5.48 & 5.96 & 6.66 \\
$<A_{UT}>$ &  -1.12$\times 10^{-1}$ &  -2.16$\times 10^{-1}$ & -2.69$\times 10^{-1}$ & -2.77$\times 10^{-1}$ & -2.67$\times 10^{-1}$ & -2.33$\times 10^{-1}$ & -1.31$\times 10^{-1}$ \\
$\delta A_{UT}$  &  1.92$\times 10^{-2}$ &  8.49$\times 10^{-3}$ & 8.18$\times 10^{-3}$ & 9.64$\times 10^{-3}$ & 1.32$\times 10^{-2}$ & 1.37$\times 10^{-2}$ & 1.76$\times 10^{-2}$ \\
\hline

\multicolumn{8}{|c|}{$Q^{2}$ bin-set\#2 } \\
\hline
      &  t-bin\#1 & t-bin\#2 & t-bin\#3 & t-bin\#4 & t-bin\#5 & t-bin\#6 & t-bin\#7 \\
  \hline
$<-t>$     &  0.13 &  0.20 & 0.30 & 0.40 & 0.50 & 0.63 & 0.92 \\
$<Q^{2}>$   &  4.35 &  4.87 & 5.45 & 5.98 & 6.43 & 6.92 & 7.63 \\
$<A_{UT}>$   &  -8.15$\times 10^{-2}$ &  -1.39$\times 10^{-1}$ & -1.83$\times 10^{-1}$ & -2.03$\times 10^{-1}$ & -2.11$\times 10^{-1}$ & -1.95$\times 10^{-1}$ & -1.31$\times 10^{-1}$ \\
$\delta A_{UT}$   &  2.20$\times 10^{-2}$ &  9.17$\times 10^{-3}$ & 8.91$\times 10^{-3}$ & 1.04$\times 10^{-2}$ & 1.25$\times 10^{-2}$ & 1.23$\times 10^{-2}$ & 1.57$\times 10^{-2}$ \\
\hline
\end{tabular}
\caption[Detailed information of projected bins]{\footnotesize{Detailed
    information of projected bins from the new DEMP measurements with SoLID,
    while $<Q^{2}>$ and $<-t>$ are in the unit of GeV$^{2}$. The data are
    divided into 14 $-t$ bins in both $-t$ (7 bins) and $Q^{2}$ (2 bins).  The
    projected uncertainties are statistical only.}}
\label{asym_bin_table}
\end{table} 

\subsection{Missing Mass and Background}
In the DEMP reaction on a neutron, all three charged particles in the final
state, $e^{-}$, $\pi^{-}$ and $p$, can be cleanly measured by the SoLID
detector system.  Hence, contamination from other reactions, including DEMP
with other two protons in $^{3}$He, can be greatly eliminated.  The dominant
background of the DEMP measurement comes from the SIDIS reactions of electrons
scattering on the neutron and two protons in $\mathrm{^{3}He}$.  In a SIDIS
event, besides the scattered electron and the hadron ($\pi^{\pm}$, $K^{\pm}$
etc.), there could be at least a proton in the target fragments, and in that
case, the SIDIS event will be possibly misidentified as a DEMP event.  In
addition to identifying the recoil protons, which should largely suppress most
of background, we will also rely on reconstructing the missing masses and
missing momentum spectra of recoil protons to ensure the exclusivity of the
DEMP events.

To calculate the missing mass and missing momentum of the recoil proton, we
reconstructed the four-momentum Lorentz vectors of the incoming electron, the
scattered electron and the pion.  We assumed the neutron target is at rest even
though its Fermi motion was simulated in the generator.  We then used the
momentum and energy conservation laws to calculate the missing mass and missing
momentum of the recoil proton.  Note that the energy loss, multiple scattering
effects, and the detector resolutions have been considered in our study.  Based
on the current tracking study, the SoLID-SIDIS system can provide a momentum
resolution of $2\%/\sqrt{E}$, a polar angle resolution of 0.6~mrad, an
azimuthal angle resolution of 5~mrad and a vertex target position of 0.5~cm.
In this measurement, we proposed to only identify the recoil proton. However,
with improved time resolution and certain tracking information, the momenta of
the recoil protons will still be determined with certain accuracies which gives
us a room to further suppress any background.

The SIDIS events, $p(e,e'\pi^{-})X$ and $n(e,e'\pi^{-})X$, were simulated with
the same generator used for the SoLID-SIDIS proposals.  The same acceptance
profiles of scattered electrons and pions in DEMP were applied to the SIDIS
events, along with the same kinematic cuts, such as $Q^2>4~GeV^2$.  It is
difficult to estimate what percentage of the target fragments in SIDIS contain
at least a proton, so we have to assumed all target fragments ($``X''$) contain
proton. Such an assumption likely results in the SIDIS background being
significantly overestimated.

We followed exact the same methods as ones in DEMP to calculate the missing
masses and missing momenta of the recoil protons in SIDIS.
Fig.~\ref{missing_mom} shows the reconstructed missing momenta of both
processes.  One immediately sees that the main peak of the DEMP missing
momentum spectrum is well separated from the SIDIS background, which can be
largely rejected with a loose cut of $P_{miss}<1.2$~GeV/c.
\begin{figure}[!ht]
\begin{center}
\includegraphics[type=pdf,ext=.pdf,read=.pdf,width=0.5\textwidth]
{./figures/missing_mom_cut}
\caption[Missing Momentum]{\footnotesize{Missing momentum spectra of recoil
    protons in DEMP (blue) and SIDIS (red) processes with a polarized $^{3}$He
    target. The dashed magenta curve is the DEMP missing mass only considering
    the Fermi motion, multiple scattering and the energy loss, while the blue
    and red curves have further taken into account the detector
    resolutions. The light-blue dashed line indicates a cut at
    $P_{miss}<1.2$~GeV/c to remove most of the SIDIS events.  Note that the
    SIDIS background is overestimated since we assume all SIDIS events contain
    protons in the final state. }}
  \label{missing_mom}
  \end{center}
\end{figure}

Fig.~\ref{missing_mass} shows the reconstructed missing mass spectra of the
DEMP and SIDIS events w/ and w/o the missing momentum cuts. Before applying the
missing momentum cut, the tail of the SIDIS background significantly leaks into
the DEMP peak. Of cause, keep in mind that the SIDIS rate is likely
overestimated.  After applying the missing momentum cut, the SIDIS background
is largely suppressed.  The total integrated SIDIS becomes 0.04Hz compared with
the rate of DEMP rate (0.5Hz).  Further considering the fact that on partial
$``X''$ in SIDIS contains a proton, we conclude that the SIDIS background is
negligible after the cut.
\begin{figure}[!ht]
 \begin{center}
 \subfloat[Before cutting on missing momentum] {
      \includegraphics[type=pdf,
        ext=.pdf,read=.pdf,width=0.45\textwidth]{./figures/Missing_Mass}
 }
  \subfloat[Before cutting on missing momentum] {
     \includegraphics[type=pdf,
       ext=.pdf,read=.pdf,width=0.45\textwidth]{./figures/Missing_Mass_cut}
 }
   \caption[Missing Mass]{\footnotesize{Missing mass spectra of recoil protons
       in DEMP (blue) and SIDIS (red) processes with a polarized $^{3}$He
       target. The left (right) plot shows the background contamination from
       SIDIS events before (after) the missing momentum cut (shown in
       Fig.~\ref{missing_mom}).  The dashed magenta curve is the DEMP missing
       mass only considering the Fermi motion, multiple scattering and the
       energy loss, while the blue and red curves have further taken into
       account the detector resolutions. Note that the SIDIS background is
       overestimated since we assume all SIDIS events contain protons in the
       final state. }}
  \label{missing_mass}
  \end{center}
\end{figure}

The background sources, such as the random coincident background events, will
show up in the missing mass spectrum with more uniform distributions. We should
be able to suppress most of them with tight missing momentum and missing mass
cuts.  The residual background events will be largely suppressed or corrected
during the real asymmetry extraction. In general, we expect to have a clean
measurement of the DEMP process because all of the final particles being
detected.

\subsection{Systematic Uncertainties}

\begin{table}[!htp]
\centering
\begin{tabular}{|c|c|}
\hline
{\bf Sources}            & {\bf Relative Value} \\\hline
Beam Polarization        & $2\%$ \\\hline 
Target Polarization      & $3\%$ \\\hline 
Dilution Factor          & $1\%$ \\\hline 
Nuclear Effect           & $<4\%$ \\\hline 
Acceptance               & $3\%$ \\\hline
Radiation Correction     & $2\%$ \\\hline
Background Contamination & $<5\%$ \\\hline
\end{tabular}
\caption{\footnotesize{Expected systematic errors.}}\label{table:det_sys_err}
\end{table}

The systematic errors are expected to be close to the ones given in the
E12-10-006 proposal as well as in other SIDIS experiments with SoLID. The
procedure of extracting DEMP asymmetries is also expected to be similar to the
SIDIS asymmetry extraction.  The contamination of background should be well
controlled by the proton detection and cuts on missing momenta and mass.
However, to be conservative, we quote the overall systematic errors of
background contamination to be $5\%$ level.  Here we list several major sources
of systematic uncertainties as shown in Table~\ref{table:det_sys_err}.
