\section{Experimental Method}

We propose to carry out the $^3He(e,e'\pi^-)p(pp)_{sp}$ measurement using the
Solenoidal Large Intensity Device (SoLID~\cite{solid_pcdr}), in parallel with
the already approved experiment, E12-10-006~\cite{solid:e12-10-006}, which will
measure Semi-Inclusive Deep-Inelastic Scattering (SIDIS). There are two SoLID
configurations, called SoLID-SIDIS and SoLID-PVDIS. Besides E12-10-006, two
SIDIS experiments, E12-11-007~\cite{solid:e12-11-007} and
E12-11-108~\cite{solid:e12-11-108}, along with the $J/\psi$ experiment
(E12-12-006~\cite{solid:e12-12-006}), will use the SoLID-SIDIS
configuration. All of these experiments have been approved with A or A-
rating. In addition, two ``bonus-run'' experiments,
E12-10-006A~\cite{solid:e12-10-006A} and E12-11-108A~\cite{solid:e12-11-108A},
have also been approved to run in parallel with the SIDIS experiments. The
SoLID-PVDIS configuration is for the Parity Violation in Deep Inelastic
Scattering (PVDIS).

The experiment will use a near identical setup as E12-10-006, but with few
additions without affecting the approved experiment. We will use exactly the
same online production trigger, which is the coincidence of electron triggers
and hadron triggers. However, we request to add a new trigger type on top of
the existing ones to identify the proton events for the offline triple
coincidence analysis. The SoLID-SIDIS detector can only detect protons with
scattering angles from 8$^{\circ}$ up to 24$^{\circ}$, while the main proton
events from the DEMP process can cover up to 65$^{\circ}$. We propose to add a
new proton detector based on scintillator counters to detect protons from
24$^{\circ}$ to 65$^{\circ}$. The new detector will be placed between the
target system and the entrance and of CLEO-II magnet. The new proton trigger
and the new proton detector will be discussed in more detailed in the following
sections.

\subsection {Transversely Polarized $\mathrm{^{3}He}$ Target}

The proposed measurement will utilize the same polarized $\mathrm{^{3}He}$
target as E12-10-006~\cite{e12-10-006}. Such a target was successfully employed
in E06-110, a 6~GeV SIDIS experiment in Hall A. 
A wide range of experiments have utilized polarized $^3$He as an effective
neutron target over a wide range of kinematics. And over the past decades
several authors have calculated the effective neutron polarizations in $^3$He
using three-nucleon wave functions and various models of the $N-N$
interaction~\cite{3hepol1}.  These are now well established, and the error
introduced by uncertainty in the wave functions are small.

Other nuclear effects which can influence the experimental asymmetry for a
neutron bound inside $^3$He include, Fermi motion, off-shell effects, meson
exchange currents, delta isobar contributions and $\pi^-$ final state
interactions. The exclusive nature of the process, the selected kinematics such
as high $Q^2$, large recoil momentum and a complete coverage of the azimuthal
angle $\phi$ ensures that corrections due to these nuclear effects will be
small and can be modeled effectively.

\begin{table}[!ht]
\centering
\begin{tabular}{|c|c|}
\hline
Target                       & $^3$He              \\\hline 
Length                       & 40 cm               \\\hline          
Target Polarization          & $\sim$60\%          \\\hline 
Target Spin Flip             & $\leq$20 mins       \\\hline 
Target Dilution              & 90\%                \\\hline
Effective Neutron            & 86.5\%              \\\hline
Target Polarimetry Accuracy  & $\sim$ 3\%          \\\hline
\end{tabular}
\caption{\footnotesize{Key Parameters of the $\mathrm{^{3}He}$
    target.}\label{table:target}}
\end{table} 

The $\mathrm{^{3}He}$ polarization direction is held by three sets of Helmholtz
coils with a 25~Gauss magnetic field. Both the transverse and longitudinal
directions can be provided by rotating the magnetic field. The
$\mathrm{^{3}He}$ gas with density of about 10~atm (at $0^{\circ}$) is stored
in a 40~cm target cell made of thin glasses. With a 15~$\mu A$ electron beam,
the neutron luminosity can be as high as $10^{36} cm^{-2}s^{-1}$. In-beam
polarization of 60\% was archived during the E06-110 experiment. Two kinds of
polarimetry, NMR and EPR, were used to measure the polarization with relative
5\% precision. We have planed to improve the accuracy of the measurement to
reach 3\%.

The target spin will be reversed for every 20 minutes by using the RF AFP
technique. The additional polarization loss due to the spin reversal was kept
at $<10~\%$ which has been taken into account in the overall 60\% in-beam
polarization. A new method for spin reversal using field rotation has been
tested and was able to eliminate the polarization loss. Such an improvement
will enable us to perform the spin-reversal in few minutes to reduce the
target-spin-correlated systematic errors. The key parameters of the
$\mathrm{^{3}He}$ target are summarized in Table~\ref{table:target}.
  
A collimator, similar to the one used in the E06-110, will be placed next to
the target cell window to minimize the target cell contamination and to reduce
the event rate. Several calibration targets will also be installed in this
target system, including a multi-foil $^{12}$C for optics study, a BeO target
for beam tuning, and a reference target cell for dilution study and other
calibration purposes.
  
\subsection {SoLID Spectrometer and Detectors} 

The solenoid magnet for SoLID will be based on the CLEO-II magnet built by
Cornell University. The magnet is 3 meters long with the out diameter of 3
meters and the inner diameter of 1 meter. The field strength is greater than
1.35 Tesla with integrated BDL of 5 Tesla-meters. The fringe field at the front
end after shielding is less than 5 Gauss. In the SIDIS-configuration, the
CLEO-II magnet provides 2$\pi$ acceptance in the azimuthal angle ($\phi$) and
covers the polar angle ($\theta$) from 8$^{\circ}$ up to 24$^{\circ}$. The
momentum acceptance is between 0.8 and 7.5~GeV/c and the momentum resolution is
about 2\%.

\begin{figure}[!ht]
 \begin{center}
  \includegraphics[width=0.8\textwidth]{./figures/SoLID_SIDIS_setup.pdf}
   \caption[The Detector Layout of the SoLID-SIDIS
     configuration]{\footnotesize{The Detector Layout of the SoLID-SIDIS
       configuration. The detector system includes six Gas Electron Multiplier
       (GEM) planes for charged particle tracking, two Scintillator Pad
       Detectors (SPD) followed by two Shashlyk sampling EM Calorimeters (EC)
       for energy measurement and particle identification, a Light Gas
       \v{C}erenkov Detector (LGC) for e-$\pi^{\pm}$ separation, a Heavy Gas
       \v{C}erenkov Detector (HGC) for $\pi^{\pm}$-$K^{\pm}$ separation, as
       well as a Multi-gap Resistive Plate Chamber (MRPC) for timing
       measurement. The first four GEM trackers, the first SPD (i.e. LASPD) and
       EC (i.e. LAEC) form the large-angle detection system for electron
       measurement. The forward-angle detection system, to measure electron and
       hadrons, is composed of all six GEM trackers, LGC, HGC, MRPC, the second
       SPD (i.e. FASPD) and the second EC (FAEC). The photon-detection in the
       large-angle is given by the veto-signal of the SPD in coincidence with
       the EC signal, where the photons in the forward-angle system will be
       triggered by the EC signal plus the veto-signals of LGC, SPD, and
       MRPC.}}
   \label{solid_sidis}
 \end{center}
\end{figure}

The layout of the SoLID detectors in the SIDIS-configuration is shown in
Fig.~\ref{solid_sidis}. The detector system is divided into two regions for the
forward-angle (FA) detection and the large-angle (LA) detection. Six Gas
Electron Multiplier (GEM) tracking chambers will be used for charged particle
tracking, where only the first four of them will be used for the large-angle
detection. In each region, a Shashlyk-type sampling EM calorimeter (LAEC or
FAEC) will measure the particle energy and identify electrons from hadrons. A
scintillator-pad detector (LASPD and FASPD) will be installed in front of each
EC to reject photons and provide timing information. The forward-angle
detectors will detect both the electrons and hadrons (mainly $\pi^{\pm}$). A
light-gas \v{C}erenkov detector (LGC) and a heavy-gas \v{C}erenkov detector
(HGC) will perform the $e/\pi^{\pm}$ and $\pi^{\pm}/K^{\pm}$ separation,
respectively. The Multi-gas Resistive Plate Chamber (MRPC) will provide a
precise timing measurement and serve as a backup of the FASPD on photon
rejection. A more detailed discussion of the design, simulation, prototype-test
of each detector is given in the SoLID preliminary conceptual design report
(pCDR)~\cite{solid_pcdr}.

Table~\ref{table:key_par_sidis_dvcs} summarizes the key parameters of the
detector system in the SIDIS configuration for both the SIDIS and DEMP
measurements.
\begin{table}\centering
\begin{tabular}{|c|c|c|c|c|}
\hline
Experiments                & SIDIS                    & DEMP  \\\hline
Reaction channel           & $\vec{n}(e,e'\pi^{\pm})X$ & $\vec{n}(e,e'\pi^{-}p)$	\\\hline
Target                     & $^3$He                   &same 	\\\hline
Unpolarized luminosity     & $\sim10^{37}$ cm$^{-2}$s$^{-1}$ per nucleon & same	\\\hline 
Momentum coverage          & 0.8-7.5 (GeV/c)          &same 	\\\hline
Momentum resolution        & $\sim$2\%                & same\\\hline
Azimuthal angle coverage   & 0$^{\circ}$ ~360$^{\circ}$ & same	\\\hline
Azimuthal angle resolution & 5 mr                     & same	\\\hline
Polar angle coverage       & 8$^{\circ}$-24$^{\circ}$ for $e$ &  same \\\hline
Polar angle coverage       & 8$^{\circ}$-14.8$^{\circ}$ for $\pi^{\pm}$  &  same 	\\\hline
                           &                          & 8$^{\circ}$-24 $^{\circ}$ for $p$ \\\hline
                           &                          & 24$^{\circ}$-65$^{\circ}$ for $p$ with recoil detector         \\\hline
Polar angle resolution     & 0.6 mr                   & same	\\\hline
Target Vertex resolution   & 0.5~cm                   & same \\\hline
 Energy resolution on ECs  & 5\%$\sim$10\%            & same   \\\hline
Trigger type               & Double Coincidence $e^-+\pi^{\pm}$ & Triple Coincidence $e^-+\pi^{-}+p$\\\hline
Expected DAQ rates         &  $<$100 kHz              &  same online ($<$30 Hz offline)\\\hline
Main Backgrounds           & (e,e'K$^\pm$)            &(e,e'$\pi^{\pm}$) \\
                           &   Accidental Coincidence & Accidental Coincidence	\\\hline
Key requirements           &  Radiation hardness      & Radiation hardness	\\
                           &  Kaon Rejection          & Proton Detection	\\
                           &  DAQ                     &       \\
                        \hline
\end{tabular}
\caption{\footnotesize{Summary of Key Parameters for DEMP Measurement compared
    with SIDIS Experiments.}}\label{table:program_summary}
\label{table:key_par_sidis_dvcs}
\end{table} 

\subsection{A Proton Recoil Detector}

In the SoLID-SIDIS detector system, protons can be isolated from other hadronic
events by using the time-of-fly (TOF) information which requires the timing to
be as good as {\bf 100} ps ({\bf Check it!}).

\subsection{Trigger Design}
In E12-10-006, the online production trigger will be the double-coincidence of
the scattered electrons and hadrons. One will use the particle identification
detectors, such as LGC, HGC and ECs, during the offline analysis to select
$\pi^{\pm}$ out from other hadrons. The DEMP events will be identified with the
triple-coincidence trigger of the scattered electron, $\pi^{-}$ and proton. We
will use the same online trigger as the SIDIS one, and hence the new experiment
will share the same data set as E12-10-006. However, a new trigger type will be
added to the DAQ system to record recoil proton events, and we will perform the
offline analysis to isolate the triple-coincidence events..

The proton trigger will be produced in two regions, the new proton
recoil detector and the standard SoLID timing detectors (e.g., MRCP and LASPD).

The actual trigger design will be far more complicated, and the detailed
discussion of the trigger and DAQ design has been given in the SoLID
pCDR~\cite{solid_pcdr}.
