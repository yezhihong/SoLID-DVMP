\section{Scientific Justification
\label{sec:motivation}}

\subsection{Generalized Parton Distributions and Contribution from the Pion
  Pole}

In recent years, much progress has been made in the theory of generalized
parton distributions (GPDs).  Unifying the concepts of parton distributions and
of hadronic form factors, they contain a wealth of information about how quarks
and gluons make up hadrons. The key difference between the usual parton
distributions and their generalized counterparts can be seen by representing
them in terms of the quark and gluon wavefunctions of the hadron.  While the
usual parton distributions are obtained from the squared hadron wavefunction
representing the probability to find a parton with specified polarization and
longitudinal momentum fraction $x$ in the fast moving hadron (Fig. 1a), GPDs
represent the interference of different wavefunctions, one where the parton has
momentum fraction $x+\xi$ and one where this fraction is $x-\xi$ (Fig. 1b).
GPDs thus correlate different parton configurations in the hadron at the
quantum mechanical level.  A special kinematic regime is probed in deep
exclusive meson production, where the initial hadron emits a quark-antiquark or
gluon pair (Fig. 1c).  This has no counterpart in the usual parton
distributions and carries information about $q\bar{q}$ and $gg$-components in
the hadron wavefunction.

\begin{figure}[hbtp!]
\begin{center}
\includegraphics[height=6cm]{./figures/pdist_gpd_comparo.pdf}
\end{center}
\caption{\label{fig:pdis_gpd_comparo}
\footnotesize{
(a) Usual parton distribution, representing the probability to find a parton
with momentum fraction $x$ in the nucleon.
(b) GPD in the region where it represents the emission of a parton with
momentum fraction $x+\xi$ and its reabsorption with momentum fraction $x-\xi$.
(c) GPD in the region where it represents the emission of a quark-antiquark
pair, and has no counterpart in the usual parton distributions.
This figure has been adapted from Ref. \cite{Di00}.}
}
\end{figure}

Apart from the momentum fraction variables $x$ and $\xi$, GPDs depend on the
four momentum transfer $t$.  This is an independent variable, because the
momenta $p$ and $p'$ may differ in either their longitudinal or transverse
components.  GPDs thus interrelate the longitudinal and transverse momentum
structure of partons within a fast moving hadron.

In order to access the physics contained within GPDs, one is restricted to the
hard scattering regime.  An important feature of hard scattering reactions is
the possibility to separate clearly the perturbative and non-perturbative stages
of the interaction.  Qualitatively speaking, the presence of a hard probe
allows one to create small size quark-antiquark and gluon configurations, whose
interactions are described by perturbative QCD (pQCD).  The non-perturbative
stage of the reaction describes how the hadron reacts to this configuration, or
how this probe is transformed into hadrons.  This separation is the so-called
factorization property of hard reactions.  Deep Exclusive Meson
electro-Production (DEMP) was first shown to be factorizable in
Ref. \cite{Co97}.  This factorization applies when the virtual photon is
longitudinally polarized, which is more probable to produce a small size
configuration compared to a transversely polarized photon.

GPDs are universal quantities and reflect the structure of the nucleon
independently of the reaction which probes the nucleon.  At leading twist-2
level, the nucleon structure information can  be parameterized in terms of four
quark chirality conserving GPDs, denoted $H$, $E$, $\tilde{H}$ and $\tilde{E}$.
$H$ and $E$ are summed over quark helicity, while $\tilde{H}$ and $\tilde{E}$
involve the difference between  left and right handed quarks.  $H$ and
$\tilde{H}$ conserve the helicity of the proton, while $E$ and $\tilde{E}$
allow for the possibility that the proton helicity is flipped.  Because quark
helicity is conserved in the hard scattering regime, the produced meson acts as
a helicity filter.  In particular, leading order QCD predicts that vector meson
production is sensitive only to the unpolarized GPDs, $H$ and $E$, whereas
pseudoscalar meson production is sensitive only to the polarized GPDs,
$\tilde{H}$ and $\tilde{E}$.  In contrast, deeply virtual Compton scattering
(DVCS) depends at the same time on both the polarized ($\tilde{H}$ and  
$\tilde{E}$) and the unpolarized ($H$ and $E$) GPDs.  This makes DEMP
reactions complementary to the DVCS process, as it
provides an additional tool to disentangle the different GPDs \cite{Go01}.

Besides coinciding with the parton distributions at vanishing momentum transfer
$\xi$, the GPDs have interesting links with other nucleon structure quantities.
Their first moments are related to the elastic form factors of the nucleon
through model-independent sum rules \cite{Ra00}:
\begin{eqnarray}
\sum_q e_q \int^{+1}_{-1} dx H^q(x,\xi,t) = F_1(t),\\
\sum_q e_q \int^{+1}_{-1} dx E^q(x,\xi,t) = F_2(t),\\
\sum_q e_q \int^{+1}_{-1} dx \tilde{H}^q(x,\xi,t) = G_A(t),\\
\sum_q e_q \int^{+1}_{-1} dx \tilde{E}^q(x,\xi,t) = G_P(t),
\end{eqnarray}
where $e_q$ is the charge of the relevant quark, $F_1(t)$, $F_2(t)$ are the
Dirac and Pauli elastic nucleon form factors, and $G_A(t)$, $G_P(t)$ are the
isovector axial and pseudoscalar nucleon form factors.  The $t$-dependence of
$G_A(t)$ is poorly known, and although $G_P(t)$ is an important quantity, it
remains highly uncertain because it is negligible at the momentum transfer of
$\beta$-decay\cite{Th01}.  Because of partial conservation of the axial current
(PCAC), $G_P(t)$ alone
receives contributions from $J^{PG}=0^{--}$ states\cite{Ma69}, which are the
quantum numbers of the pion, and so $\tilde{E}$ contains an important pion pole
contribution (Fig. 2a).

\begin{figure}[hbtp!]
\begin{center}
\includegraphics[height=4cm]{figures/PCAC_pion_pole.pdf}
\end{center}
\caption{\label{fig:PCAC_pion_pole}
\footnotesize{(a) Pion pole contribution to $G_P(t)$, and hence to $\tilde{E}$.
(b) Pion pole contribution to meson electroproduction at low $-t$.}
}
\end{figure}

Accordingly, Refs. \cite{Pe00,Be01} have adopted the pion pole-dominated
ansatz
\begin{equation}
\tilde{E}^{ud}(x,\xi,t) = F_{\pi}(t)\frac{\theta (\xi>|x|)}{2\xi
}\phi_{\pi}(\frac{x+\xi}{2\xi}),
\label{eqn:Etilde}
\end{equation}
where $F_{\pi}(t)$ is the pion electromagnetic form factor, and $\phi_{\pi}$ is
the pion distribution amplitude.  In \cite{Go10}, to obtain a better
description of the kinematic region accessible until the construction of the
EIC, the perturbative (or one-gluon exchange) expression for $F_{\pi}$ is
replaced with a parameterization based on the experimental form factor,
replresenting an additional soft contribution not (fully) related to
$\tilde{E}$.  Regardless of which expression is used, $\tilde{E}$ cannot be
related to already known parton distributions, and so experimental information
about $\tilde{E}$ via DEMP can provide new information on nucleon structure
which is unlikely to be available from any other source.

\begin{figure}[hbt!]
\begin{center}
\includegraphics[height=5cm]{./figures/atpi_planes.png}
\end{center}
\caption{\label{fig:planes}
\footnotesize{
Scattering and hadronic reaction planes for exclusive $\vec{N}(e,e'\pi)N'$.
$\phi$ is the azimuthal angle between the hadronic reaction plane and the
electron scattering plane, and $\theta_q$ is the scattering angle of the
virtual photon with respect to the incident electron beam.  $\phi_S$ is the
azimuthal angle between the target nucleon polarization and the scattering
plane, and $\beta=(\phi-\phi_s)$ is the angle between the target nucleon
polarization vector and the reaction plane.}}
\end{figure}

\subsection{Single spin asymmetry in exclusive pion electroproduction}

Frankfurt et al. \cite{Fr99} have considered a specific polarization observable
which is the most sensitive observable to probe the spin-flip $\tilde{E}$.
This variable is the single-spin asymmetry for exclusive charged pion
production, $\vec{p}(e,e'\pi^+)n$ or $\vec{n}(e,e'\pi^-)p$, from a transversely
polarized nucleon target, and is defined \cite{Be01} as
\begin{equation} \label{eqn:asy}
A_L^{\perp}=(\int^{\pi}_0 d\beta \frac{d\sigma^{\pi}_L}{d\beta} -
\int^{2\pi}_{\pi} d\beta \frac{d\sigma^{\pi}_L}{d\beta})
(\int^{2\pi}_0 d\beta \frac{d\sigma^{\pi}_L}{d\beta})^{-1},
\end{equation}
where $d\sigma^{\pi}_L$ is the exclusive charged pion electroproduction cross
section using longitudinally polarized photons and $\beta$ is the angle between
the nucleon polarization vector and the reaction plane
(Fig.~\ref{fig:planes}). 

This asymmetry is related to the parton-helicity-conserving part of the
scattering process and is sensitive to the interference between $\tilde{H}$ and
$\tilde{E}$ \cite{Fr99,Di05}:
\begin{equation} \label{eqn:asy2}
A_L^{\perp}=\frac{\sqrt{-t'}}{m_p}
\frac{\xi\sqrt{1-\xi^2}\ \mathrm{Im}(\tilde{E}^*\tilde{H})}
{(1-\xi^2)\tilde{H}^2-\frac{t\xi^2}{4m_p}
\tilde{E}^2-2\xi^2\mathrm{Re}(\tilde{E}^*\tilde{H})}.
\end{equation}
Frankfurt et al. \cite{Fr99} have shown that this asymmetry must vanish if
$\tilde{E}$ is zero.  If $\tilde{E}$ is not zero, the asymmetry will display a
$\sin\beta$ dependence.  Their predicted asymmetry using the
$\tilde{E}$ ansatz from Ref. \cite{Va99} is shown in
Fig. \ref{fig:frankfurt_atpi}.  This calculation is $Q^2$-independent,
depending only on how well the soft contributions cancel in the asymmetry.

\begin{figure}[hbt!]
\begin{center}
\includegraphics[height=7cm]{./figures/frankfurt_atpi.pdf}
\end{center}
\caption{\label{fig:frankfurt_atpi}
\footnotesize{
Transverse single-spin asymmetry for the longitudinal electroproduction of
$\pi^+n$ and $\pi^+\Delta^0$ at different values of $t$ [indicated on the
curves in GeV$^2$].  The asymmetry drops to zero at the parallel kinematic
limit, which is different for each $t$ value, because the definition of $\beta$
is ill-defined at this point.  This figure is taken from
Ref.~\protect{\cite{Fr00}}.
}}
\end{figure}

It seems likely that a precocious factorization of the meson production
amplitude into three parts -- the overlap integral between the photon and pion
wave functions, the hard interaction, and the GPD -- will lead to a precocious
scaling of $A_L^{\perp}$ as a function of $Q^2$ at moderate $Q^2\sim 2-4$
GeV$^2$ \cite{Fr99}.  This precocious scaling arises from the fact that higher
twist corrections, which are expected to be significant at low $Q^2$, will
likely cancel when one examines the ratio of two longitudinal observables.  In
contrast, the onset of scaling for the absolute cross section is only expected
for much larger values of $Q^2>10$ GeV$^2$.

\begin{figure}[hbt!]
\begin{center}
\includegraphics[height=10cm, angle=90]{./figures/belitsky2.pdf}
\end{center}
\caption{\label{fig:belitsky_atpi}
\footnotesize{
Calculation of the longitudinal photon transverse nucleon spin asymmetry
including twist-four corrections by A. Belitsky \cite{belitsky} at $-t=0.3$
GeV$^2$, $Q^2$=4 GeV$^2$.  The red curves are the leading order calculation,
while the black curves have twist-four power effects taken into account.  While
the cross section is very sensitive to these corrections, the transverse spin
asymmetry is stable.}}
\end{figure}

This point is made clear in Fig. \ref{fig:belitsky_atpi}.  This figure shows
renormalon model calculations \cite{belitsky} of both the asymmetry and the
longitudinal cross section at $Q^2=4$ GeV$^2$.  While the magnitude of the
cross section changes significantly when taking into account the twist-four
corrections, $A_L^{\perp}$ is essentially insensitive to them and displays the
expected precocious scaling.  The relatively low value of $Q^2$ for the
expected onset of precocious scaling is important, because it should be
experimentally accessible at Jefferson Lab.  This places $A_L^{\perp}$ among
the most important GPD measurements that can be made in the meson scalar.  If
precocious scaling cannot be experimentally demonstrated in this ratio of two
cross sections, then it may not be possible to determine GPDs from DEMP data.

Refs. \cite{Go01} and \cite{Fr00} also point out that the study of the
transverse target single-spin asymmetry versus $t$ is important for the
reliable extraction of the pion form factor from electroproduction experiments
(Fig. 2b).  Investigations of hard exclusive $\pi^+$ electroproduction using a
pQCD factorization model \cite{Ma99,Ca90} find that at $x_B=0.3$ and
$-t=-t_{min}$, the pion pole contributes about 80\% of the longitudinal cross
section.  Since the longitudinal photon transverse single-spin asymmetry is an
interference between
pseudoscalar and pseudovector contributions, its measurement would help
constrain the non-pole pseudovector contribution, and so assist the more
reliable extraction of the pion form factor.  The upper $Q^2=6$ GeV$^2$ limit
of the approved pion form factor measurements in the JLab 12 GeV program
\cite{12GeV} is dictated primarily by the requirement $-t_{min}<0.2$ GeV$^2$,
to keep non-pion pole contributions to $\sigma_L$ at an acceptable level
\cite{Ca90}.  Transverse target single-spin asymmetry studies versus $t$ may
eventually allow, with theoretical input, the use of somewhat larger $-t$ data
for pion form factor measurements, ultimately extending the $Q^2$-reach of pion
form factor data acquired with JLab 12 GeV beam.  Thus, measurements of
the transverse single-spin asymmetry are a logical step in the support of the
pion form factor program.

\subsection{The Complementarity of Separated and Unseparated Asymmetry
  Measurements}

The reaction of interest is $^3$He$(e,e'\pi^-)p(pp)_{sp}$.
The measurement of the transverse single-spin asymmetry requires the detection
of the $\pi^-$ in non-parallel kinematics.  It is the component of the target
polarization parallel to $\hat{q}\times\hat{p_{\pi}}$ that is important, and
this direction is uniquely defined only in non-parallel kinematics.

Experimentally, the angle between the target polarization and the reaction
plane, $\beta$, and the angle between the scattering and reaction planes,
$\phi$, are not independent.  If the target polarization is at some angle,
$\phi_s$, relative to the scattering plane, then $\beta = \phi-\phi_s$.  

The polarized nucleon cross section can be expressed \cite{Di05,hermes-thesis} 
in terms of these variables as:
\begin{equation}
\begin{split}\label{eqn:sigtarg}
d\sigma_{UT}(\phi,\phi_s) =  \sum_k d\sigma_{UT_k}(\phi,\phi_s) \;
= -\frac{P_\perp \cos\theta_q}{\sqrt{1-\sin^2\theta_q \sin^2\phi_s}} \biggl[\; 
  &\; \sin\beta \; \rm{Im}(\sigma^{+-}_{++} + \epsilon\sigma^{+-}_{00})\\
 +&\; \sin\phi_s \; \sqrt{\epsilon(1+\epsilon)}\; \rm{Im}(\sigma^{+-}_{+0})\\
 +&\; \sin(\phi+\phi_s) \; \frac{\epsilon}{2} \; \rm{Im}(\sigma^{+-}_{+-})\\
 +&\; \sin(2\phi-\phi_s) \; \sqrt{\epsilon(1+\epsilon)} \; \rm{Im}(\sigma^{-+}_{+0}) \\
 +&\; \sin(3\phi-\phi_s) \; \frac{\epsilon}{2} \; \rm{Im}(\sigma^{-+}_{+-})\;\; \biggr] ,
\end{split}
\end{equation}
where the $\cos\theta_q$ factor is needed to convert the target $P_\perp$
relative to the lepton beam to that relative to the virtual photon
(in accordance with the Trento convention), and much smaller factors
proportional to $\sin\theta_q$ have been neglected for clarity.  
The $\sigma^{ij}_{mn}$ correspond to nucleon polarizations
$ij=(+\frac{1}{2},-\frac{1}{2})$ and photon polarizations $mn=(+1,0,-1)$, and
$\sigma_T=\frac{1}{2}(\sigma^{++}_{++}+\sigma^{--}_{++})$,
$\sigma_L=\sigma^{++}_{00}$ are the usual unpolarized transverse and
longitudinal cross sections.

From the above equation, it is clear that to extract $A_L^{\perp}$
it is necessary to first isolate the 
$\sin \beta$ Fourier component of the polarized nucleon cross section.
Once that has been accomplished, one must then separate the desired
$\sigma^{+-}_{00}$ term 
from the $\sigma^{+-}_{++}$ term via a Rosenbluth-type separation.  All of the
other polarized terms can be distinguished from their azimuthal dependences,
without need of a Rosenbluth separation \cite{Di05}.

It has not yet been possible to perform an experiment to measure $A_L^{\perp}$.
The conflicting experimental requirements of transversely polarized target,
high luminosity, L--T separation, and closely controlled systematic
uncertainty, make this an exceptionally challenging observable to measure.  The
SHMS+HMS is the only facility with the necessary resolution and systematic
error control to allow a measurement of $A_L^{\perp}$.  However, the beamtime
required to do a good measurement with current polarized target technology is
in the range of 10$^3$ days.  To minimize the beamtime required, PR12-12-005
\cite{atpi39}
proposed the use of a next generation, externally polarized, continuous flow,
high luminosity $^3$He target based on a large volume polarizer and compressor
developed at the University of New Hampshire.  The science case
for this measurement was favorably reviewed by PAC39, and they encouraged the
continued development of the target technology.  Although the New Hampshire
group is making continued progress on the development of the target, there is
no timeline for its actual implementation at Jefferson Lab.

The most closely related measurement, of the transverse single-spin asymmetry
in exclusive $\pi^+$ electroproduction without an L--T separation, was
published by the HERMES Collaboration in 2010 \cite{hermes10}.  Their data were
obtained for average values of $\langle x_B \rangle =0.13$, $\langle Q^2 \rangle
=2.38$ GeV$^2$ and $\langle t' \rangle = -0.46$ GeV$^2$, subject to the
criterion $W^2>10$ GeV$^2$.  The six Fourier amplitudes in terms of the
azimuthal angles $\phi$, $\phi_s$ of the pion-momentum and proton-polarization
vectors relative to the lepton scattering plane were determined.  Of these, at
leading twist only the $\sin\beta_{UT}$ Fourier amplitude receives a
contribution from longitudinal photons.  If one assumes that longitudinal
contributions dominate, these $A_{UT}^{sin(\phi-\phi_s)}$ values can be
compared to GPD models for $\tilde{E}$, $\tilde{H}$.

\begin{figure}[hbt!]
\begin{center}
\includegraphics[height=7cm]{./figures/hermes_Aut.pdf}
\end{center}
\caption{\label{fig:hermes_aut}
\footnotesize{
Predictions by Goloskokov and Kroll for the $\sin(\phi-\phi_s)$ moment of
$A_{UT}$ in the handbag approach, in comparison to the data from HERMES at
$Q^2=2.45$ GeV$^2$, $W=3.99$ GeV.  Please note that the HERMES data
follow the Trento convention, while the Eqns. \ref{eqn:asy}, \ref{eqn:asy2} and
Figs. \ref{fig:frankfurt_atpi}, \ref{fig:belitsky_atpi} do not, leading to a
normalization difference of $-\pi/2$ \cite{hermes-thesis}.  The independent
variable is $-t'=|t-t_{min}|$.  Dashed line: contribution from longitudinal
photons only.  Solid line: full calculation including both transverse and
longitudinal photons.  This figure is taken from Ref. \protect{\cite{Go10}}.}}
\end{figure}

Because transverse photon amplitudes are suppressed by $1/Q$, at very high
$Q^2$ it is safe to assume that all observed meson production is due to
longitudinal photons.  At the lower $Q^2$ typical of the JLab and HERMES
programs, however, this is not the case.  Handbag-approach
calculations by Goloskokov and Kroll
\cite{Go10} indicate much of the unseparated cross section measured by HERMES
\cite{hermes10} is due to contributions from transversely polarized photons.
In addition, there are contributions to $A_{UT}^{\sin(\phi-\phi_s)}$ from the
interference between two amplitudes, both for longitudinal photons, as well as
transverse photons~\cite{Di05}.
At the amplitude level, the transverse suppression is given by $\mu/Q$, where
$\mu\sim 2$ GeV is a mass parameter given by the pion mass enhanced by the
large ratio between the pion mass and the sum of the $u$ and $d$ current quark
masses (chiral condensate).  For experimentally accessible $Q^2$, hardly any
suppression of the twist-3 contribution is expected.  As indicated in
Fig. \ref{fig:hermes_aut}, the contribution from transverse photons tends to
make the asymmetry smaller.  At the HERMES kinematics, the dilution caused by
transverse photons is about 50\%.  Although the observed unseparated asymmetry
is small, the HERMES data are consistent with GPD models based on the dominance
of $\tilde{E}$ over $\tilde{H}$ at low $-t'$.  Indeed, the change in sign in
the model curves at large $-t'$ in Fig.~\ref{fig:hermes_aut} is due to the
large contribution from $\tilde{E}$ demanded by the data \cite{Go10}.  An
improved measurement of the transverse target spin asymmetry, in particular the
$\sin(\phi-\phi_S)$ modulation, is clearly a high priority.

\begin{figure}[hbt!]
\begin{center}
\includegraphics[height=10cm]{./figures/goloskokov2.png}
\caption{\label{fig:golo_aut}
\footnotesize{
Calculation of the cross section components and $\sin(\phi-\phi_s)$ moment of
the transverse nucleon spin asymmetry $A_{UT}$ in the handbag approach by
Goloskokov and Kroll \cite{GoPC} for kinematics similar to those in
Fig. \ref{fig:belitsky_atpi}.  Our measurement will be at higher
$0.55<\epsilon<0.75$ than the $\epsilon=0.35$ kinematics of this figure,
so the dilution in the asymmetry will be significantly less.}}
\end{center}
\end{figure}

A run-group proposal concurrent with the SoLID transversely polarized $^3$He
SIDIS experiment allows for an unseparated asymmetry measurement to be obtained
on a sooner timescale than the Hall C measurement.  In comparison to the HERMES
measurement, the experiment proposed here will probe higher $Q^2$ and $x_B$,
with much smaller statistical errors over a wider range of $-t$.
SoLID will allow the first measurement for $Q^2>4$ GeV$^2$, where GPD-based
calculations are expected to apply.  Thus, the measurements should be more
readily interpretable than those from HERMES.  Similar measurements using
CLAS-12 and a transversely polarized $^1$H target have been discussed
previously \cite{clas}, but this measurement will allow for smaller statistical
uncertainties, due to SoLID's higher luminosity capabilities.

Handbag model calculations by Goloskokov and Kroll \cite{GoPC} shed further
light on the expected asymmetry dilution.  The bottom panel of
Fig. \ref{fig:golo_aut} shows their predictions for the cross section
components in exclusive charged pion production.  Although their calculations
tend to underestimate the $\sigma_L$ values measured in the JLab $F_{\pi}-2$
experiment \cite{Fpi2}, their model is in reasonable agreement with the
unseparated cross sections \cite{Go10}.  They predict significant transverse
contributions for JLab kinematics.  A comparison of the unseparated asymmetry
at $-t=0.3$ GeV$^2$, $x_B=0.365$ in Fig. \ref{fig:golo_aut} with the separated
longitudinal asymmetry at the same values of $x_B$, $-t$ in
Fig. \ref{fig:belitsky_atpi} indicates a substantial dilution of the
unseparated asymmetry due to transverse photon contributions, similar to that
observed in Fig. \ref{fig:hermes_aut}.

In addition to allowing a measurement at $Q^2>4$ GeV$^2$, a measurement by
SoLID of $A_{UT}^{\sin(\phi-\phi_s)}$ will cover a fairly large range of $-t$,
allowing the asymmetry to be mapped over its full range with good statistical
uncertainties -- from its required zero-value in parallel kinematics, through
its maximum, and then back to near-zero or even positive at larger $-t$.  
The shape of the asymmetry curve versus
$-t$, as well as its maximum value, are critical information for comparison to
GPD-based models.

\subsection{Motivation for and Status of the other Fourier Azimuthal Components
\label{sec:sinphiS}}

An important point is that any model that describes exclusive pion production
will need to describe not only the leading-twist Fourier amplitude
$A_{UT}^{\sin(\phi-\phi_s)}$, but also the other contributions to the
target-spin azimuthal asymmetry listed in Eqn.~\ref{eqn:sigtarg}, providing
additional GPD model constraints.  Like HERMES, we plan to determine the full
set of asymmetries from their azimuthal modulations,
\begin{equation}
A(\phi,\phi_s)=\frac{d\sigma_{UT}(\phi,\phi_s)}{d\sigma_{UU}(\phi)}
=-\sum_k A_{UT}^{\sin(\mu\phi+\lambda\phi_s)_k}\sin(\mu\phi+\lambda\phi_s)_k ,
\label{eqn:six_asym}
\end{equation}
where $d\sigma_{UU}$ is the unpolarized nucleon cross section in terms of the
well-known L, T, LT and TT response functions.  These asymmetries include all
five terms listed in Eqn.~\ref{eqn:sigtarg}, plus a small
$\sin(2\phi+\phi_s)$ term proportional to $\sin\theta_q$ not listed there.

\begin{figure}[hbt!]
\begin{center}
\includegraphics[height=7cm]{./figures/hermes_sinphiS.png}
\vspace*{-5mm}
\end{center}
\caption{\label{fig:hermes_phiS}
\footnotesize{
Data from HERMES for the $\sin(\phi_s)$ moment with a transversely polarized
target at $Q^2=2.45$ GeV$^2$, $W=3.99$ GeV.  The solid line is the prediction
of the handbag calculation by Goloskokov and Kroll under the assumption that
the dominant transversity GPD is $H_T$ and that the other three can be
neglected.  The dashed line is obtained disregarding the twist-3 contribution.
This figure is taken from Ref. \protect{\cite{Go10}}.}}
\end{figure}

While most of the theoretical interest and the primary motivation of our
experiment is the target asymmetry proportional to $\sin\beta$, there is
growing interest in the $\sin(\phi_S)_{UT}$ asymmetry, as it may be
interpretable in terms of the transversity GPDs.  
Independent of a specific dynamical interpretation
(e.g. the handbag approach), the $A_{UT}^{\sin(\phi_S)}$ asymmetry will say
something on the strength of the contributions from transverse photons at
small $-t$:
\begin{equation}
A_{UT}^{\sin(\phi_S)}\sim\mathrm{Im}[M^*_{0+,++}M_{0-,0+} - M^*_{0-,++}M_{0+,0+}],
\end{equation}
where the helicities are in the order: pion neutron, photon proton
\cite{Go10}.  In contrast to the $\sin(\phi-\phi_S)$ modulation, which
has contributions from LL and TT interferences, the $\sin(\phi_s)$ modulation
measures only the LT interference.  The first term is proportional to $t'$, as
is forced by angular momentum conservation, while the second one is not forced
to vanish \cite{GoPC}.  Indeed, HERMES measured the $\sin(\phi_s)$ modulation
to be large and apparently nonzero at $-t'=0$ (Fig.~\ref{fig:hermes_phiS}).
Hence, both the amplitudes $M_{0-,++}$ and $M_{0+,0+}$ must be large, giving the
first clear signal for strong contributions from transversely polarized photons
at rather large values of $W$ and $Q^2$ \cite{Go10}.  This is very interesting
in its own right.

\begin{figure}[hbt!]
\begin{center}
\includegraphics[height=7cm]{./figures/hermes_sin2phiphiS.png}
\vspace*{-5mm}
\end{center}
\caption{\label{fig:hermes_sin2phiphiS}
\footnotesize{
Data from HERMES for the $\sin(2\phi-\phi_s)$ moment with a transversely
polarized target at $Q^2=2.45$ GeV$^2$, $W=3.99$ GeV.  The solid line is the
prediction of the handbag calculation by Goloskokov and Kroll
\protect{\cite{Go10}}.}}
\end{figure}

The calculation by Goloskokov and Kroll \cite{Go10} for the
$\sin(2\phi-\phi_S)$ modulation measured by HERMES is shown in
Fig.~\ref{fig:hermes_sin2phiphiS}.  Both the experimental values, as well as
the calculation, are small.  The agreement is fairly good, except that the
change of sign at large $-t'$ is not reproduced by the model.  However, this
observable is given by an interference between longitudinal amplitudes and
transversal other than $M_{0-,++}$, so an improvement of this moment probably
requires a detailed modeling of the small transverse amplitudes contributing.
The other three moments, $\sin(\phi+\phi_S)$ $\sin(2\phi+\phi_S)$ and
$\sin(3\phi-\phi_S)$, are only fed by transverse-transverse interference terms
and are therefore small in the handbag approach of Ref.~\cite{Go10}.

\begin{figure}[htb!]
\begin{center}
\includegraphics[height=20cm]{./figures/gk16.png}
\caption{\label{fig:gk16}
\footnotesize{
$\sin(\mu\phi+\lambda\phi_s)_k$ moments of the transverse nucleon spin
  asymmetry $A_{UT}$ calculated in the handbag approach by Goloskokov and Kroll
  \cite{GoPC} for the kinematics of this experiment.
Solid black: $Q^2=4.107$ GeV$^2$, $W=$3.166 GeV;  
Long-dash red: $Q^2=5.138$ GeV$^2$, $W=$2.796 GeV;  
Dash-dot green: $Q^2=6.049$ GeV$^2$, $W=$2.718 GeV;  
Short-dash blue: $Q^2=6.894$ GeV$^2$, $W=$2.562 GeV.}}
\end{center}
\end{figure}

\begin{figure}[hbt!]
\begin{center}
\includegraphics[height=20cm]{./figures/hermes_6asym.png}
\end{center}
\caption{\label{fig:hermes6}
\footnotesize{
HERMES data for the set of six Fourier amplitudes $A_{UT}^{\sin(\phi-\phi_S)}$
describing the sine modulations of the single-spin azimuthal asymetry.  The
error bars (bands) represent the statistical (systematic) uncertainties.  The
results receive an additional 8.2\% scale uncertainty corresponding to the
target polarization uncertainty.  This figure is taken from
Ref. \protect{\cite{hermes10}}.}}
\end{figure}

A calculation for the kinematics of this experiment by S.V. Goloskokov and
P. Kroll \cite{Go10,GoPC,Go11} for the amplitudes of the five azimuthal
modulations listed in Eqn.~\ref{eqn:sigtarg} is shown in Fig.~\ref{fig:gk16}.
It clearly shows that the two asymmetries of greatest physics interest
dominate, while the other asymmetries are much smaller.  This is consistent
with the HERMES result, which found the four asymmetries $\sin(2\phi-\phi_s)$, 
$\sin(\phi+\phi_s)$, $\sin(3\phi-\phi_s)$, $\sin(2\phi+\phi_s)$, to be small
over most of the measured $t$ range (Fig.~\ref{fig:hermes6}).  
This is good news, as large expected values for these asymmetries would
complicate the extraction of the most-valued Fourier components.
