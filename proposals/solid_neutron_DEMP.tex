% ----------------------------------------------------------------------
% ----------------------------------------------------------------------
\documentclass{article}
\usepackage{color,rotate,graphics,graphicx,epsfig,amssymb,latexsym,epsfig,graphicx,authblk}
\usepackage{booktabs}
\usepackage{epsfig}
\usepackage{amsmath}
\usepackage{rotating}
\usepackage{caption}
\usepackage{subfig}
\usepackage{graphicx}
\usepackage{booktabs}
\usepackage{epstopdf}
\usepackage{url} 
\usepackage{listings}
\usepackage{cite}
\usepackage{upquote}
\usepackage{xcolor}
\usepackage{pdfpages}
\usepackage[section]{placeins}
\usepackage[margin=1in]{geometry}
\usepackage{hyperref}

%-----------------------------------------------------------------------
%\textwidth 175mm \textheight 236mm \topmargin -1in \oddsidemargin -1in
%\textwidth 5.5in \textheight 9in \topmargin -1in \oddsidemargin -1in
%\evensidemargin 1in  
%-----------------------------------------------------------------------

\catcode`\@=11

% ----------------------------------------------------------------------
% ----------------------------------------------------------------------

\def \be  {\begin{equation}}
\def \ee  {\end{equation}}
\def \beq  {\begin{equation}}
\def \eeq  {\end{equation}}
\def \ba  {\begin{eqnarray}}
\def \ea  {\end{eqnarray}}
\def \baa {\begin{eqnarray*}}
\def \eaa {\end{eqnarray*}}
\def \lab #1 {\label{#1}}

\newcommand \ci [1] {\cite{#1}}
\newcommand \bi [1] {\bibitem{#1}}
\newcommand\re[1]{(\ref{#1})}

\def \qqquad {\qquad\quad}
\def \qqqquad {\qquad\qquad}

\def \matrix #1 {\left(\begin{array}{cc} #1 \end{array}\right)}
\def \Tr {\mathop{\rm Tr}\nolimits}
\def \tr {\mathop{\rm tr}\nolimits}
\def \res{\mathop{\rm res}\nolimits}
\def \e  {\mathop{\rm e}\nolimits}

\newcommand{\inclfig}[2]{\mbox{\epsfxsize=#1cm \epsfbox{#2.ps}}}
\newcommand{\insertfig}[2]{\mbox{\epsfxsize=#1cm \epsfbox{#2.eps}}}
\newcommand{\Asym}{\mathop{\mbox{\bf A}}}
\newcommand{\Sym}{\mathop{\mbox{\bf S}}}
\newcommand{\ft}[2]{{\textstyle\frac{#1}{#2}}}
\newcommand\lr[1]{{\left({#1}\right)}}
\newcommand \widebar [1] {\overline{#1}}
\newcommand\bin[2]{\left({#1}\atop{#2}\right)}
\newcommand \vev [1] {\langle{#1}\rangle}
\newcommand \VEV [1] {\left\langle{#1}\right\rangle}
\newcommand \ket [1] {|{#1}\rangle}
\newcommand \bra [1] {\langle {#1}|}
\newcommand \partder [1] {{\partial \over\partial #1}}
\newcommand{\as}{\ifmmode\alpha_{\rm s}\else{$\alpha_{\rm s}$}\fi}
\newcommand{\bit}[1]{\mbox{\boldmath$#1$}}
\newcommand{\OO}{\mathop{\otimes}}
\newcommand{\CA}{\mbox{CA}}
\newcommand{\CoA}{\mbox{CoA}}
\newcommand{\CeA}{\mbox{CeA}}
\newcommand{\SSA}{\mbox{SSA}}
\newcommand{\spin}[1]{ \langle\mskip-3mu \langle{#1}\rangle\mskip-3mu\rangle}
\newcommand\pbar{\bar{\psi}}
\newcommand\p{\psi }
\newcommand{\Dperp}{\Delta_\perp}
\newcommand{\Eperp}{\epsilon^{\perp \sigma k}}
\newcommand{\Eperpik}{\epsilon^\perp_{i k}}
\newcommand{\dxi}{\frac{\partial}{\partial \xi}}
\newcommand{\ddxi}{\frac{\partial^2}{\partial \xi^2}}
\newcommand{\du}{\frac{\partial}{\partial u}}

\renewcommand{\labelenumi}{\alph{enumi}.} % Make numbering in the enumerate environment by letter rather than number (e.g. section 6)

\def\Gtilde{\tilde{G}}
\def \CO {{\cal O}}
\def \CP {{\cal P}}
\def \CT {{\cal T}}
\def \CM {{\cal M}}
\def \CK {{\cal K}}
\def \CH {{\cal H}}
\def \CI {{\cal I}}
\def \CV {{\cal V}}
\def \CJ {{\cal J}}
\def \CL {{\cal L}}

\font\cmss=cmss12 \font\cmsss=cmss10 at 11pt
\def\1{\hbox{{1}\kern-.25em\hbox{l}}}
\def\bfZ{\relax{\hbox{\cmss Z\kern-.4em Z}}}

\font\cmss=cmss12 \font\cmsss=cmss10 at 11pt
\def\inbar{\,\vrule height1.5ex width.4pt depth0pt}
\def\IC{\relax\hbox{$\inbar\kern-.3em{\rm C}$}}
\def\IZ{\relax{\hbox{\cmss Z\kern-.4em Z}}}
\def\IR{{\hbox{{\rm I}\kern-.2em\hbox{\rm R}}}}
\def\R{{\tiny \IR}}
\def\IP{{\hbox{{\rm I}\kern-.2em\hbox{\rm P}}}}
\def\II{\hbox{{1}\kern-.25em\hbox{l}}}

\renewcommand{\baselinestretch}{1.3}
% --------------------------------------------------------------------
% --------------------------------------------------------------------

\begin {document}

% --------------------------------------------------------------------
% --------------------------------------------------------------------

\begin{center}

{\Huge A Run-Group Proposal Submitted to PAC 45}

\vspace*{25pt}

{\LARGE\bf
Measurement of Deep Exclusive $\mathrm\pi^-$ Production
using a Transversely Polarized $\mathrm{^{3}He}$ Target
and the SoLID Spectrometer}

\vspace*{3ex}
DRAFT: \today

\vspace*{30pt}

J. Arrington, K. Hafidi, M. Hattawy, P. Reimer, S. Riordan, Z. Ye$^\ast$ \\
{\it Argonne National Laboratory, Physics Division, Argonne, IL, USA}
\vspace*{15pt}

Z. Ahmed$^\ast$, G.R. Ambrose, S. Basnet, R.S. Evans, G.M. Huber$^\dagger$,
W. Li, D. Paudyal, Z. Papandreou\\
{\it University of Regina, Regina, SK, Canada}
\vspace*{15pt}

H. Gao, X. Li, T. Liu, C. Peng, W. Xiong, X.F. Yan, Z. Zhao\\
{\it Duke University, Durham, NC, USA}
\vspace*{15pt}

A. Camsonne,  J-P. Chen, S. Covrig-Dusa, D. Gaskell\\
{\it  Jefferson Lab, Newport News, VA, USA}
\vspace*{15pt}

T. Brecelj, M. Mihovilovi\v{c}, S. \v{S}irca, S. \v{S}tajner\\
{\it Jo\v{z}ef Stefan Institute and University of Ljubljana, Slovenia}
\vspace*{15pt}

K. Aniol\\
{\it California State University -- Los Angeles, Los Angeles, CA, USA}
\vspace*{15pt}

M. Boer\\
{\it Los Alamos National Laboratory, Physics Division, Los Alamos, NM, USA}
\vspace*{15pt}

D. Dutta, L. Ye\\
{\it Mississippi State University, Mississippi State, MS, USA}
\vspace*{15pt}

C. E. Hyde, F.R. Wesselmann \\
{\it Old Dominion University, Norfolk, VA, USA}
\vspace*{15pt}

P. Markowitz\\
{\it Florida International University, Miami, FL, USA}
\vspace*{15pt}

\clearpage
V. Sulkosky  \\
{\it University of Virginia, Charlottesville, VA, USA}
\vspace*{15pt}

E. Voutier\\
{\it Institut de Physique Nucl\'eaire IN2P3/CNRS, Universit\'e Paris Sud, 
Orsay, France}
\vspace*{15pt}

\end{center}

$^\dagger$ Contact person , $^\ast$ Spokesperson

\vfill\eject

\clearpage

% --------------------------------------------------------------------
% --------------------------------------------------------------------
%\newpage

%\null\vfill
%\hrule

\newpage

\tableofcontents
%\listoffigures
%\listoftables

\newpage
\begin{abstract}

We propose to measure the transverse nucleon, single-spin asymmetries
$A_{UT}(\phi,\phi_S)$ in the exclusive $\vec{n}(e,e'\pi^-)p$ reaction,
during the transversely polarized $^3$He target SIDIS experiment (i.e.
E12-10-006~\cite{solid:e12-10-006}) with SoLID~\cite{solid_pcdr}.  In
principle, as many as six azimuthal modulations of the form
$A_{UT}^{\sin(\mu\phi+\lambda\phi_S)}$ contribute (where $\mu$ and $\lambda$
are integers) \cite{Di05}, but four of these modulations are small and the
physics interest is concentrated in the largest two.

Our primarily goal is the measurement of $A_{UT}^{\sin(\phi-\phi_S)}$.  This
polarization observable has been noted as being sensitive to the spin-flip
generalized parton distribution (GPD) $\tilde{E}$, and factorization studies
have indicated that precocious scaling is likely to set in at moderate $Q^2\sim
2-4$ GeV$^2$, as opposed to the absolute cross section, where scaling is not
expected until $Q^2>10$ GeV$^2$.  Furthermore, this observable has been noted
as being important for the reliable extraction of the charged pion form factor
from pion electroproduction.  Our secondary goal is the measurement of the
$A_{UT}^{\sin(\phi_s)}$ asymmetry, which is sensitive to the higher twist
transversity GPDs, and provides valuable information on transverse photon
contributions at small $-t$.  Both asymmetry datasets are projected to be of
much higher quality than a pioneering measurement by HERMES \cite{hermes10}.
The remaining four $A_{UT}^{\sin(\mu\phi+\lambda\phi_S)}$ modulations will also
be measured, but are expected to be small.

This measurement is complementary to a proposal reviewed by PAC39 \cite{atpi39}
for the SHMS+HMS in Hall C.  The asymmetry that is most sensitive to
$\tilde{E}$ is the longitudinal photon, transverse nucleon, single-spin
asymmetry $A_L^{\perp}(\sin(\phi-\phi_S))$ in exclusive charged pion
electroproduction.  The SHMS+HMS allow the L--T separation needed to reliably
measure this quantity.  However, the limited detector acceptance and the
error-magnification inherent in an L--T separation necessitates the use of a
next generation, externally polarized, continuous flow, high luminosity
$\mathrm{\vec{^3He}}$ target based on a large volume polarizer and compressor
being developed at the University of New Hampshire.

A wide $-t$ coverage is needed to obtain a good understanding of the
asymmetries.  Thus, it has always been intended to complement the SHMS+HMS
$A_L^{\perp}(\sin(\phi-\phi_S))$ measurement with an unseparated
$A_{UT}^{\sin(\phi-\phi_s)}$ measurement using a large solid angle detector.
The high luminosity capabilities of SoLID make it well-suited for this
measurement.  Since an L--T separation is not possible with SoLID, the observed
asymmetry is expected to be diluted by the ratio of the longitudinal cross
section to the unseparated cross section.  This was also true for the
pioneering HERMES measurements, which provided a valuable constraint to models
for the $\tilde{E}$ GPD.  In order to assure a clean measurement of exclusive
$\pi^-$ production, it is required to detect the recoil proton from the
$\vec{n}(e,e'\pi^-)p$ reaction.  We propose to analyze the E12-10-006 event
files off-line to look for $e-\pi^--p$ triple coincidence events in SoLID for
the case where the proton is emitted $8^o<\theta<24^o$.  This has no impact
upon E12-01-006 whatsoever, and yields valuable unseparated asymmetry data.

% $(b)$ the improved asymmetry data that could be obtained over a broader
%kinematic range if a proton recoil detector (PRD) could be constructed to
%detect those additional protons emitted $24^o<\theta<50^o$.  Here, we are
%seeking PAC approval for scenario $(a)$, as well as advice on whether we
%should pursue the design and prototyping of the PRD to enable scenario $(b)$
%under the proviso that  it have minimal impact upon the running of E12-10-006.

\end{abstract}

\newpage
\section{Scientific Justification
\label{sec:motivation}}

\subsection{Generalized Parton Distributions and Contribution from the Pion
  Pole}

In recent years, much progress has been made in the theory of generalized
parton distributions (GPDs).  Unifying the concepts of parton distributions and
of hadronic form factors, they contain a wealth of information about how quarks
and gluons make up hadrons. The key difference between the usual parton
distributions and their generalized counterparts can be seen by representing
them in terms of the quark and gluon wavefunctions of the hadron.  While the
usual parton distributions are obtained from the squared hadron wavefunction
representing the probability to find a parton with specified polarization and
longitudinal momentum fraction $x$ in the fast moving hadron (Fig. 1a), GPDs
represent the interference of different wavefunctions, one where the parton has
momentum fraction $x+\xi$ and one where this fraction is $x-\xi$ (Fig. 1b).
GPDs thus correlate different parton configurations in the hadron at the
quantum mechanical level.  A special kinematic regime is probed in deep
exclusive meson production, where the initial hadron emits a quark-antiquark or
gluon pair (Fig. 1c).  This has no counterpart in the usual parton
distributions and carries information about $q\bar{q}$ and $gg$-components in
the hadron wavefunction.

\begin{figure}[hbtp!]
\begin{center}
\includegraphics[height=6cm]{./figures/pdist_gpd_comparo.pdf}
\end{center}
\caption{\label{fig:pdis_gpd_comparo}
\footnotesize{
(a) Usual parton distribution, representing the probability to find a parton
with momentum fraction $x$ in the nucleon.
(b) GPD in the region where it represents the emission of a parton with
momentum fraction $x+\xi$ and its reabsorption with momentum fraction $x-\xi$.
(c) GPD in the region where it represents the emission of a quark-antiquark
pair, and has no counterpart in the usual parton distributions.
This figure has been adapted from Ref. \cite{Di00}.}
}
\end{figure}

Apart from the momentum fraction variables $x$ and $\xi$, GPDs depend on the
four momentum transfer $t$.  This is an independent variable, because the
momenta $p$ and $p'$ may differ in either their longitudinal or transverse
components.  GPDs thus interrelate the longitudinal and transverse momentum
structure of partons within a fast moving hadron.

In order to access the physics contained within GPDs, one is restricted to the
hard scattering regime.  An important feature of hard scattering reactions is
the possibility to separate clearly the perturbative and non-perturbative stages
of the interaction.  Qualitatively speaking, the presence of a hard probe
allows one to create small size quark-antiquark and gluon configurations, whose
interactions are described by perturbative QCD (pQCD).  The non-perturbative
stage of the reaction describes how the hadron reacts to this configuration, or
how this probe is transformed into hadrons.  This separation is the so-called
factorization property of hard reactions.  Deep Exclusive Meson
electro-Production (DEMP) was first shown to be factorizable in
Ref. \cite{Co97}.  This factorization applies when the virtual photon is
longitudinally polarized, which is more probable to produce a small size
configuration compared to a transversely polarized photon.

GPDs are universal quantities and reflect the structure of the nucleon
independently of the reaction which probes the nucleon.  At leading twist-2
level, the nucleon structure information can  be parameterized in terms of four
quark chirality conserving GPDs, denoted $H$, $E$, $\tilde{H}$ and $\tilde{E}$.
$H$ and $E$ are summed over quark helicity, while $\tilde{H}$ and $\tilde{E}$
involve the difference between  left and right handed quarks.  $H$ and
$\tilde{H}$ conserve the helicity of the proton, while $E$ and $\tilde{E}$
allow for the possibility that the proton helicity is flipped.  Because quark
helicity is conserved in the hard scattering regime, the produced meson acts as
a helicity filter.  In particular, leading order QCD predicts that vector meson
production is sensitive only to the unpolarized GPDs, $H$ and $E$, whereas
pseudoscalar meson production is sensitive only to the polarized GPDs,
$\tilde{H}$ and $\tilde{E}$.  In contrast, deeply virtual Compton scattering
(DVCS) depends at the same time on both the polarized ($\tilde{H}$ and  
$\tilde{E}$) and the unpolarized ($H$ and $E$) GPDs.  This makes DEMP
reactions complementary to the DVCS process, as it
provides an additional tool to disentangle the different GPDs \cite{Go01}.

Besides coinciding with the parton distributions at vanishing momentum transfer
$\xi$, the GPDs have interesting links with other nucleon structure quantities.
Their first moments are related to the elastic form factors of the nucleon
through model-independent sum rules \cite{Ra00}:
\begin{eqnarray}
\sum_q e_q \int^{+1}_{-1} dx H^q(x,\xi,t) = F_1(t),\\
\sum_q e_q \int^{+1}_{-1} dx E^q(x,\xi,t) = F_2(t),\\
\sum_q e_q \int^{+1}_{-1} dx \tilde{H}^q(x,\xi,t) = G_A(t),\\
\sum_q e_q \int^{+1}_{-1} dx \tilde{E}^q(x,\xi,t) = G_P(t),
\end{eqnarray}
where $e_q$ is the charge of the relevant quark, $F_1(t)$, $F_2(t)$ are the
Dirac and Pauli elastic nucleon form factors, and $G_A(t)$, $G_P(t)$ are the
isovector axial and pseudoscalar nucleon form factors.  The $t$-dependence of
$G_A(t)$ is poorly known, and although $G_P(t)$ is an important quantity, it
remains highly uncertain because it is negligible at the momentum transfer of
$\beta$-decay\cite{Th01}.  Because of partial conservation of the axial current
(PCAC), $G_P(t)$ alone
receives contributions from $J^{PG}=0^{--}$ states\cite{Ma69}, which are the
quantum numbers of the pion, and so $\tilde{E}$ contains an important pion pole
contribution (Fig. 2a).

\begin{figure}[hbtp!]
\begin{center}
\includegraphics[height=4cm]{figures/PCAC_pion_pole.pdf}
\end{center}
\caption{\label{fig:PCAC_pion_pole}
\footnotesize{(a) Pion pole contribution to $G_P(t)$, and hence to $\tilde{E}$.
(b) Pion pole contribution to meson electroproduction at low $-t$.}
}
\end{figure}

Accordingly, Refs. \cite{Pe00,Be01} have adopted the pion pole-dominated
ansatz
\begin{equation}
\tilde{E}^{ud}(x,\xi,t) = F_{\pi}(t)\frac{\theta (\xi>|x|)}{2\xi
}\phi_{\pi}(\frac{x+\xi}{2\xi}),
\label{eqn:Etilde}
\end{equation}
where $F_{\pi}(t)$ is the pion electromagnetic form factor, and $\phi_{\pi}$ is
the pion distribution amplitude.  In \cite{Go10}, to obtain a better
description of the kinematic region accessible until the construction of the
EIC, the perturbative (or one-gluon exchange) expression for $F_{\pi}$ is
replaced with a parameterization based on the experimental form factor,
replresenting an additional soft contribution not (fully) related to
$\tilde{E}$.  Regardless of which expression is used, $\tilde{E}$ cannot be
related to already known parton distributions, and so experimental information
about $\tilde{E}$ via DEMP can provide new information on nucleon structure
which is unlikely to be available from any other source.

\begin{figure}[hbt!]
\begin{center}
\includegraphics[height=5cm]{./figures/atpi_planes.png}
\end{center}
\caption{\label{fig:planes}
\footnotesize{
Scattering and hadronic reaction planes for exclusive $\vec{N}(e,e'\pi)N'$.
$\phi$ is the azimuthal angle between the hadronic reaction plane and the
electron scattering plane, and $\theta_q$ is the scattering angle of the
virtual photon with respect to the incident electron beam.  $\phi_S$ is the
azimuthal angle between the target nucleon polarization and the scattering
plane, and $\beta=(\phi-\phi_s)$ is the angle between the target nucleon
polarization vector and the reaction plane.}}
\end{figure}

\subsection{Single spin asymmetry in exclusive pion electroproduction}

Frankfurt et al. \cite{Fr99} have considered a specific polarization observable
which is the most sensitive observable to probe the spin-flip $\tilde{E}$.
This variable is the single-spin asymmetry for exclusive charged pion
production, $\vec{p}(e,e'\pi^+)n$ or $\vec{n}(e,e'\pi^-)p$, from a transversely
polarized nucleon target, and is defined \cite{Be01} as
\begin{equation} \label{eqn:asy}
A_L^{\perp}=(\int^{\pi}_0 d\beta \frac{d\sigma^{\pi}_L}{d\beta} -
\int^{2\pi}_{\pi} d\beta \frac{d\sigma^{\pi}_L}{d\beta})
(\int^{2\pi}_0 d\beta \frac{d\sigma^{\pi}_L}{d\beta})^{-1},
\end{equation}
where $d\sigma^{\pi}_L$ is the exclusive charged pion electroproduction cross
section using longitudinally polarized photons and $\beta$ is the angle between
the nucleon polarization vector and the reaction plane
(Fig.~\ref{fig:planes}). 

This asymmetry is related to the parton-helicity-conserving part of the
scattering process and is sensitive to the interference between $\tilde{H}$ and
$\tilde{E}$ \cite{Fr99,Di05}:
\begin{equation} \label{eqn:asy2}
A_L^{\perp}=\frac{\sqrt{-t'}}{m_p}
\frac{\xi\sqrt{1-\xi^2}\ \mathrm{Im}(\tilde{E}^*\tilde{H})}
{(1-\xi^2)\tilde{H}^2-\frac{t\xi^2}{4m_p}
\tilde{E}^2-2\xi^2\mathrm{Re}(\tilde{E}^*\tilde{H})}.
\end{equation}
Frankfurt et al. \cite{Fr99} have shown that this asymmetry must vanish if
$\tilde{E}$ is zero.  If $\tilde{E}$ is not zero, the asymmetry will display a
$\sin\beta$ dependence.  Their predicted asymmetry using the
$\tilde{E}$ ansatz from Ref. \cite{Va99} is shown in
Fig. \ref{fig:frankfurt_atpi}.  This calculation is $Q^2$-independent,
depending only on how well the soft contributions cancel in the asymmetry.

\begin{figure}[hbt!]
\begin{center}
\includegraphics[height=7cm]{./figures/frankfurt_atpi.pdf}
\end{center}
\caption{\label{fig:frankfurt_atpi}
\footnotesize{
Transverse single-spin asymmetry for the longitudinal electroproduction of
$\pi^+n$ and $\pi^+\Delta^0$ at different values of $t$ [indicated on the
curves in GeV$^2$].  The asymmetry drops to zero at the parallel kinematic
limit, which is different for each $t$ value, because the definition of $\beta$
is ill-defined at this point.  This figure is taken from
Ref.~\protect{\cite{Fr00}}.
}}
\end{figure}

It seems likely that a precocious factorization of the meson production
amplitude into three parts -- the overlap integral between the photon and pion
wave functions, the hard interaction, and the GPD -- will lead to a precocious
scaling of $A_L^{\perp}$ as a function of $Q^2$ at moderate $Q^2\sim 2-4$
GeV$^2$ \cite{Fr99}.  This precocious scaling arises from the fact that higher
twist corrections, which are expected to be significant at low $Q^2$, will
likely cancel when one examines the ratio of two longitudinal observables.  In
contrast, the onset of scaling for the absolute cross section is only expected
for much larger values of $Q^2>10$ GeV$^2$.

\begin{figure}[hbt!]
\begin{center}
\includegraphics[height=10cm, angle=90]{./figures/belitsky2.pdf}
\end{center}
\caption{\label{fig:belitsky_atpi}
\footnotesize{
Calculation of the longitudinal photon transverse nucleon spin asymmetry
including twist-four corrections by A. Belitsky \cite{belitsky} at $-t=0.3$
GeV$^2$, $Q^2$=4 GeV$^2$.  The red curves are the leading order calculation,
while the black curves have twist-four power effects taken into account.  While
the cross section is very sensitive to these corrections, the transverse spin
asymmetry is stable.}}
\end{figure}

This point is made clear in Fig. \ref{fig:belitsky_atpi}.  This figure shows
renormalon model calculations \cite{belitsky} of both the asymmetry and the
longitudinal cross section at $Q^2=4$ GeV$^2$.  While the magnitude of the
cross section changes significantly when taking into account the twist-four
corrections, $A_L^{\perp}$ is essentially insensitive to them and displays the
expected precocious scaling.  The relatively low value of $Q^2$ for the
expected onset of precocious scaling is important, because it should be
experimentally accessible at Jefferson Lab.  This places $A_L^{\perp}$ among
the most important GPD measurements that can be made in the meson scalar.  If
precocious scaling cannot be experimentally demonstrated in this ratio of two
cross sections, then it may not be possible to determine GPDs from DEMP data.

Refs. \cite{Go01} and \cite{Fr00} also point out that the study of the
transverse target single-spin asymmetry versus $t$ is important for the
reliable extraction of the pion form factor from electroproduction experiments
(Fig. 2b).  Investigations of hard exclusive $\pi^+$ electroproduction using a
pQCD factorization model \cite{Ma99,Ca90} find that at $x_B=0.3$ and
$-t=-t_{min}$, the pion pole contributes about 80\% of the longitudinal cross
section.  Since the longitudinal photon transverse single-spin asymmetry is an
interference between
pseudoscalar and pseudovector contributions, its measurement would help
constrain the non-pole pseudovector contribution, and so assist the more
reliable extraction of the pion form factor.  The upper $Q^2=6$ GeV$^2$ limit
of the approved pion form factor measurements in the JLab 12 GeV program
\cite{12GeV} is dictated primarily by the requirement $-t_{min}<0.2$ GeV$^2$,
to keep non-pion pole contributions to $\sigma_L$ at an acceptable level
\cite{Ca90}.  Transverse target single-spin asymmetry studies versus $t$ may
eventually allow, with theoretical input, the use of somewhat larger $-t$ data
for pion form factor measurements, ultimately extending the $Q^2$-reach of pion
form factor data acquired with JLab 12 GeV beam.  Thus, measurements of
the transverse single-spin asymmetry are a logical step in the support of the
pion form factor program.

\subsection{The Complementarity of Separated and Unseparated Asymmetry
  Measurements}

The reaction of interest is $^3$He$(e,e'\pi^-)p(pp)_{sp}$.
The measurement of the transverse single-spin asymmetry requires the detection
of the $\pi^-$ in non-parallel kinematics.  It is the component of the target
polarization parallel to $\hat{q}\times\hat{p_{\pi}}$ that is important, and
this direction is uniquely defined only in non-parallel kinematics.

Experimentally, the angle between the target polarization and the reaction
plane, $\beta$, and the angle between the scattering and reaction planes,
$\phi$, are not independent.  If the target polarization is at some angle,
$\phi_s$, relative to the scattering plane, then $\beta = \phi-\phi_s$.  

The polarized nucleon cross section can be expressed \cite{Di05,hermes-thesis} 
in terms of these variables as:
\begin{equation}
\begin{split}\label{eqn:sigtarg}
d\sigma_{UT}(\phi,\phi_s) =  \sum_k d\sigma_{UT_k}(\phi,\phi_s) \;
= -\frac{P_\perp \cos\theta_q}{\sqrt{1-\sin^2\theta_q \sin^2\phi_s}} \biggl[\; 
  &\; \sin\beta \; \rm{Im}(\sigma^{+-}_{++} + \epsilon\sigma^{+-}_{00})\\
 +&\; \sin\phi_s \; \sqrt{\epsilon(1+\epsilon)}\; \rm{Im}(\sigma^{+-}_{+0})\\
 +&\; \sin(\phi+\phi_s) \; \frac{\epsilon}{2} \; \rm{Im}(\sigma^{+-}_{+-})\\
 +&\; \sin(2\phi-\phi_s) \; \sqrt{\epsilon(1+\epsilon)} \; \rm{Im}(\sigma^{-+}_{+0}) \\
 +&\; \sin(3\phi-\phi_s) \; \frac{\epsilon}{2} \; \rm{Im}(\sigma^{-+}_{+-})\;\; \biggr] ,
\end{split}
\end{equation}
where the $\cos\theta_q$ factor is needed to convert the target $P_\perp$
relative to the lepton beam to that relative to the virtual photon
(in accordance with the Trento convention), and much smaller factors
proportional to $\sin\theta_q$ have been neglected for clarity.  
The $\sigma^{ij}_{mn}$ correspond to nucleon polarizations
$ij=(+\frac{1}{2},-\frac{1}{2})$ and photon polarizations $mn=(+1,0,-1)$, and
$\sigma_T=\frac{1}{2}(\sigma^{++}_{++}+\sigma^{--}_{++})$,
$\sigma_L=\sigma^{++}_{00}$ are the usual unpolarized transverse and
longitudinal cross sections.

From the above equation, it is clear that to extract $A_L^{\perp}$
it is necessary to first isolate the 
$\sin \beta$ Fourier component of the polarized nucleon cross section.
Once that has been accomplished, one must then separate the desired
$\sigma^{+-}_{00}$ term 
from the $\sigma^{+-}_{++}$ term via a Rosenbluth-type separation.  All of the
other polarized terms can be distinguished from their azimuthal dependences,
without need of a Rosenbluth separation \cite{Di05}.

It has not yet been possible to perform an experiment to measure $A_L^{\perp}$.
The conflicting experimental requirements of transversely polarized target,
high luminosity, L--T separation, and closely controlled systematic
uncertainty, make this an exceptionally challenging observable to measure.  The
SHMS+HMS is the only facility with the necessary resolution and systematic
error control to allow a measurement of $A_L^{\perp}$.  However, the beamtime
required to do a good measurement with current polarized target technology is
in the range of 10$^3$ days.  To minimize the beamtime required, PR12-12-005
\cite{atpi39}
proposed the use of a next generation, externally polarized, continuous flow,
high luminosity $^3$He target based on a large volume polarizer and compressor
developed at the University of New Hampshire.  The science case
for this measurement was favorably reviewed by PAC39, and they encouraged the
continued development of the target technology.  Although the New Hampshire
group is making continued progress on the development of the target, there is
no timeline for its actual implementation at Jefferson Lab.

The most closely related measurement, of the transverse single-spin asymmetry
in exclusive $\pi^+$ electroproduction without an L--T separation, was
published by the HERMES Collaboration in 2010 \cite{hermes10}.  Their data were
obtained for average values of $\langle x_B \rangle =0.13$, $\langle Q^2 \rangle
=2.38$ GeV$^2$ and $\langle t' \rangle = -0.46$ GeV$^2$, subject to the
criterion $W^2>10$ GeV$^2$.  The six Fourier amplitudes in terms of the
azimuthal angles $\phi$, $\phi_s$ of the pion-momentum and proton-polarization
vectors relative to the lepton scattering plane were determined.  Of these, at
leading twist only the $\sin\beta_{UT}$ Fourier amplitude receives a
contribution from longitudinal photons.  If one assumes that longitudinal
contributions dominate, these $A_{UT}^{sin(\phi-\phi_s)}$ values can be
compared to GPD models for $\tilde{E}$, $\tilde{H}$.

\begin{figure}[hbt!]
\begin{center}
\includegraphics[height=7cm]{./figures/hermes_Aut.pdf}
\end{center}
\caption{\label{fig:hermes_aut}
\footnotesize{
Predictions by Goloskokov and Kroll for the $\sin(\phi-\phi_s)$ moment of
$A_{UT}$ in the handbag approach, in comparison to the data from HERMES at
$Q^2=2.45$ GeV$^2$, $W=3.99$ GeV.  Please note that the HERMES data
follow the Trento convention, while the Eqns. \ref{eqn:asy}, \ref{eqn:asy2} and
Figs. \ref{fig:frankfurt_atpi}, \ref{fig:belitsky_atpi} do not, leading to a
normalization difference of $-\pi/2$ \cite{hermes-thesis}.  The independent
variable is $-t'=|t-t_{min}|$.  Dashed line: contribution from longitudinal
photons only.  Solid line: full calculation including both transverse and
longitudinal photons.  This figure is taken from Ref. \protect{\cite{Go10}}.}}
\end{figure}

Because transverse photon amplitudes are suppressed by $1/Q$, at very high
$Q^2$ it is safe to assume that all observed meson production is due to
longitudinal photons.  At the lower $Q^2$ typical of the JLab and HERMES
programs, however, this is not the case.  Handbag-approach
calculations by Goloskokov and Kroll
\cite{Go10} indicate much of the unseparated cross section measured by HERMES
\cite{hermes10} is due to contributions from transversely polarized photons.
In addition, there are contributions to $A_{UT}^{\sin(\phi-\phi_s)}$ from the
interference between two amplitudes, both for longitudinal photons, as well as
transverse photons~\cite{Di05}.
At the amplitude level, the transverse suppression is given by $\mu/Q$, where
$\mu\sim 2$ GeV is a mass parameter given by the pion mass enhanced by the
large ratio between the pion mass and the sum of the $u$ and $d$ current quark
masses (chiral condensate).  For experimentally accessible $Q^2$, hardly any
suppression of the twist-3 contribution is expected.  As indicated in
Fig. \ref{fig:hermes_aut}, the contribution from transverse photons tends to
make the asymmetry smaller.  At the HERMES kinematics, the dilution caused by
transverse photons is about 50\%.  Although the observed unseparated asymmetry
is small, the HERMES data are consistent with GPD models based on the dominance
of $\tilde{E}$ over $\tilde{H}$ at low $-t'$.  Indeed, the change in sign in
the model curves at large $-t'$ in Fig.~\ref{fig:hermes_aut} is due to the
large contribution from $\tilde{E}$ demanded by the data \cite{Go10}.  An
improved measurement of the transverse target spin asymmetry, in particular the
$\sin(\phi-\phi_S)$ modulation, is clearly a high priority.

\begin{figure}[hbt!]
\begin{center}
\includegraphics[height=10cm]{./figures/goloskokov2.png}
\caption{\label{fig:golo_aut}
\footnotesize{
Calculation of the cross section components and $\sin(\phi-\phi_s)$ moment of
the transverse nucleon spin asymmetry $A_{UT}$ in the handbag approach by
Goloskokov and Kroll \cite{GoPC} for kinematics similar to those in
Fig. \ref{fig:belitsky_atpi}.  Our measurement will be at higher
$0.55<\epsilon<0.75$ than the $\epsilon=0.35$ kinematics of this figure,
so the dilution in the asymmetry will be significantly less.}}
\end{center}
\end{figure}

A run-group proposal concurrent with the SoLID transversely polarized $^3$He
SIDIS experiment allows for an unseparated asymmetry measurement to be obtained
on a sooner timescale than the Hall C measurement.  In comparison to the HERMES
measurement, the experiment proposed here will probe higher $Q^2$ and $x_B$,
with much smaller statistical errors over a wider range of $-t$.
SoLID will allow the first measurement for $Q^2>4$ GeV$^2$, where GPD-based
calculations are expected to apply.  Thus, the measurements should be more
readily interpretable than those from HERMES.  Similar measurements using
CLAS-12 and a transversely polarized $^1$H target have been discussed
previously \cite{clas}, but this measurement will allow for smaller statistical
uncertainties, due to SoLID's higher luminosity capabilities.

Handbag model calculations by Goloskokov and Kroll \cite{GoPC} shed further
light on the expected asymmetry dilution.  The bottom panel of
Fig. \ref{fig:golo_aut} shows their predictions for the cross section
components in exclusive charged pion production.  Although their calculations
tend to underestimate the $\sigma_L$ values measured in the JLab $F_{\pi}-2$
experiment \cite{Fpi2}, their model is in reasonable agreement with the
unseparated cross sections \cite{Go10}.  They predict significant transverse
contributions for JLab kinematics.  A comparison of the unseparated asymmetry
at $-t=0.3$ GeV$^2$, $x_B=0.365$ in Fig. \ref{fig:golo_aut} with the separated
longitudinal asymmetry at the same values of $x_B$, $-t$ in
Fig. \ref{fig:belitsky_atpi} indicates a substantial dilution of the
unseparated asymmetry due to transverse photon contributions, similar to that
observed in Fig. \ref{fig:hermes_aut}.

In addition to allowing a measurement at $Q^2>4$ GeV$^2$, a measurement by
SoLID of $A_{UT}^{\sin(\phi-\phi_s)}$ will cover a fairly large range of $-t$,
allowing the asymmetry to be mapped over its full range with good statistical
uncertainties -- from its required zero-value in parallel kinematics, through
its maximum, and then back to near-zero or even positive at larger $-t$.  
The shape of the asymmetry curve versus
$-t$, as well as its maximum value, are critical information for comparison to
GPD-based models.

\subsection{Motivation for and Status of the other Fourier Azimuthal Components
\label{sec:sinphiS}}

An important point is that any model that describes exclusive pion production
will need to describe not only the leading-twist Fourier amplitude
$A_{UT}^{\sin(\phi-\phi_s)}$, but also the other contributions to the
target-spin azimuthal asymmetry listed in Eqn.~\ref{eqn:sigtarg}, providing
additional GPD model constraints.  Like HERMES, we plan to determine the full
set of asymmetries from their azimuthal modulations,
\begin{equation}
A(\phi,\phi_s)=\frac{d\sigma_{UT}(\phi,\phi_s)}{d\sigma_{UU}(\phi)}
=-\sum_k A_{UT}^{\sin(\mu\phi+\lambda\phi_s)_k}\sin(\mu\phi+\lambda\phi_s)_k ,
\label{eqn:six_asym}
\end{equation}
where $d\sigma_{UU}$ is the unpolarized nucleon cross section in terms of the
well-known L, T, LT and TT response functions.  These asymmetries include all
five terms listed in Eqn.~\ref{eqn:sigtarg}, plus a small
$\sin(2\phi+\phi_s)$ term proportional to $\sin\theta_q$ not listed there.

\begin{figure}[hbt!]
\begin{center}
\includegraphics[height=7cm]{./figures/hermes_sinphiS.png}
\vspace*{-5mm}
\end{center}
\caption{\label{fig:hermes_phiS}
\footnotesize{
Data from HERMES for the $\sin(\phi_s)$ moment with a transversely polarized
target at $Q^2=2.45$ GeV$^2$, $W=3.99$ GeV.  The solid line is the prediction
of the handbag calculation by Goloskokov and Kroll under the assumption that
the dominant transversity GPD is $H_T$ and that the other three can be
neglected.  The dashed line is obtained disregarding the twist-3 contribution.
This figure is taken from Ref. \protect{\cite{Go10}}.}}
\end{figure}

While most of the theoretical interest and the primary motivation of our
experiment is the target asymmetry proportional to $\sin\beta$, there is
growing interest in the $\sin(\phi_S)_{UT}$ asymmetry, as it may be
interpretable in terms of the transversity GPDs.  
Independent of a specific dynamical interpretation
(e.g. the handbag approach), the $A_{UT}^{\sin(\phi_S)}$ asymmetry will say
something on the strength of the contributions from transverse photons at
small $-t$:
\begin{equation}
A_{UT}^{\sin(\phi_S)}\sim\mathrm{Im}[M^*_{0+,++}M_{0-,0+} - M^*_{0-,++}M_{0+,0+}],
\end{equation}
where the helicities are in the order: pion neutron, photon proton
\cite{Go10}.  In contrast to the $\sin(\phi-\phi_S)$ modulation, which
has contributions from LL and TT interferences, the $\sin(\phi_s)$ modulation
measures only the LT interference.  The first term is proportional to $t'$, as
is forced by angular momentum conservation, while the second one is not forced
to vanish \cite{GoPC}.  Indeed, HERMES measured the $\sin(\phi_s)$ modulation
to be large and apparently nonzero at $-t'=0$ (Fig.~\ref{fig:hermes_phiS}).
Hence, both the amplitudes $M_{0-,++}$ and $M_{0+,0+}$ must be large, giving the
first clear signal for strong contributions from transversely polarized photons
at rather large values of $W$ and $Q^2$ \cite{Go10}.  This is very interesting
in its own right.

\begin{figure}[hbt!]
\begin{center}
\includegraphics[height=7cm]{./figures/hermes_sin2phiphiS.png}
\vspace*{-5mm}
\end{center}
\caption{\label{fig:hermes_sin2phiphiS}
\footnotesize{
Data from HERMES for the $\sin(2\phi-\phi_s)$ moment with a transversely
polarized target at $Q^2=2.45$ GeV$^2$, $W=3.99$ GeV.  The solid line is the
prediction of the handbag calculation by Goloskokov and Kroll
\protect{\cite{Go10}}.}}
\end{figure}

The calculation by Goloskokov and Kroll \cite{Go10} for the
$\sin(2\phi-\phi_S)$ modulation measured by HERMES is shown in
Fig.~\ref{fig:hermes_sin2phiphiS}.  Both the experimental values, as well as
the calculation, are small.  The agreement is fairly good, except that the
change of sign at large $-t'$ is not reproduced by the model.  However, this
observable is given by an interference between longitudinal amplitudes and
transversal other than $M_{0-,++}$, so an improvement of this moment probably
requires a detailed modeling of the small transverse amplitudes contributing.
The other three moments, $\sin(\phi+\phi_S)$ $\sin(2\phi+\phi_S)$ and
$\sin(3\phi-\phi_S)$, are only fed by transverse-transverse interference terms
and are therefore small in the handbag approach of Ref.~\cite{Go10}.

\begin{figure}[htb!]
\begin{center}
\includegraphics[height=20cm]{./figures/gk16.png}
\caption{\label{fig:gk16}
\footnotesize{
$\sin(\mu\phi+\lambda\phi_s)_k$ moments of the transverse nucleon spin
  asymmetry $A_{UT}$ calculated in the handbag approach by Goloskokov and Kroll
  \cite{GoPC} for the kinematics of this experiment.
Solid black: $Q^2=4.107$ GeV$^2$, $W=$3.166 GeV;  
Long-dash red: $Q^2=5.138$ GeV$^2$, $W=$2.796 GeV;  
Dash-dot green: $Q^2=6.049$ GeV$^2$, $W=$2.718 GeV;  
Short-dash blue: $Q^2=6.894$ GeV$^2$, $W=$2.562 GeV.}}
\end{center}
\end{figure}

\begin{figure}[hbt!]
\begin{center}
\includegraphics[height=20cm]{./figures/hermes_6asym.png}
\end{center}
\caption{\label{fig:hermes6}
\footnotesize{
HERMES data for the set of six Fourier amplitudes $A_{UT}^{\sin(\phi-\phi_S)}$
describing the sine modulations of the single-spin azimuthal asymetry.  The
error bars (bands) represent the statistical (systematic) uncertainties.  The
results receive an additional 8.2\% scale uncertainty corresponding to the
target polarization uncertainty.  This figure is taken from
Ref. \protect{\cite{hermes10}}.}}
\end{figure}

A calculation for the kinematics of this experiment by S.V. Goloskokov and
P. Kroll \cite{Go10,GoPC,Go11} for the amplitudes of the five azimuthal
modulations listed in Eqn.~\ref{eqn:sigtarg} is shown in Fig.~\ref{fig:gk16}.
It clearly shows that the two asymmetries of greatest physics interest
dominate, while the other asymmetries are much smaller.  This is consistent
with the HERMES result, which found the four asymmetries $\sin(2\phi-\phi_s)$, 
$\sin(\phi+\phi_s)$, $\sin(3\phi-\phi_s)$, $\sin(2\phi+\phi_s)$, to be small
over most of the measured $t$ range (Fig.~\ref{fig:hermes6}).  
This is good news, as large expected values for these asymmetries would
complicate the extraction of the most-valued Fourier components.


\newpage
\section{Experimental Method}

We propose to carry out the $^3$He$(e,e'\pi^-)p(pp)_{sp}$ measurement using the
Solenoidal Large Intensity Device (SoLID~\cite{solid_pcdr}), in parallel with
the already approved experiment, E12-10-006~\cite{solid:e12-10-006}, which will
measure Semi-Inclusive Deep-Inelastic Scattering (SIDIS). 
Our discussion will concentrate on the region of clearest physics
interpretation ($Q^2>$4~GeV$^2$), even through lower $Q^2$ events will also be
contained in the experimental data-set.

There are two SoLID
configurations, called SoLID-SIDIS and SoLID-PVDIS. Besides E12-10-006, two
SIDIS experiments, E12-11-007~\cite{solid:e12-11-007} and
E12-11-108~\cite{solid:e12-11-108}, along with the $J/\psi$ experiment
(E12-12-006~\cite{solid:e12-12-006}), will also use the SoLID-SIDIS
configuration. All of these experiments have been approved with A or A-
rating.  In addition, two ``bonus-run'' experiments,
E12-10-006A~\cite{solid:e12-10-006A} and E12-11-108A~\cite{solid:e12-11-008A},
have also been approved to run in parallel with the SIDIS experiments. The
SoLID-PVDIS configuration is for the Parity Violation in Deep Inelastic
Scattering (PVDIS)~\cite{solid:e12-10-007}.

In order to assure a clean measurement of exclusive $\pi^-$ production, it is
required to detect the recoil proton from the $\vec{n}(e,e'\pi^-)p$ reaction.
The existing SoLID detectors already have the capabilities of detecting protons
from 8$^{\circ}$ up to 24$^{\circ}$, while the main proton events from the DEMP
process can cover 0$^{\circ}$ up to 50$^{\circ}$.  The experiment will use
exactly the same setup and online production trigger as E12-10-006, which is
the coincidence of electron triggers and hadron triggers from SoLID. We will
perform the offline analysis to identify the recoil protons from DEMP and form
the triple coincidence events together with electrons and $\pi^{-}$ provided by
SIDIS triggers. The discussion of proton detection is given in
Sec.~\ref{sec:proton_ID}.

%The SoLID-SIDIS detector can only detect protons with scattering angles from
%8$^{\circ}$ up to 24$^{\circ}$, while the main proton events from the DEMP
%process can cover 0$^{\circ}$ up to 50$^{\circ}$. In a new Letter-Of-Intent submitted together with this proposal, we
%propose to add a new proton recoil detector (PRD) based on scintillator
%counters to detect protons with angles from 24$^{\circ}$ to 50$^{\circ}$, to
%acquire asymmetry data over a larger kinematic range. The new detector will be
%placed between the target system and the entrance of the solenoid magnet. The
%proton identification and the conceptual design of the new proton detector will
%be discussed in more detail in the following sections and in Appendix-B.  We
%are seeking PAC advice on whether we should pursue the design and prototyping
%of the PRD under the proviso that it have minimal impact upon the running of
%E12-10-006.

\subsection {Transversely Polarized $\mathrm{^{3}He}$ Target}

\begin{table}[!ht]
\centering
\begin{tabular}{|c|c|}
\hline
Target                       & $^3$He              \\\hline 
Length                       & 40 cm               \\\hline          
Target Polarization          & $\sim$60\%          \\\hline 
Target Spin Flip             & $\leq$20 mins       \\\hline 
Target Dilution              & 90\%                \\\hline
Effective Neutron            & 86.5\%              \\\hline
Target Polarimetry Accuracy  & $\sim$ 3\%          \\\hline
\end{tabular}
\caption{\footnotesize{Key Parameters of the $\mathrm{^{3}He}$
    target.}\label{table:target}}
\end{table}

The proposed measurement will utilize the same polarized $\mathrm{^{3}He}$
target as E12-10-006~\cite{solid:e12-10-006}. Such a target was successfully
employed in E06-110, a 6~GeV SIDIS experiment in Hall A.  A wide range of
experiments have utilized polarized $^3$He as an effective neutron target over
a wide range of kinematics. And over the past decades several authors have
calculated the effective neutron polarization in $^3$He using three-nucleon
wave functions and various models of the $N-N$ interaction~\cite{3hepol1}.
These are now well established, and the error introduced by uncertainty in the
wave functions are small.

Other nuclear effects which can influence the experimental asymmetry for a
neutron bound inside $^3$He include Fermi motion, off-shell effects, meson
exchange currents, delta isobar contributions and $\pi^-$ final state
interactions. The exclusive nature of the process, the selected kinematics such
as high $Q^2$, large recoil momentum and a complete coverage of the azimuthal
angle $\phi$ ensures that corrections due to these nuclear effects will be
small and can be modeled effectively.

The $\mathrm{^{3}He}$ polarization direction is held by three sets of Helmholtz
coils with a 25~Gauss magnetic field. Both the transverse and longitudinal
directions can be provided by rotating the magnetic field. The
$\mathrm{^{3}He}$ gas, with density of about 10~atm (at $0^{\circ}$C), is stored
in a 40~cm target cell made of thin glasses. With a 15~$\mu$A electron beam,
the neutron luminosity can be as high as $\mathrm{10^{36}~cm^{-2}s^{-1}}$. In-beam
polarization of 60\% was archived during the E06-110 experiment. Two kinds of
polarimetry, NMR and EPR, were used to measure the polarization with relative
5\% precision. We have plans to improve the accuracy of the measurement to
reach 3\%.

The target spin will be reversed for every 20 minutes by using the RF AFP
technique. The additional polarization loss due to the spin reversal was kept
at $<10\%$, which has been taken into account in the overall 60\% in-beam
polarization. A new method for spin reversal using field rotation has been
tested and was able to eliminate the polarization loss. Such an improvement
will enable us to perform the spin-reversal in few minutes to reduce the
target-spin-correlated systematic errors. The key parameters of the
$\mathrm{^{3}He}$ target are summarized in Table~\ref{table:target}.
  
A collimator, similar to the one used in the E06-110, will be placed next to
the target cell window to minimize the target cell contamination and to reduce
the event rate. Several calibration targets will also be installed in this
target system, including a multi-foil $^{12}$C for optics study, a BeO target
for beam tuning, and a reference target cell for dilution study and other
calibration purposes.
  
  \subsection {SoLID Spectrometer and Detectors} 
\begin{figure}[!h]
 \begin{center}
  \includegraphics[width=0.8\textwidth]{./figures/SoLID_setup_SIDIS_He3_coil.pdf}
   \caption[The Detector Layout of the SoLID-SIDIS
     configuration]{\footnotesize{The Detector Layout of the SoLID-SIDIS
       configuration. The detector system includes six Gas Electron Multiplier
       (GEM) planes for charged particle tracking, two Scintillator Pad
       Detectors (SPD) followed by two Shashlyk sampling EM Calorimeters (EC)
       for energy measurement and particle identification, a Light Gas
       \v{C}erenkov Detector (LGC) for e-$\pi^{\pm}$ separation, a Heavy Gas
       \v{C}erenkov Detector (HGC) for $\pi^{\pm}$-$K^{\pm}$ separation, as
       well as a Multi-gap Resistive Plate Chamber (MRPC) for timing
       measurement. The first four GEM trackers, the first SPD (i.e. LASPD) and
       EC (i.e. LAEC) form the large-angle detection system for electron
       measurement. The forward-angle detection system, to measure electron and
       hadrons, is composed of all six GEM trackers, LGC, HGC, MRPC, the second
       SPD (i.e. FASPD) and the second EC (FAEC).}  The transversely polarized
     $^3$He target field coils are shown to scale.}
   \label{solid_sidis}
 \end{center}
\end{figure}

The solenoid magnet for SoLID is based on the CLEO-II magnet built by
Cornell University. The magnet is 3~m long with an inner cryostat
diameter of 2.8~m. The field strength is greater than
1.35~Tesla, with an integrated BDL of 5~Tesla-meters. The fringe field at the
front end after shielding is less than 5~Gauss. In the SIDIS-configuration, the
CLEO-II magnet provides 2$\pi$ acceptance in the azimuthal angle ($\phi$) and
covers polar angle ($\theta$) from 8$^{\circ}$ up to 24$^{\circ}$. The
momentum acceptance is between 0.8 and 7.5~GeV/c for electrons and for hadrons,
the momentum can be lower depending on the trigger efficiency.  The momentum
resolution is about 2\%.

The layout of the SoLID detectors in the SIDIS-configuration is shown in
Fig.~\ref{solid_sidis}. The detector system is divided into two regions for the
forward-angle (FA) detection and the large-angle (LA) detection. Six tracking
chambers based on Gas Electron Multipliers (GEM) will be used for charged
particle tracking in the forward-angle region, and the first four of them will
be shared by the large-angle
region. In each region, a Shashlyk-type sampling EM calorimeter (LAEC or
FAEC) will measure the particle energy and identify electrons from hadrons. A
scintillator-pad detector (LASPD or FASPD) will be installed in front of each
EC to reject photons and provide timing information. The forward-angle
detectors will detect both the electrons and hadrons (mainly $\pi^{\pm}$). A
light-gas \v{C}erenkov detector (LGC) and a heavy-gas \v{C}erenkov detector
(HGC) will perform the $e/\pi^{\pm}$ and $\pi^{\pm}/K^{\pm}$ separation,
respectively. The Multi-gas Resistive Plate Chamber (MRPC) will provide a
precise timing measurement and serve as a backup of the FASPD on photon
rejection. A more detailed discussion of the design, simulation, prototype-test
of each detector is given in the SoLID preliminary conceptual design report
(pCDR)~\cite{solid_pcdr}.

Table~\ref{table:key_par_sidis_dvcs} summarizes the key parameters of the
detector system in the SIDIS configuration for both the SIDIS and DEMP
measurements.
\begin{table} \centering
\begin{tabular}{|c|c|c|c|c|}
\hline
Experiments                & SIDIS                    & DEMP  \\\hline
Reaction channel           & $\vec{n}(e,e'\pi^{\pm})X$ & $\vec{n}(e,e'\pi^{-}p)$	\\\hline
Target                     & $^3$He                   &same 	\\\hline
Unpolarized luminosity     & $\sim10^{37}$ cm$^{-2}$s$^{-1}$ per nucleon & same	\\\hline 
Momentum coverage          & 0.8-7.5 (GeV/c) for  $e^-$,$\pi^{\pm}$           &same 	\\
          										&   & 0.3~1.2 (GeV/c) for protons	\\\hline
Momentum resolution        & $\sim$2\%                & same\\\hline
Azimuthal angle coverage   & 0$^{\circ}$ ~360$^{\circ}$ & same	\\\hline
Azimuthal angle resolution & 5 mr                     & same	\\\hline
Polar angle coverage       & 8$^{\circ}$-24$^{\circ}$ for $e$ &  same \\
       & 8$^{\circ}$-14.8$^{\circ}$ for $\pi^{\pm}$  &  same 	\\
                           &                          & 8$^{\circ}$-24
                                                        $^{\circ}$ for $p$ in SoLID\\
                           &                          & 24$^{\circ}$-50$^{\circ}$ for $p$ with recoil detector         \\\hline
Polar angle resolution     & 0.6 mr                   & same	\\\hline
Target Vertex resolution   & 0.5~cm                   & same \\\hline
 Energy resolution on ECs  & 5\%$\sim$10\%            & same   \\\hline
Trigger type               & Double Coincidence $e^-+\pi^{\pm}$ & same (online)\\
              &  & Triple Coincidence $e^-+\pi^{-}+p$ (offline)\\\hline

Expected DAQ rates         &  $<$100 kHz              &  same (online)\\\hline

Main Backgrounds           & $\mathrm{^{3}He}$(e,e'K$^\pm$/$\pi^{0}$)X            &$\mathrm{^{3}He}$(e,e'$\pi^{\pm}$/K$^\pm$)X  \\
                           &   Accidental Coincidence & Accidental Coincidence	\\\hline
Key requirements           &  Radiation hardness      & Proton Detection	\\
                           &  Kaon Rejection          & Exclusivity	\\
                           &  DAQ                     &    Timing Resolution   \\
                        \hline
\end{tabular}
\caption{\footnotesize{Summary of Key Parameters for DEMP Measurement compared
    with SIDIS Experiments.}}\label{table:program_summary}
\label{table:key_par_sidis_dvcs}
\end{table} 

%\newpage
\subsection{Recoil Proton Identification
\label{sec:proton_ID}}
The cleanest way to identify the DEMP events is to detect all particles in the
final state. The SoLID-SIDIS detector system has the capability of measuring
electrons and pions, while protons can be isolated from other charged particles
by using the time-of-flight (TOF) information. The TOF is provided by the timing
detectors, including the MRPC and FASPD at the forward-angle detection region,
and the LASPD at the large-angle detection region. 

We examined the requirement of the timing resolution on these detectors by
looking at the time difference between electrons and other heavier charged
particles when they reach these detectors with the same momentum and flight
path. As shown in the next section, the good protons from the DEMP reaction
carry momenta from 0.3~GeV/c up to 1.2~GeV/c with angles from
$0^{\circ}$ to $50^{\circ}$. The FA-MRPC covers angles from
$8^{\circ}$ to $14.8^{\circ}$, and the angular range of the LASPD is from
$16^{\circ}$ to $24^{\circ}$.  Hence we simulated
events of electrons, pions, kaons and protons with the momentum from
0.3~GeV/c up to 1.2~GeV/c, and calculated the time when they reach two different
detectors with linear trajectories and at fixed angles.

The results are shown in Fig.~\ref{tof_diff}. To clearly identify two types of
charged particles with the same momentum, we normally require the timing
difference between two particles to be larger than 5 times of the overall
timing resolution, while the SoLID timing detectors can reach the resolution
in the range of 150~ps down to 50~ps.  At the FA-MRPC, which is more than 7
meters from the target, protons come 3~ns later than kaons, even at the highest
momenta in the DEMP reaction. Hence, protons will be easily distinguished from
other lighter particles.  At the LA-SPD, which is about 3 meters away from the
target, the time difference between protons and kaons is still more than 1~ns,
which doesn't demand precise timing resolution.

\begin{figure}[!ht]
 \begin{center}
  \includegraphics[width=0.8\textwidth]{./figures/time_diff.pdf}
   \caption[Time-of-time]{\footnotesize{The time differences (in $ns$) between
       electrons and other charged particles, i.e. pions (red solid line),
       kaons (blue dashed line) and proton (black dash-dotted line), and their
       distributions as functions of particles' momentum at two different
       timing detectors, including the forward-angle (FA) MRPC and the large-angle
       (LA) SPD.}}
   \label{tof_diff}
 \end{center}
\end{figure}

In general, the misidentified proton events can be mostly removed by cutting on the
reconstructed missing quantities, e.g. angles, momenta and masses. The residual
background will also be largely suppressed in the target-spin asymmetry
extraction.

\subsection{Trigger Design}

In E12-10-006, the online production trigger will be the double-coincidence of
the scattered electrons and hadrons. One will use the particle identification
detectors, such as LGC, HGC and ECs, during the offline analysis to select
$\pi^{\pm}$ out from other hadrons. The DEMP events will be identified with the
triple-coincidence of the scattered electron, $\pi^{-}$ and proton, while the
proton identification has been discussed above. We will use the same online
trigger as the SIDIS one, and hence the new experiment will share exactly the same
data-set as E12-10-006. The actual design of the SIDIS triggers will be far
more complicated, and the detailed discussion of the trigger and DAQ designs is
given in the SoLID pCDR~\cite{solid_pcdr}.


\newpage
\section{Projected Results}

To perform the simulation study and obtain the projected results, we developed
a DEMP generator, as discussed in Appendix-\ref{sec:generator}, and used it to
generate events within a kinematic phase space slightly larger than the
SoLID-SIDIS acceptance.  The Fermi motion of the neutron in $\mathrm{^{3}He}$,
radiation of the incident and scattered electrons, multiple scattering of the
final state particles, and energy loss due to the ionization have been taken in
account in this generator.  Then for every detected particle in each event, we
added the acceptance profiles obtained from the GEANT4 simulation with the
SoLID-SIDIS configuration and smeared the momenta and angles of the final state
particles by the detector resolutions based on current knowledge of the
tracking reconstruction study.  We also compared the obtained distributions
with GEMC simulations taking into account the target and individual detector
geometries.  To better simulate the real experimental conditions, we generated
two sets of data with the target polarization up and down, respectively.

\subsection{Kinematic Coverage}

\begin{figure}[!ht]
 \begin{center}
      \includegraphics[width=\textwidth]
   {./figures/demp_qsq_x_Wgt2.png}
   \caption{\footnotesize{The triple coincidence $e$+$\pi^-$+$p$ kinematic
       coverage of DEMP events at 11~GeV within the SoLID acceptance as
       simulated by GEMC. A $W>$2~GeV cut was applied.}}
  \label{fig:kin_cor}
  \end{center}
\end{figure}

The kinematic coverage in $Q^{2}$ vs. $x_{B}$ is shown in
Fig.~\ref{fig:kin_cor}, using the existing SoLID detectors to detect electrons,
pions and protons at $8^{\circ}\sim24^{\circ}$. These distributions were
weighted by the DEMP unpolarized cross sections and the SoLID acceptance
profiles for electrons, pions and protons.  A cut $W>2~GeV$ was also applied to
exclude non-DIS events.

%\begin{figure}[!ht]
% \begin{center}
%   \includegraphics[type=pdf,
%     ext=.pdf,read=.pdf,width=0.75\textwidth]{./figures/dvmp_acceptance_narrow}
%   \caption[The acceptance of the momenta and scattering angles for electrons,
%     $\pi^{-}$ and protons]{\footnotesize{The acceptance of the momenta and
%       polar angles. The top, middle and bottom plots are for electrons,
%       $\pi^{-}$ and protons, respectively. Cuts of $Q^{2}>4~\mathrm{GeV^{2}}$
%       and $W>$2~GeV were applied.}}
%  \label{p_theta}
%  \end{center}
%\end{figure}

\begin{figure}[hbt!]
\begin{center}
\includegraphics[width=0.49\textwidth]{./figures/demp_trig_acceptance_qsq1.png}
\includegraphics[width=0.49\textwidth]{./figures/demp_trig_acceptance_qsq4.png}
%\includegraphics[width=0.49\textwidth]{./figures/sidis_trig_acceptance_qsq4.png}
\end{center}
\caption{\label{fig:p_theta}
\footnotesize{The triple coincidence $e$+$\pi^-$+$p$ acceptance of the momenta
  and scattering angles for electrons, $\pi^{-}$ and protons for DEMP events
  as simulated by GEMC.
{\em Left:} $Q^{2}>1~\mathrm{GeV^{2}}$ and $W>$2~GeV cuts.
{\em Right:} $Q^{2}>4~\mathrm{GeV^{2}}$ and $W>$2~GeV cuts.
The plotted kinematic quantities are the emitted particle values in the lab
frame.  See Sec.~\ref{sec:resp_gemc} for more details.
}}
\end{figure}

Fig.~\ref{fig:p_theta} shows the momentum and angular acceptance of $\pi^{-}$,
electrons and protons which form the DEMP events and can be detected with the
SoLID detectors (as determined by GEMC simulation).  The recoil protons shown
in Fig.~\ref{fig:p_theta} have low momenta ranging from 0.3~GeV/c up to
1.5~GeV/c and are distributed in both the large- and forward-angle regions.
The right panel has applied cuts of $Q^{2}>4~\mathrm{GeV^{2}}$ and $W>$2~GeV,
corresponding to the region of greatest physics interest.  See
Sec.~\ref{sec:resp_gemc} for more details on the GEMC simulations.

\subsection{Estimated Rates
\label{sec:rates}}

\begin{table}[!ht]
\centering
\begin{tabular}{|c|c|}
 \hline
  $Q^2>$1~GeV$^2$ & $Q^2>$4~GeV$^2$\\
 \hline
\multicolumn{2}{|c|}{DEMP: $\vec{n}(e,e'\pi^{-}p)$ Triple-Coincidence (Hz)}\\
 \hline
 4.95   &  0.40 \\
 \hline
\multicolumn{2}{|c|}{SIDIS: $\vec{n}(e,e'\pi^{-})X$ Double-Coincidence (Hz)}\\
 \hline
   1424.62  & 35.77   \\
 \hline
\end{tabular}
\caption[Triple-Coincidence rates for
  neutron-DEMP]{\footnotesize{Triple-Coincidence rates for DEMP events compared
    with the SIDIS rates. A $W>$2~GeV cut was applied. The online production
    trigger will be the SIDIS double-coincidence trigger of which rates are
    also given.}}
\label{rate_table}
\end{table} 

Table~\ref{rate_table} lists the triple-coincidence rate of the DEMP
events. The rates were calculated with the simulated events weighted by the
target luminosity, the SoLID acceptances and unpolarized cross sections.  The
``raw'' rates are not corrected by the beam and target polarization, target
dilution and so on.  Our conservatively estimated rate is around 4.95~Hz at
$Q^{2}>$1~GeV$^{2}$, or 0.40~Hz at $Q^{2}>$4~GeV$^{2}$. For comparison, the
table also gives the SIDIS rate which will be the online production trigger
rate and is the main background of DEMP events.

\subsection{Asymmetry Projections
\label{sec:asym}}

\begin{figure}[!ht]
 \begin{center}
      \includegraphics[width=\textwidth]
   {./figures/demp_qsq_t_Wgt2.png}
   \caption{\footnotesize{$Q^{2}$ vs. $-t$ coverage of triple coincidence
       $e$+$\pi^-$+$p$ DEMP events as simulated by GEMC, where the black dashed
       lines specify the boundaries of 7 $-t$ bins.  A $W>$2~GeV cut was
       applied.}}
  \label{fig:Q2_t_bin}
  \end{center}
\end{figure}

The proposed experiment will run in parallel with E12-10-006, which has already
been approved to run 48 days at $E_{0}$=11~GeV.  As shown in
Fig.~\ref{fig:Q2_t_bin}, we defined 7 $-t$ bins of which the boundaries are
defined by the array:
 \begin{equation}
-t[8] = [0.05, 0.20, 0.30, 0.40, 0.50, 0.70, 1.0, 3.0]~~~~(\mathrm{in~GeV^{2}})
 \end{equation}
The number of events ($N^{\uparrow\downarrow}_{i}$) in the $i$th bin is
calculated from the total simulated events after applying cuts on important
kinematic variables, e.g. $Q^{2}>$4~GeV$^{2}$, $W>$2~GeV, 0.55$<\epsilon<$0.75
and $-t_{min}<-t<-t_{max}$. Two simulated data sets with target polarization up
and down follow exactly the same cuts and binning.  As shown in
Eqn.~\ref{eqn:ncount}, each event surviving the cuts is then weighted by the
polarized cross section, together with the acceptance of the electron, pion and
proton. $N^{\uparrow\downarrow}_{i}$ is further corrected by the phase-space
factor ($PSF$) defined in the event generator, the total number of randomly
generated events ($N_{gen}$), beam-time ($T$), the target luminosity
($L=10^{36}$~cm$^{-2}$s$^{-1}$), and the overall detector efficiency
($\epsilon_{eff}$):
 \begin{equation}
     N^{\uparrow\downarrow}_{i} = \bigl(\sum_{j\in i-bin}
     \sigma^{\uparrow\downarrow}_{j}\cdot A^{e}_{j} \cdot
     A^{\pi^{-}}_{j} \cdot A^{p}_{j}\bigr) \cdot (PSF/N_{gen}) \cdot T \cdot L
     \cdot \epsilon_{eff},
     \label{eqn:ncount}
 \end{equation}
where $j$ is the $j$th event in the $i$th bin,
$\sigma^{\uparrow\downarrow}_{j}$ is the cross section of the $j$th event with
the target polarization up or down. $A^{e(\pi^{-},p)}_{j}$ is the acceptance
weight of the electron (pion, proton) in this event. The detector efficiency,
$\epsilon_{eff}$, is approximately fixed at 85\% as was used in the SIDIS
proposals. $N^{\uparrow\downarrow}_{i}$ corresponds to the raw experimental
count of electrons scattering on neutrons in $\mathrm{^{3}He}$ after taking
into account the target polarization ($P\sim60\%$), the effective polarization
of neutrons ($\eta_{n}\sim0.865$), and the dilution effect from other reaction
channels when electrons scattering on $\mathrm{^{3}He}$ ($d \sim 0.9$).

%In addition, we further divide each $-t$-bin into two $Q^{2}$ bins with similar
%statistics, as indicated in Fig.~\ref{Q2_t_bin}.  By doing that, we are able to
%examine the $Q^{2}$-dependence of the asymmetries, and also check the model
%dependence of the other corrections that are directly related to the values of
%$Q^{2}$.

With the numbers of simulated events in each bin for two anti-parallel target
polarizations, one is able to reconstruct the average target single-spin
asymmetry in that bin, which is identical to the experimental extracted
asymmetry:
\begin{equation}
   <A_{UT}> = \frac{1}{P\cdot\eta_{n}\cdot d}
   \frac{N^{\uparrow}-N^{\downarrow}}{N^{\uparrow}+N^{\downarrow}}.
   \label{eqn:asym_exp}
\end{equation}
The statistical error of the target single spin asymmetry ($A_{UT}$) in each
bin can be given as:
  \begin{equation}
    \delta A_{UT} = \frac{1}{P\cdot\eta_{n}\cdot d} \sqrt{\frac{1-(P\cdot
        <A_{UT}>)^{2}}{N^{\uparrow}_{i}+N^{\downarrow}_{i}}},
    \label{eqn:stat_err}
 \end{equation}
 
%Because of not performing a L/T separation in this experiment, the asymmetry
%should be corrected by another dilution factor, which is defined as:
%\begin{equation}
%  f_{L/T} =\frac{\epsilon\sigma_{L} }{\sigma_{T}+\epsilon\cdot\sigma_{L} },
%\end{equation} 
%where $\epsilon=(1+\frac{2\nu^{2}}{Q^{2}}\tan^{2}(\theta))^{-1}$. Additional
%dilution due to $\sigma_{TT}$ is assumed to be small.  A factor of $-1$ is also
%applied after comparing Eq.~\ref{eqn:asy} and Eq.~\ref{eqn:sigtarg}. Hence,
%$A_{UT} = -f_{L/T}\cdot A_{L}^{\perp,model}$.

As shown in Eqn.~\ref{eqn:six_asym} of Sec.~\ref{sec:sinphiS}, $A_{UT}$ can be
further decomposed into six asymmetries with different azimuthal angular
modulations:
   \begin{eqnarray}
        A_{UT}(\phi, \phi_{S}) &=& A_{UT}^{\sin(\phi-\phi_{S})}
        \sin(\phi-\phi_{S})+ A_{UT}^{\sin(\phi_{S})} \sin(\phi_{S}) \nonumber \\
       &+& A_{UT}^{\sin(2\phi-\phi_{S})} \sin(2\phi-\phi_{S})+
        A_{UT}^{\sin(3\phi-\phi_{S})} \sin(3\phi-\phi_{S}) \nonumber \\
      &+& A_{UT}^{\sin(\phi+\phi_{S})}
        \sin(\phi+\phi_{S})+A_{UT}^{\sin(2\phi+\phi_{S})} \sin(2\phi+\phi_{S}).
   \label{eqn:six_asym2}
   \end{eqnarray}
In our generator, the first five different azimuthal modulations of $A_{UT}$
are predicted with a phenomenological model as discussed in
Appendix-\ref{sec:generator}, and the last modulation is fixed to be zero as
the model predicts a negligible asymmetry.  As discussed in
Sec.~\ref{sec:motivation}, two modulations, $A_{UT}^{\sin(\phi-\phi_{S})} $ and
$A_{UT}^{\sin(\phi_{S})}$, are particularly interesting and are the main
quantities this proposal aims to measure.

To demonstrate that the proposed measurement has the capability of extracting
these asymmetries, we adopted the procedure presented in
Ref.~\cite{hermes-thesis} to extract all five asymmetries by using a unbinned
maximum likelihood (UML) method. Compared with the regular extraction methods
where the data in each $-t$ bin are further binned into two dimensional
($\phi$, $\phi_S$) bins, the UML method can perform much better fitting when
the statistics is limited.

\begin{figure}[hbt!]
\begin{center}
\includegraphics[width=0.95\textwidth]{./figures/asym_comp.pdf}
\end{center}
\caption{\label{fig:asym_comp}
\footnotesize{As an illustration of the validity of the asymmetry fitting
  method, the $A_{UT}$ asymmetries of Eqn.~\ref{eqn:asym_exp} binned as a
  2-dimensional scatter plot for $12\times 12$ $(\phi,\phi_S)$ bins, where
  dark (bright) color indicates negative (positive) single-spin asymmetry for
  that bin.
   {\it Left:} Average input model values per bin (Eqn.~\ref{eqn:six_asym2}), 
   {\it Right:} Average fit values from the UML analysis.
Note, the UML analysis uses the full power of the data, without binning, as
explained in the text.}}
\end{figure}

The polarized cross sections with two target polarization directions are given
approximately:
\begin{eqnarray}
     \sigma_{\uparrow} &=& \sigma_{UT}(\phi, \phi_{S}) = [1 +
       \frac{|P_{T}|}{\sqrt{1-\sin^2(\theta_{q})\sin^2(\phi_{S})}} A_{UT}(\phi,
       \phi_{S})] \cdot \sigma_{UU}(\phi) \\
     \sigma_{\downarrow} &=& \sigma_{UT}(\phi, \phi_{S}+\pi) = [1
       -\frac{|P_{T}|}{\sqrt{1-\sin^2(\theta_{q})\sin^2(\phi_{S})}} A_{UT}(\phi,
       \phi_{S})]\cdot \sigma_{UU}(\phi).
\end{eqnarray}
where $|P_{T}| = P\cdot \eta_{n} \cdot d $. Hence, the probability density
function can be constructed as:
\begin{equation}
  f_{\uparrow\downarrow}(\phi, \phi_{S}; \theta_{k}) =
  \frac{1}{C_{\uparrow\downarrow}} [1 \pm
    \frac{|P_{T}|}{\sqrt{1-\sin^2(\theta_{q})\sin^2(\phi_{S})}} \sum_{k=1}^{5}
    \theta_{k} \sin(\mu\phi+\lambda\phi_{S})],
\end{equation}
where $\theta_{k}$, $k=1-5$, are the values of asymmetries that can maximize
the likelihood function.  $C_{\uparrow\downarrow}$ is a normalization constant
and is set to one as it is not important in the UML fitting. Here we have
dropped out the sixth asymmetry which is zero. The UML function can be defined
as:
\begin{equation}
	L(\theta_{k}) = L_{\uparrow}(\theta_{k})\cdot
        L_{\downarrow}(\theta_{k})=\prod_{l=1}^{N_{MC}^{\uparrow}}[f_{\uparrow}(\phi_{l},
          \phi_{S,l};\theta_{k})]^{w^{\uparrow}_{l}}\cdot
        \prod_{m=1}^{N_{MC}^{\downarrow}}[f_{\downarrow}(\phi_{m},
          \phi_{S,m};\theta_{k})]^{w^{\downarrow}_{m}}.
\end{equation}
where $w^{\uparrow}_{l} = \sigma^{\uparrow}_{l}\cdot A^{e}_{l} \cdot
A^{\pi^{-}}_{l} \cdot A^{p}_{l} \cdot PSF/N_{gen} \cdot T \cdot L \cdot
\epsilon_{eff}$, is the weight of the $l^{th}$ simulated event. It takes into
account the fact that the Monte-Carlo events are generated uniformly, and it
also includes the experimental conditions such as the acceptances of three
particles and the detector efficencies. For the real experimental data, the
weight will only takes into account the acceptance correction, detector
efficiencies correction and other experimental related corrections. From
Eq.~\ref{eqn:ncount}, one has $ N^{\uparrow}_{i} = \bigl(\sum_{l\in
  i-bin}w^{\uparrow}_{l})$. Note that $N_{MC}^{\uparrow}= \bigl(\sum_{l\in
  i-bin})$ is simply the total number of simulated events in the $i^{th}$ $-t$
bin without any weighting. In practice, we use the TMinuite package to minimize
the following negative log-likelihood function:
\begin{equation}
  -lnL(\theta_{k}) =-ln
  L_{\uparrow}(\theta_{k})-lnL_{\downarrow}(\theta_{k})=-\sum_{l=1}^{N_{MC}^{\uparrow}}
  w^{\uparrow}_{l}\cdot lnf_{\uparrow}(\phi_{l},
  \phi_{S,l};\theta_{k})-\sum_{m=1}^{N_{MC}^{\downarrow}}
  w^{\downarrow}_{m}\cdot lnf_{\downarrow}(\phi_{m}, \phi_{S,m};\theta_{k}).
\end{equation}

\begin{figure}[hbt!]
\begin{center}
\includegraphics[height=20cm]{./figures/t_mult_fit5.png}
\end{center}
\caption{\label{fig:asym1_t}
\footnotesize{Projected uncertainties of the five single-spin asymmetry
  azimuthal modulations $A_{UT}^{\sin(\mu\phi+\lambda\phi_S)}$, as determined
  in the UML analysis, as a function of $-t$.  The black squares are the
  extracted values for each bin, while the blue curves are the input values, as
  obtained from the model of Fig.~\ref{fig:gk16} but averaged over the $Q^2$
  acceptance of each bin.  The simulated data include all radiative, Fermi,
  multiple scattering and energy loss effects.  The difference between the
  input and output values is largely due to Fermi momentum, as discussed
  further in Sec.~\ref{sec:resp_fermi}.}}
\end{figure}

Fig.~\ref{fig:asym_comp} compares the distribution of $A_{UT}$
vs. $(\phi,\phi_S)$ from the UML fitting (right) and from the direct
statistical averaged model calculations (left).
%The $\sin(\phi_S)$ asymmetry is the only one 
%not required to vanish at $-t_{min}$, giving a simple left-right modulation in
%the top panels (lowest $-t$ bin).  In contrast, the fourth $t$-bin displays a
%additional modulation structure reflecting primarily the contributions of
%the $\sin(\phi-\phi_S)$ and $\sin(\phi+\phi_S)$ Fourier components.  
The variations in intensity from left to right and top to bottom of each panel
indicate the presence of the different $\sin(\mu\phi+\lambda\phi_S)$ azimuthal
modulations in the single-spin asymmetry.  The
similarity between the left and right panels for both $t$-bins is readily
apparent, confirming the validity of the fitting procedure.  The equivalent
HERMES figure, with lower statistics and coarser binning, can be found in
Ref.~\cite{hermes-thesis}.

Fig.~\ref{fig:asym1_t} presents our expected precision for the five
$A_{UT}(\phi,\phi_S)$ Fourier components extracted from the UML analysis
vs. $-t$.  Compared with the existing HERMES results (Fig.~\ref{fig:hermes6}),
the new measurement can provide more precision data to be directly compared
with theoretical predictions.  The detailed information for the two asymmetries
of main interest are listed in Table~\ref{asym_bin_table}.
\begin{table}[!ht]
\centering
 \small
\begin{tabular}{|c|ccc|cc|cc|}
\hline
$t$-bin   &  $<-t>$ & $<Q^{2}>$ & $<x_{B}>$ & 
$A_{UT}^{\sin(\phi-\phi_{S})}$ & $\delta A_{UT}^{\sin(\phi-\phi_{S})}$ & 
$A_{UT}^{\sin(\phi_{S})}$ & $\delta A_{UT}^{\sin(\phi_{S})}$ \\
  \hline
1 & 0.1625 & 4.378 & 0.3368 & -0.0390 & 0.0060 & 0.3817 & 0.0059 \\
2 & 0.2504 & 4.843 & 0.3886 & -0.0214 & 0.0038 & 0.3399 & 0.0038 \\
3 & 0.3478 & 5.301 & 0.4332 & -0.0289 & 0.0031 & 0.3403 & 0.0031 \\
4 & 0.4463 & 5.770 & 0.4711 & -0.0347 & 0.0028 & 0.3484 & 0.0028 \\
5 & 0.5854 & 6.324 & 0.5140 & -0.0419 & 0.0026 & 0.3543 & 0.0025 \\
6 & 0.8091 & 6.856 & 0.5595 & -0.0449 & 0.0025 & 0.3570 & 0.0024 \\
7 & 1.0783 & 7.002 & 0.5844 & -0.0466 & 0.0025 & 0.3585 & 0.0024 \\
\hline
\end{tabular}
\caption[Detailed information of projected bins]{\footnotesize{Detailed
    information of projected bins from the new DEMP measurements with SoLID.
    $<Q^{2}>$ and $<-t>$ are in units of GeV$^{2}$. The values are as obtained
    in the UML analysis, projected uncertainties are statistical only.}}
\label{asym_bin_table}
\end{table} 

\subsection{Missing Mass and Background}

In the DEMP reaction on a neutron, all three charged particles in the final
state, $e^{-}$, $\pi^{-}$ and $p$, can be cleanly measured by the SoLID
detector system.  Hence, contamination from other reactions, including DEMP
with other two protons in $^{3}$He, can be greatly eliminated.  The dominant
background of the DEMP measurement comes from the SIDIS reactions of electrons
scattering on the neutron and two protons in $\mathrm{^{3}He}$.  In a SIDIS
event, besides the scattered electron and the hadron ($\pi^{\pm}$, $K^{\pm}$
etc.), there could be at least a proton in the target fragments, and in that
case, the SIDIS event will be possibly misidentified as a DEMP event.  In
addition to identifying the recoil protons, which should largely suppress most
of background, we will also rely on reconstructing the missing masses and
missing momentum spectra of recoil protons to ensure the exclusivity of the
DEMP events.

To calculate the missing mass and missing momentum of the recoil proton, we
reconstructed the four-momentum Lorentz vectors of the incoming electron, the
scattered electron and the pion.  We assumed the neutron target is at rest even
though its Fermi motion was simulated in the generator.  We then used the
momentum and energy conservation laws to calculate the missing mass and missing
momentum of the recoil proton.  Note that the energy loss, multiple scattering
effects, and the detector resolutions have been considered in our study.  Based
on the current tracking study, the SoLID-SIDIS system can provide a momentum
resolution of $2\%/\sqrt{E}$, a polar angle resolution of 0.6~mrad, an
azimuthal angle resolution of 5~mrad and a vertex target position of 0.5~cm.
In this measurement, we proposed to only identify the recoil proton. However,
with improved time resolution and certain tracking information, the momenta of
the recoil protons will still be determined with certain accuracies which gives
us a room to further suppress any background.

The SIDIS events, $p(e,e'\pi^{-})X$ and $n(e,e'\pi^{-})X$, were simulated with
the same generator used for the SoLID-SIDIS proposals.  The same acceptance
profiles of scattered electrons and pions in DEMP were applied to the SIDIS
events, along with the same kinematic cuts, such as $Q^2>4~GeV^2$.  It is
difficult to estimate what percentage of the target fragments in SIDIS contain
at least a proton, so we have to assumed all target fragments ($``X''$) contain
proton. Such an assumption likely results in the SIDIS background being
significantly overestimated.

We followed exactly the same methods as ones in DEMP to calculate the missing
masses and missing momenta of the recoil protons in SIDIS.
Fig.~\ref{fig:missing_mom} shows the reconstructed missing momenta of both
processes.  One immediately sees that the main peak of the DEMP missing
momentum spectrum is well separated from the SIDIS background, which can be
largely rejected with a loose cut of $P_{miss}<1.2$~GeV/c.
\begin{figure}[!ht]
\begin{center}
\includegraphics[width=0.5\textwidth]{./figures/missing_mom_cut}
\caption[Missing Momentum]{\footnotesize{Missing momentum spectra of recoil
    protons in DEMP (blue) and SIDIS (red) processes with a polarized $^{3}$He
    target. The dashed magenta curve is the DEMP missing mass only considering
    the Fermi motion, multiple scattering and the energy loss, while the blue
    and red curves have further taken into account the detector
    resolutions. The light-blue dashed line indicates a cut at
    $P_{miss}<1.2$~GeV/c to remove most of the SIDIS events.  Note that the
    SIDIS background is overestimated since we assume all SIDIS events contain
    protons in the final state. }}
  \label{fig:missing_mom}
  \end{center}
\end{figure}

Fig.~\ref{fig:missing_mass} shows the reconstructed missing mass spectra of the
DEMP and SIDIS events w/ and w/o the missing momentum cuts. Before applying the
missing momentum cut, the tail of the SIDIS background significantly leaks into
the DEMP peak. Of course, keep in mind that the SIDIS rate is likely
overestimated.  After applying the missing momentum cut, the SIDIS background
is largely suppressed.  The total integrated SIDIS becomes 0.04~Hz compared
with the rate of DEMP rate (0.5~Hz).  Further considering the fact that only a
fraction $``X''$ in SIDIS contains a proton, we conclude that the SIDIS
background is negligible after the cut.
\begin{figure}[!ht]
 \begin{center}
 \subfloat[Before cutting on missing momentum] {
      \includegraphics[width=0.45\textwidth]{./figures/Missing_Mass.pdf}
 }
  \subfloat[Before cutting on missing momentum] {
     \includegraphics[width=0.45\textwidth]{./figures/Missing_Mass_cut.pdf}
 }
   \caption[Missing Mass]{\footnotesize{Missing mass spectra of recoil protons
       in DEMP (blue) and SIDIS (red) processes with a polarized $^{3}$He
       target. The left (right) plot shows the background contamination from
       SIDIS events before (after) the missing momentum cut (shown in
       Fig.~\ref{fig:missing_mom}).  The dashed magenta curve is the DEMP
       missing mass only considering the Fermi motion, multiple scattering and
       the energy loss, while the blue and red curves have further taken into
       account the detector resolutions. Note that the SIDIS background is
       overestimated since we assume all SIDIS events contain protons in the
       final state.}}
  \label{fig:missing_mass}
  \end{center}
\end{figure}

The background sources, such as the random coincident background events, will
show up in the missing mass spectrum with more uniform distributions. We should
be able to suppress most of them with tight missing momentum and missing mass
cuts.  The residual background events will be largely suppressed or corrected
during the real asymmetry extraction. In general, we expect to have a clean
measurement of the DEMP process because all of the final particles being
detected.

\subsection{Systematic Uncertainties}

\begin{table}[!htp]
\centering
\begin{tabular}{|c|c|}
\hline
{\bf Sources}            & {\bf Relative Value} \\\hline
Beam Polarization        & $2\%$ \\\hline 
Target Polarization      & $3\%$ \\\hline 
Dilution Factor          & $1\%$ \\\hline 
Nuclear Effect           & $<4\%$ \\\hline 
Acceptance               & $3\%$ \\\hline
Radiation Correction     & $2\%$ \\\hline
Background Contamination & $<5\%$ \\\hline
\end{tabular}
\caption{\footnotesize{Expected systematic errors.}}\label{table:det_sys_err}
\end{table}

The systematic errors are expected to be close to the ones given in the
E12-10-006 proposal as well as in other SIDIS experiments with SoLID. The
procedure of extracting DEMP asymmetries is also expected to be similar to the
SIDIS asymmetry extraction.  The contamination of background should be well
controlled by the proton detection and cuts on missing momenta and mass.
However, to be conservative, we quote the overall systematic errors of
background contamination to be $5\%$ level.  Here we list several major sources
of systematic uncertainties as shown in Table~\ref{table:det_sys_err}.


\newpage
\section{Responses to Items Identified in the 2016 Review}

Our 2016 SoLID Run-Group proposal was deemed to be of high scientific merit,
but there were a number of technical questions that were asked to be studied
before final approval can be given.  The following list is compiled from the
TAC, Theory, and SoLID review reports, reordered according to topic.  Our 
response to each item is also given.

\subsection{SoLID Acceptance Simulations
\label{sec:resp_gemc}}

{\it [TAC]\ The simulations for this measurement may benefit from tracking
  DEMP events through the full SoLID GEANT4 simulation (GEMC), particularly for
  kinematics with the lowest momentum protons (300~MeV/c).}\\[0.2ex]

GEMC is a GEANT4-based simulation framework designed mainly for CLAS12. SoLID has adopted this framework to do some preliminary studies but implementation of the realistic SoLID detector information is still underway.  
To address this comment, we have
used our generator in two different types of simulations.  The projected
results in Secs.~\ref{sec:rates}, \ref{sec:asym} incorporate electron
radiation, multiple scattering and ionization energy loss in the generator.
Then for every detected particle in each event, we smeared the momenta and
angles of the final state particle by the detector resolutions based on tracking study, 
and then used the acceptance profiles
obtained with the SoLID-SIDIS GEANT4 simulation.

In the second study, the event generator generates the LUND-format file for
the three outgoing particles, $e^-$, $\pi^-$, $p$. 
We can store up to eight variables ($Q^2$, $-t$, $W$, etc.) in the header 
lines of the LUND file.  Events are generated with 
bremsstrahlung and multiple scattering for the incoming electron and
Fermi motion of the target neutron. 
Events are generated uniformly over the 40~cm length of
the SIDIS $^3$He target and a $25 mm \times 25 mm$ raster size is
assumed.  Particle acceptance is studied with the SoLID SIDIS 
GEANT4 configuration, incorporating multiple scattering and energy loss, as
well as the individual detector apertures via the GEANT4 framework.
The GEMC flux tree is used to see if a particle hits a specific detector. 

In this second study, a DEMP event is defined as the following `trigger':
\begin{center}
DEMP Event: =  $T^{e'}_{LA+FA}$ and $T^{\pi^{-}}_{FA}$  and $T^{p}_{LA+FA}$,
\end{center}
where
\begin{itemize}
\item  $T^{e'}_{LA+FA}$ is true if an electron hits the virtual planes of (LGC and
HGC and EC-FA) or EC-LA,
\item $T^{\pi^{-}}_{FA}$ is true if a pion hits the virtual planes of HGC and EC-FA and
MRPC,
\item $T^{p}_{LA+FA}$ is true if a proton hits the virtual planes of (MRPC and SPD-FA)
or SPD-LA.
\end{itemize}

\begin{figure}[!ht]
 \begin{center}
      \includegraphics[width=0.65\textwidth]{./figures/demp_Wp-x.png}
      \includegraphics[width=0.65\textwidth]{./figures/demp_W-x.png}
   \caption{\footnotesize{{\em Top:} $W'$ vs. $x$ coverage, {\em Bottom:} $W$
       vs. $x$ coverage of triple coincidence $e$+$\pi^-$+$p$ DEMP events as
       simulated by GEMC.  A $W>$2~GeV cut was applied.}}
  \label{fig:Wp_x}
  \end{center}
\end{figure}

Fig.~\ref{fig:p_theta} shows the acceptances for all three particles when an
event passes these DEMP event triple-coincidence conditions.  As expected, the
results resemble those obtained in the first study, with the low
momentum proton acceptance now being more accurately modeled. 
Because the DEMP process has very low physics rate, and meanwhile, it takes huge amount of computer time  to accumulate enough statistics, some regions may have stronger statistical fluctuation effect. However, the experimental requirement of detecting protons is more relax compared with the detection of electrons and pions, simply because we only reply on the Time-of-Flight information to separate protons from other positive particles. We will perform more carefully study of reconstructing low momentum protons but we believe the SoLID detector systems will have good coverage of detecting protons.

It is important to note that the angles plotted in Fig.~\ref{fig:p_theta} are not in the
detector frame, but correspond to the reaction vertex location within
the simulated target volume for that event.  This causes the gap between the FA
and LA angular acceptances to be largely blurred out, as well as a slight
enlargement of the accepted angular range.  Similarly, the plotted momenta
are the emitted particle values, not the detected values, corresponding to what
would be obtained in the data analysis after correcting for energy loss.  In 
comparison to the first study, this difference in plotting variables causes the 
lower range of the detected proton momentum to appear to have shifted upward
slightly, to 300-350~MeV/c.  However, the actual difference is very small and
the overall triple-coincidence rate and accepted $t$-range are only marginally
affected.

Figs.~\ref{fig:kin_cor}, \ref{fig:Q2_t_bin} also are the result of GEMC
simulations.  Finally, because there were some oral questions about the $W'$
acceptance (invariant mass of the hadronic system without the detected
$\pi^-$), we show both the $W$ and $W'$ coverage versus $x$ in
Fig.~\ref{fig:Wp_x}.  As expected for an exclusive reaction, this quantity is
sharply peaked at the mass of the recoil proton, with a small tail due to
nuclear effects.  In contrast, the SIDIS events are distributed over a broad
range $1.5<W'<3.5$ GeV \cite{solid:e12-10-006}.

We would like to stress that the SoLID GEMC simulation tool is still under massive development and the detector designs are still needed to be finalized. Together with other approved physics programs on SoLID, we plan to perform more careful study when the SoLID detector designs are completed and the full simulation software is available.

\subsection{Experimental Background
\label{sec:resp_bkd}}

{\it [SoLID]\ The committee is convinced that the SIDIS background is likely
  not a major problem.  However, an alternate approach (rather than SIDIS
  fragmentation functions) could be used.  The primary background channel under
  study is $^3$He$(e,e'\pi^-)pp$ with the two undetected protons as spectators.
  The continuum background that can leak under the quasi-exclusive peak can be
  of the form $e+n\rightarrow e'+\pi^-+\Delta^+$ with the $\Delta^+$ decaying
  to $p+\pi^0$.}\\[0.2ex]

{\it [Theory]\ The authors may want, however, to expand on possible
  contamination arising from $\Delta^{++}$ production on bound protons, and
  subsequent decay into $\pi^+$ and $p$.}\\
We take this comment to mean the $e+p\rightarrow e'+\pi^-+\Delta^{++}$
background reaction, as otherwise there is no $\pi^-$ in the final state to
satisfy the offline event finder.\\[0.2ex]

Neither $\pi^-\Delta$ final state is expected to be a substantial source of
contamination.  We have investigated and compared the kinematics of the
$\pi^-n$ and $\pi^-\Delta$ final states.  The missing momentum of the
$\pi^-\Delta$ state is about 500~MeV/c higher than the $\pi^-n$ final state,
very similar to the difference between the DEMP and SIDIS distributions in
Fig.~\ref{fig:missing_mom}.  A cut at 1.2~GeV/c missing momentum cut will 
discriminate against
%eliminate about 50\% of the
$\pi^-\Delta$ events.  In addition, the $\pi^-\Delta$ final state is centered
about 300 MeV higher in missing mass than the $\pi^-n$ state, and will be
further suppressed by a cut of approximately $M_{miss}<$1.05~GeV.  We estimate
the $\pi^-\Delta$ contamination remaining after the application of both cuts to
be very similar to that already shown in Fig.~\ref{fig:missing_mass}.

\subsection{Resolution and Energy Loss}

{\it [SoLID]\ The effects of Fermi-smearing, detector resolution, ionization
  energy loss and bremsstrahlung need to be clarified.  Although they seem to
  all be included in Figs.~15 and 16, it was not clear which curves included
  which effects.}\\[0.2ex]

We have revised the text to make this more clear.  All of the results shown
incorporate all these effects.

\subsection{Projected Uncertainties}

{\it [SoLID]\ The extraction of the term $|\sigma^y_{TT}+2\epsilon\sigma^y_L|$
  in Eqn.~8 from the other $\sin\beta$ and $\cos\beta$ terms requires good
  knowledge of the $\beta$-acceptance in each $t$-bin.  This should be shown,
  in addition to the acceptance plots of Fig.~12.}\\[0.2ex]

We have put much more effort into understanding the $(\phi,\phi_S)$ coverage of
each $t$-bin, and we hope the proposal more clearly addresses both
the goals of the experiment and the potential limitations.  As presented in
Sec.~\ref{sec:asym}, we have used the same unbinned maximum likelihood (UML)
analysis that was used by the HERMES Collaboration in the analysis of their
experimental data.  It is clear that the large azimuthal acceptance and high
luminosity capabilities of the SoLID detector makes it very well suited to this
measurement, and there should be no major obstacles to cleanly identify the
desired $\sin(\phi-\phi_S)$ and $\sin(\phi_S)$ asymmetries.

The two limitations we face are the following:
\begin{itemize}
\item{the effect of $^3$He Fermi momentum.  This is discussed more fully in
  Sec.~\ref{sec:resp_fermi}.}
\item{we are unlikely to extract definitive values for the more rapidly varying
  $\sin(2\phi-\phi_{S})$, $\sin(2\phi+\phi_{S})$, $\sin(3\phi-\phi_{S})$
  asymmetries after all statistical and systematic uncertainties are included.
  However, we have shown that this should not adversely affect the main physics
  goals of our measurement.}
\end{itemize}

We are confident that we can obtain a significant physics result should this
proposal be accepted.  and can help justify funding for the SoLID project as
one of its flagship measurements.
\\[0.2ex]

{\it [SoLID]\ The collaboration should attempt to quantify the projected
  precision of the measured spin-dependent cross section.  Although the
  asymmetry may have a smaller error bar, the spin-dependent cross section
  difference has a simpler interpretation.}\\[0.2ex]

One of the biggest advantages of DEMP measurements with transversely polarized
target is that the asymmetries can suffer many fewer higher twist effects than
the cross sections.  As shown in Fig.~\ref{fig:belitsky_atpi}, higher twist
corrections, which are expected to be significant in Jefferson Lab kinematics,
likely cancel in the formation of the single spin asymmetry, leading to a
precocious scaling of $A_L^{\perp}$ at moderate $Q^2$ \cite{Fr99, belitsky}.
This is one of the reasons for the intense theoretical interest in these
measurements.

Thus, while the extraction of the absolute spin-dependent cross sections will
undoubtedly be helpful for confirming the theoretical interpretation of our
results, they will be dominated by the higher twist effects.  It is difficult
to quantify the expected precision at this early stage, given that SoLID is not
designed for absolute cross section measurements.  To be conservative, this
proposal only relies on the extraction of the azimuthal asymmetry components
from the target up-down polarization difference, where many systematic
uncertainties cancel.  This can be done reliably, even if the systematic
uncertainties are otherwise too large for reliable absolute cross section
measurements.  The quantitative estimate of detector efficiency and acceptance
correction uncertainties, absolute kinematic offsets, etc. requires
dramatically more study and a possible optimization of the SoLID detector
system.

Ultimately, the measurement presented in this proposal is important preparatory
work for future measurements at the EIC, where the higher twist contributions
should be smaller.  The Electron-Ion Collider is optimized for transverse
single spin asymmetry measurements such as these, and the ability to have both
polarized $^3$He and proton beams will allow $A_{UT}^{sin(\phi-\phi_s)}$ to be
directly compared for the $\vec{n}(e,e'\pi^-)p$ and $\vec{p}(e,e'\pi^+)n$
reactions, without target dilution, over a broad kinematic range.  In the
meantime, the proposed measurement with SoLID is our best short-term
opportunity to considerably advance over the pioneering HERMES data.

\begin{figure}[hbt!]
\begin{center}
(a) \includegraphics[height=6cm]{./figures/t_simple_fit5.png}\\
(b)\includegraphics[height=6cm]{./figures/t_fermi_fit5.png}\\
(c)\includegraphics[height=6cm]{./figures/t_mult_nofermi_fit5.png}\\
\end{center}
\caption{\label{fig:asym_nofermi}{Projected values and uncertainties of the two
    dominant single-spin asymmetry modulations, $\sin(\phi-\phi_S)$ (left) and
    $\sin(\phi_S)$ (right) for the a) {\it simple}, b) {\it fermi} and c) {\it
      mult\_nofermi} cases, as discussed in the text.}}
\end{figure}

\subsection{Fermi Momentum Effects
\label{sec:resp_fermi}}

{\it [SoLID]\ Fermi-momentum is not just a kinematic effect.  It also affects
  the DEMP amplitude.  The $^3$He momentum distribution $\rho(p)$ is plotted in
  Fig.~10 (Appendix~A).  The weighted distribution $p^2\rho(p)$ peaks at
  $p_n\approx$60~MeV/c.  This means that the effective $x_B$ is smeared by
  $\approx p_n/M\approx$6\%.  The significance of this effect should be
  discussed.  Also, if the proton momentum resolution is good enough, it will
  be possible to correct for this effct, event-by-event.}\\[0.2ex]

{\it [Theory]\ The authors may want to switch off $^3$He Fermi motion in their
  simulations and determine how large and in which kinematics they see a
  difference.  Having evidence of non-negligible nuclear effects at an early
  stage would encourage theorists to extend now their calculations from
  inclusive to exclusive measurements for a timely and correct utilization of
  the data the authors propose to take.  It would also be helpful to elaborate
  on the possible corrections in addition to Fermi motion, such as from binding
  and nucleon off-shell effects, as well as corrections beyond the impulse
  approximation from rescattering or final state interactions of the detected
  proton.}\\[0.2ex]

We agree that if the proton momentum resolution is sufficiently good, it will
be possible to correct for Fermi momentum on an event-by-event basis.  We are
familiar with this technique from our work with the A2 Collaboration at Mainz,
for example.  For the purposes of this proposal we take the more pessimistic
view that the proton resolution is likely not good enough to make this
correction, even though the removal of the Fermi momentum affect would simplify
the physics interpretation of our data and further increase its likely physics
impact.

To get a better estimate of the impact of Fermi momentum, we have run the
generator in a variety of configurations and repeated the analysis of
Sec.~\ref{sec:asym}.  These configurations were:
\begin{itemize}
\item{{\it simple:} All Fermi momentum, multiple scattering, energy loss effects
  turned off.}
\item{{\it fermi:} Same as {\it simple} except that Fermi momentum is turned
  on.  This allows the Fermi momentum effect to be isolated separate from other
  effects.}
\item{{\it mult:} All Fermi momentum, multiple scattering, energy loss effects
  turned on.  This is the default configuration, presented in
  Fig.~\ref{fig:asym1_t}.}
\item{{\it mult\_nofermi:} Same as {\it simple} except that Fermi momentum is
  turned off.  This allows the anticipated effect of correcting Fermi momentum
  event-by-event should the recoil proton momentum resolution be sufficiently
  good to be modeled.}
\end{itemize}

While all five modulations were fit in each case, only the two of physics
interest are shown in Fig.~\ref{fig:asym_nofermi}.  It is clearly seen in panel
a) that the agreement between input and output values is very good when all
Fermi momentum, scattering and energy loss effects are turned off.  This
validates our UML procedure.  The effect of Fermi momentum can be seen by
comparing panels b) and a), or alternately Fig~\ref{fig:asym1_t} with panel c).
Their effects are similar, giving rise to a deviation to the
$\sin(\phi-\phi_S)$ asymmetry of about -0.02, and a deviation to the
$\sin(\phi_S)$ asymmetry of about -0.04.  Panel c) clearly shows that if we are
able to correct the data for Fermi momentum on an event-by-event basis, it is
in principle possible to remove most of this effect on the asymmetries, even
though there is multiple scattering, scattering resolution, and other smearing
effects.

We hope this estimate of Fermi-momentum effects at an early stage will
encourage theorists to extend their calculations for a timely and correct
utilization of our proposed data, as suggested in the Theory review comment.

\subsection{Dialog with Theorists}

{\it [SoLID]\ There are a number of important theory issues raised by this
  proposal.  These probably cannot be fully resolved before re-submission, but
  it will be important to have a clear dialog with relevant theorists (and
  experimentalists) in place...  Both Goloskokov and Kroll, and Liutti and
  Goldstein, have published estimates of $\sigma_T$, based on transversity GPDs
  and a twist-3 helicity-flip pion distribution amplitude.  One or the other of
  these theory groups should be engaged in a discussion of both the
  $|\sigma^y_L|$ and $|\sigma^y_{TT}|$ terms.}\\[0.2ex]

We have been in communication with Goloskokov and Kroll on the physics
objectives of this proposal for some years, and they provided helpful comments
to our 2016 proposal.  With the additional time we had available for the 2017
proposal, Goloskokov and Kroll generously provided new asymmetry calculations,
based on their best estimates of $\sigma_T$, $\sigma_{TT}$ for SoLID
kinematics.  We are confident that approval of this proposal will raise the
interest of other theorists to the physics potential of our measurements, and
we will have dialog with them.\\[0.2ex]

{\it [SoLID]\ The QCD factorization theorem implies color transparency for the
  final state $\pi^-$ in this proposal.  Thus the $^3$He$(e,e'\pi^-)$ final
  state interactions (FSI) are identical with $^3$He$(e,e'p)$, just with a more
  exotic scattering amplitude.  It is not practical to obtain full FSI
  calculations before resubmission, but a dialog should be started both with
  the groups doing FSI calculations, and the groups doing Deep Virtual
  calculations on light nuclei.  Empirically, it will be useful to determine if
  the FSI `peak' lies within the $^3$He$(e,e'\pi^-p)pp$ acceptance of this
  proposal.}\\[0.2ex]

We have made some estimates of FSI effects, based on the empirical
parameterization of $\pi N$ differential cross sections discussed in
Sec.~\ref{sec:fsi} of the Appendix.  The results of our study indicate that the
FSI `peak' does not lie within the acceptance of this proposal and that FSI
effects are expected to be small.  Based on the $\pi N$ differential cross
sections and the fact that there are only two proton spectators in the final
state, we anticipate only about 1\% of events will suffer FSI interactions.
The FSI fraction is weakly dependent on $Q^2$, rising to about 1.2\% for events
with $Q^2>$5 GeV$^2$ in our study.  Of these, a large fraction of the FSI
events are scattered outside the triple coincidence acceptance, reducing the
FSI fraction to $\sim 0.4\%$.  This wil be further reduced by analysis cuts
such as $P_{miss}<1.2$ GeV/c.

Over the longer term, we will consult with theoretical groups for a more
definitive study of FSI effects.  For example, Del Dotto, Kaptari, Pace,
Salme and Scopetta recently published a study of FSI effects in SIDIS from a
transversely polarized $^3$He target \cite{dotto} in SBS, SoLID and EIC
kinematics.  The SIDIS final state has more outgoing particles than DEMP, so
there are more opportunities for FSI interactions there than the simple $\pi^-
ppp$ final state considered here.  Nontheless, they were able to show that the
extracted Sivers and Collins asymmetries are basically independent of FSI,
evaluated within the generalized eikonal approximation and a realistic
distorted spin-dependent spectral function.  A similar calculation for DEMP,
after this proposal is approved, would be a natural extension of their work.


%\newpage
\section{Summary}
The $A_{UT}^{\sin(\phi-\phi_S)}$ transverse single-spin asymmetry in the
exclusive $\vec{n}(e,e'\pi^-)p$ reaction has been noted as being especially
sensitive to the spin-flip generalized parton distribution (GPD) $\tilde{E}$.
Factorization studies have indicated that precocious scaling is likely to set
in at moderate $Q^2\sim 2-4$ GeV$^2$, as opposed to the absolute cross section,
where scaling is not expected until $Q^2>10$ GeV$^2$.
%Furthermore, this observable has been noted as being important for the reliable
%extraction of the charged pion form factor from pion electroproduction.
This relatively low value of $Q^2$ for the expected onset of precocious scaling
is important, because it will be experimentally accessible at Jefferson Lab.
The $A_{UT}^{\sin(\phi_S)}$ asymmetry can also be extracted from the same data,
providing powerful additional GPD model constraints and insight into the role
of transverse photon contributions at small $-t$.

This measurement is complementary to a proposal to measure the longitudinal
photon, transverse nucleon, single-spin asymmetry $A_L^{\perp}$ with the
SHMS+HMS in Hall C \cite{atpi39}.  The good resolution and reproducible
systematic uncertainties of the SHMS+HMS setup allow the L--T separation needed
to reliably measure this quantity.  However, a wide $-t$ coverage is needed to
obtain a good understanding of the asymmetry, and it always been intended to
complement the SHMS+HMS $A_L^{\perp}$ measurement with an unseparated
$A_{UT}^{sin(\phi-\phi_s)}$ measurement using a large solid angle detector.
The high luminosity capabilities of SoLID make it well-suited for this
measurement.  Since an L--T separation is not possible with SoLID, the observed
asymmetry is expected to be diluted by the ratio of the longitudinal cross
section to the unseparated cross section.  This was also true for the
pioneering HERMES measurements, which provided a valuable constraint to models
for the $\tilde{E}$ GPD.  

In our proposal, we will analyze the E12-10-006 event files
off-line to look for $e-\pi^--p$ triple coincidence events in SoLID for the
case where the recoil proton is emitted $8^o<\theta<24^o$.  This study yields
data that are a considerable advance over the HERMES measurement in terms of
kinematic coverage and statistical precision. 
This measurement is also important preparatory work for future
measurements at the EIC, which will allow $A_{UT}^{sin(\phi-\phi_s)}$ to be
directly compared for the $\vec{n}(e,e'\pi^-)p$ and $\vec{p}(e,e'\pi^+)n$
reactions over a broad kinematic range.

%This proposal presents two scenarios: $(a)$ analyze the E12-10-006 event files
%off-line to look for $e-\pi^--p$ triple coincidence events in SoLID for the
%case where the recoil proton is emitted $8^o<\theta<24^o$.  This mode yields
%data that are a considerable advance over the HERMES measurement in terms of
%kinematic coverage and statistical precision.  $(b)$ the improved asymmetry
%data that could be obtained over a broader kinematic range if a proton recoil
%detector (PRD) is constructed to detect those additional protons emitted
%$24^o<\theta<50^o$.  Here, we are seeking PAC approval for scenario $(a)$, as
%well as advice on whether we should pursue the PRD to enable scenario $(b)$
%under the proviso that it have minimal impact upon the running of E12-10-006.

\newpage
\appendix
\section{Monte Carlo model of Deep Exclusive $\pi^{-}$ Production from the
  Neutron in $^{3}$He 
\label{sec:generator}}

The Monte Carlo studies needed for this proposal require a reaction
model for an experimentally unexplored region of kinematics, at higher
values of $Q^2$, $-t$ and $W$ than covered by existing data.  This appendix
describes the model and the constraints used.

\subsection{Definition of the Cross Section and Single-Spin Asymmetries}

The differential cross section for exclusive $\pi$ production from the nucleon
can be written as
\begin{equation}
  \frac{d^{5} \sigma}{dE' d\Omega_{e'} d\Omega_{\pi}} = \Gamma_{V} \frac{d{^2}
  \sigma}{d\Omega_{\pi}}.
\end{equation}
The virtual photon flux factor $\Gamma_{V}$ is defined as
\begin{equation}
  \Gamma_v=\frac{\alpha}{2\pi^2} \frac{E'}{E} \frac{K}{Q^2}\frac{1}{1-\epsilon},
\end{equation}
where $\alpha$ is the fine structure constant, $K$ is the energy of real photon
equal to the photon energy required to create a system with invariant mass
equal to $W$ and $\epsilon$ is the polarization of the virtual photon.
\begin{equation}
  K=(W^2-M_p^2)/(2 M_p)
\end{equation}
\begin{equation}
  \epsilon=\left(1+\frac{2 |\mathbf{q}|^2}{Q^2} \tan^2\frac{\theta_{e}}{2}
  \right)^{-1},
\end{equation}
where $\theta_{e}$ is the scattering angle of scattered electron.

The two-fold differential cross section $\frac{d{^2} \sigma}{d\Omega_{\pi}}$ in
the lab frame can be expressed in terms of the invariant cross section
in centre of mass frame of the photon and nucleon,
\begin{equation}
  \frac{d^2 \sigma}{d\Omega_\pi}= J \frac{d^2 \sigma}{dt d\phi},
\end{equation}
where $J$ is the Jacobian of transformation of coordinates from lab
$\Omega_{\pi}$ to $t$ and $\phi$ (CM). 

Following Ref.~\cite{hermes-thesis}, we consider separately the unpolarized and
polarized target contributions to the invariant photon nucleon cross section,
\begin{equation}
  d\sigma = d\sigma_{UU} + d\sigma_{UT}.
  \label{eqn:cross-1}
\end{equation}

In the one-photon exchange approximation, the unpolarized nucleon 
cross section for $n(e,e^{\prime}\pi^{-})p$
can be expressed in four terms. Two terms correspond
to the polarization states of the virtual photon (L and T) and two states
correspond to the interference of polarization states (LT and TT),
\begin{equation}
  d\sigma_{UU} =  \epsilon  \frac{d\sigma_{\mathrm{L}}}{dt}
  + \frac{d\sigma_{\mathrm{T}}}{dt} + 
  \sqrt{2\epsilon (\epsilon +1)} \frac{d\sigma_{\mathrm{LT}}}{dt} \cos{\phi}
  + \epsilon  \frac{d\sigma_{\mathrm{TT}}}{dt} \cos{2 \phi},
  \label{eqn:cross-2}
\end{equation}
where $\phi$ is the angle between lepton plane and hadron plane
(Fig.~\ref{fig:planes}). 
The first two terms of Eqn.~\ref{eqn:cross-2} correspond to the
polarization states of the virtual photon (L and T) and last two terms
correspond to the interference of polarization states (LT and TT).  $\epsilon$
is the ratio of longitudinal to transverse virtual-photon fluxes
\begin{equation}
  \epsilon=\left(1+\frac{2
  |\mathbf{q}|^2}{Q^2} \tan^2\frac{\theta_{e}}{2} \right)^{-1}.
\end{equation}
The constraints used to parameterize $d\sigma_{UU}$ are described in
Sec.~\ref{sec:model}.

The additional contribution when the target nucleon is transversely polarized
can be parameterized \cite{Di05,hermes-thesis} as
\begin{equation}
  d\sigma_{UT} = -\frac{P_{T}}{\sqrt{1 - \sin^{2}{\theta} \sin^{2}{\phi_{S}} }
  } \sum_{k=1}^{6} \sin{ (\mu\phi + \lambda\phi_{S})
  } \Sigma_{k},
\label{eqn:cross-3}
\end{equation}
where $\phi_{S}$ is angle between the target polarization and lepton planes
(Fig.~\ref{fig:planes}), the $\Sigma_{k}$ are given by
\begin{equation}
  \Sigma_{k} = A_{UT}^{\sin{( \mu\phi+\lambda\phi_{S})_{k}}} \times
  d\sigma_{UU}(\phi),
  \label{eqn:sigmas}
\end{equation}
and the $\sin (\mu\phi + \lambda\phi_{S})$ are the different azimuthal
modulations.  The calculation of $d\sigma_{UT}$ in the event generator is
described in Sec.~\ref{sec:asymparametrizations}.

\subsection{Cross Section Model for Higher $Q^2$ Kinematics
\label{sec:model}}

\subsubsection{Constraints}

All of the following data were used as constraints on the parameterizations
used in this model.
\begin{itemize}
\item
From Hall C, precise $L/T$ separated experimental data of exclusive
electroproduction of $\pi^{-}$ on $^2$H are available up to $Q^2=2.57$ GeV$^2$,
$-t=0.350$ GeV$^2$ and $W=2.168$ GeV \cite{gmhuber-2}.
\item
Also from Hall C, precise $L/T$ separated experimental data of exclusive
electroproduction of $\pi^{+}$ on $^1$H are available up to $Q^2=2.703$
GeV$^2$, $-t=0.365$ GeV$^2$ and $W=2.127$ GeV \cite{Fpi2}, and separated
$\sigma_{L}$ and $\sigma_{T}$ are measured up to $Q^2=4.703$ GeV$^2$ and
$W=2.2$ GeV \cite{hallc-1} and \cite{hallc-2}.
\item
CLAS experiment E99-105 measured the unseparated exclusive $\pi^+$ cross
section from $^1$H at $Q^2$ up to $4.35$ GeV$^2$ and $-t$ up to $4.5$ GeV$^2$
\cite{park}.
\item
The HERMES collaboration measured the unseparated cross section for $Q^2$=3.44
GeV$^2$ and 5.4 GeV$^2$ \cite{hermes} at $W$=4 GeV.
\end{itemize}

An additional constraint in our parameterization comes from the
Vrancx-Ryckebusch (VR) model \cite{vr}.  This is a Regge model with a
parametrization of the deep inelastic scattering amplitude added to improve the
description of $\sigma_{T}$.  The description of $\sigma_{L}$ in the model is
constrained by a fit to the Hall C $p(e,e'\pi^+)n$ data from
Ref.~\cite{Fpi2}.  The model provides a good description of exclusive
charged pion electroproduction above the resonance region.  It has been checked
for reliability against the Hall B and C data listed above, for $W>2$ GeV,
$Q^2$ from 0.35 to 4.98 GeV$^2$.  The model is believed to be reliable for
$-t\leq$0.5 GeV$^2$, but it overshoots the data for $-t>$0.5 GeV$^2$.

\begin{figure}[!hbt]
    \centering
    \includegraphics[width=0.85\textwidth]{./figures/pimsigma_qsq17.png}
%    \includegraphics[width=0.85\textwidth]{./figures/pimsigma_qsq.pdf}
    \caption{ A comparison of last six points of table $v$ of \cite{gmhuber-2},
      the VR model, and our parametrization values vs. $Q^{2}$ for $\pi^{-}$
      electroproduction. Experimental data are shown in blue circles, VR model
      is shown in red triangles, and our parametrization is shown in black
      boxes. In each graph, the value of $-t$ is decreasing left to right from
      a maximum value 0.35 GeV$^2$ to 0.15 GeV$^2$. Value of $W$ also decreases
      left to Right from 2.2978 GeV to 2.1688 GeV.}
    \label{fig:expvrfit}
\end{figure}

\begin{figure}[!hbt]
    \centering
    \includegraphics[width=0.85\textwidth]{./figures/pimsigma_t17.png}
%    \includegraphics[width=0.85\textwidth]{./figures/pimFit.pdf}
    \caption{ A comparison of parametrized $\sigma_{L,T,LT,TT}$ and VR model
    values at $Q^2$ = 4.421 GeV$^2$ and $W = 3.0$ GeV.  Black points are VR
    model values and blue line is parametrized $\sigma_{L,T,LT,TT}$ given by
    equations $\ref{equation:l-fit}$ to $\ref{equation:tt-fit}$. }
    \label{fig:sigall}
\end{figure}

\subsubsection{Parametrization of $\sigma_{L}$, $\sigma_{T}$, $\sigma_{LT}$, 
$\&$ $\sigma_{TT}$
\label{sec:parametrization}}

For exclusive DEMP in SoLID, the kinematic region of interest for
parametrization of $\sigma_{L,T,LT,TT}$ is $Q^2$ from 4.0 GeV to 7.5 GeV$^2$,
$-t$ from 0 GeV$^2$ to 1.0 GeV$^2$, and we set $W=3.0$ GeV. After the
parametrization of $\sigma_{L,T,LT,TT}$ for $-t$ and $Q^2$, we used the same
$W$ dependence given by \cite{Fpi2}, which is $(W^2-M^2)^{-2}$ where $M$ is
the proton mass.  Our parametrization of all four cross sections is given in
equations $\ref{equation:l-fit}$ to $\ref{equation:tt-fit}$.

\begin{equation}
        \sigma_{L} = \exp{(P_1(Q^2) + |t| * P^{\prime}_1(Q^2))}
        + \exp{(P_2(Q^2) + |t| * P^{\prime}_2(Q^2))}
     \label{equation:l-fit}
\end{equation}

\begin{equation}
        \sigma_{T} = \frac{\exp{(P_1(Q^2) + |t| *
        P^{\prime}_1(Q^2))}}{P_{1}(|t|)}
     \label{equation:t-fit}
\end{equation}

\begin{equation}
        \sigma_{LT} = P_{5}(t(Q^2))
     \label{equation:lt-fit}
\end{equation}

\begin{equation}
        \sigma_{TT} = P_{5}(t(Q^2))       
     \label{equation:tt-fit}
\end{equation}

Here, the parameters $P_{i}$ are polynomial functions of $i^{th}$ order. Each
coefficient ($P_{i}$) of the fifth order equations $\ref{equation:lt-fit}$ and
$\ref{equation:tt-fit}$ is a further second order polynomial of $Q^2$. Deep
exclusive $\pi^{-}$ events are generated using a C++ code. The quality of
parametrization is checked by plotting the parametrization functions of
$\sigma_{L,T,LT,TT}$ versus the existing data and the VR model are shown in 
Figs. \ref{fig:expvrfit}, \ref{fig:sigall}.

\subsection{Parametrization of six Single-Spin Asymmetries
\label{sec:asymparametrizations}}

\begin{figure}[!hbt]
    \centering
    \includegraphics[width=0.85\textwidth]{./figures/AsymPlots.pdf}
    \caption{Parametrization of the five single spin asymmetries
      $A_{UT}^{\sin(\mu\phi+\lambda\phi_s)_k}$ vs. $t'$ used in the event
      generator for this proposal.  The points are the calculations by
      Goloskokov and Kroll \cite{GoPC} and the curves are our fit.}
    \label{fig:asym-1}
\end{figure}

The single-spin asymmetries calculated for us by S.V Goloskokov and P. Kroll
\cite{GoPC} have been used to approximate $d\sigma_{UT}$ in the
DEMP event generator.  Their $A_{UT}^{\sin(\mu\phi+\lambda\phi_s)_k}$ values are
at discrete values of $Q^2$ from 4.107 to 7.167~GeV$^2$, $W$ from 2.362 to
3.191~GeV, and $t'$ from 0~GeV$^2$ to 0.5~GeV$^2$.  There are six different 
azimuthal modulations defined as follows:
\begin{center}
\begin{tabular}{ l|l }
  $k$ & $\sin(\mu\phi+\lambda\phi_s)$ \\
  \hline
  $1$ & $\sin(\phi-\phi_s)$ \\
  $2$ & $\sin(\phi+\phi_s)$ \\
  $3$ & $\sin(\phi_s)$ \\
  $4$ & $\sin(2\phi-\phi_s)$ \\
  $5$ & $\sin(3\phi-\phi_s)$ \\
  $6$ & $\sin(2\phi+\phi_s)$
\end{tabular}
\end{center}

These asymmetries are used to give
\begin{align}
  \Sigma_k &= d\sigma_{UU}(\phi)A_{UT}^{\sin(\mu\phi+\lambda\phi_s)_k} \\
  d\sigma_{UT}&=-\frac{P_T}{\sqrt{1-\sin^2\theta
      \sin^2\phi_s}}\sum_{k=1}^{6}\sin(\mu\phi+\lambda\phi_s)_k\Sigma_k
\end{align}

The $k=6$ asymmetry ($(\mu,\lambda)=(2,1)$) is not included in their
calculation, and is taken to be zero, in accordance with the HERMES data
\cite{hermes-thesis}. The other five are calculated based on fits to their
values.  The following fits were used:
\begin{align}
  A_{UT}^{\sin(\mu\phi+\lambda\phi_s)_k} =
  \begin{cases}
    ae^{bx}-(a+c)e^{dx}+c, &\quad k=1 \\
    ae^{bx}+c, &\quad k=2,3,4,5
  \end{cases}
\end{align}
where $a$, $b$, $c$, and $d$ are fit parameters. The forms of these functions
were chosen only to closely match the shape of the simulated data and are not
based on any physical principle. These fits are done for each given value of
$Q^2$ independently, as shown in Fig.~\ref{fig:asym-1}.  During event
generation, the asymmetry is calculated from the fit for the two nearest values
of $Q^2$. The asymmetry for the given event is then approximated by linear
interpolation of the nearest values.

\subsection{Target Neutron Fermi Momentum 
\label{sec:fermimotion}}

\begin{figure}[!hbt]
    \centering
    \includegraphics[height=7cm]{./figures/Fermi.pdf}
%    \includegraphics[width=4.0in,height=2.5in]{04.eps}
    \caption{Fermi momentum spectral function of a target nucleon in $^3$He
      generated according to the Argonne potential of Ref \cite{fermipaper}.
      The horizontal axis is nucleon momentum in MeV/c.}
    \label{fig:fermi}
\end{figure}

A histogram of the spectral function of $^3$He is shown in
Fig.~\ref{fig:fermi}, generated according to Ref.~\cite{fermipaper}. Neutron
momenta up to 300 MeV/c are generated according to this distribution, uniformly
distributed in spherical coordinates. The quasi-free collision between the
virtual photon and moving neutron is then transformed to the fixed neutron
frame, after which the parameterizations of Secs.~\ref{sec:parametrization},
\ref{sec:asymparametrizations} are applied. The outgoing particles are then
transformed back to the lab frame for tracking.

\subsection{Energy Loss and Multiple Scattering
\label{sec:energyloss}}

There is energy loss for the incoming electron $e$, and the three outgoing
particles: scattered electron $e^{\prime}$, $\pi^{-}$ and recoil proton $p$ by
bremsstrahlung and ionization.  The same code as given in SAMC (Hall A Single
Arm Monte Carlo) \cite{samc} is used.  This code is based on Sec. 33 (Passage
of articles through matter) of the Review of Particle Physics by the Particle
Data Group \cite{pdg}.

Incoming electron and three out going particles are deflected small angles by
multiple scattering in the target, target window and in the air. This small
deflection in the polar angle theta $\theta$ is calculated according to
Subsection 33.3 (Multiple scattering through small angles) of the Review of Particle
Physics by the Particle Data Group \cite{pdg}.

The incoming electron loses energy by bremsstrahlung and ionization, and
suffers multiple scattering, in the target and in the target window.  Both of
these processes and the choice of neutron Fermi momentum are applied before the
cross section terms $\sigma_{uu}$ and $\sigma_{UT}$ are calculated from the
`vertex' kinematic quantities.

The scattered electron, pion and proton lose energy by bremsstrahlung and
ionization, and suffer multiple scattering, in the target, target window and in
the air.  The energy and momentum of these particles are corrected according to
these processes prior to particle tracking.

\subsection{Final State Interactions
\label{sec:fsi}}

A separate version of the model was made in which the outgoing $\pi^-$ suffers
$\pi N$ final state interactions (FSI) with one of the recoil $p$ in the
residual nucleus.

The scattering of $\pi^-$ by protons involves both the $T=1/2$ and $T=3/2$
isospin states.  We model the $\pi^- p$ scattering via the empirical phase
shift analysis of Rowe, Solomon and Landau \cite{rowe}.  In this case,
the amplitude for the scattering of a spin-zero particle by a particle of
spin-$\frac{1}{2}$ is described for each isospin channel by a set of partial
wave amplitudes $f^{(+)}_{\ell}$, $f^{(-)}_{\ell}$ for the
$j=\ell\pm\frac{1}{2}$ states.  In terms of phase shifts, $f^{(+)}_{\ell}$ is
\begin{equation}
f^{(+)}_{\ell}\equiv\frac{1}{2ik}(e^{2i\delta^+_{\ell}} -1)
\end{equation}
with a similar expression for $f^{(-)}_{\ell}$.  The phase shift
$\delta^+_{\ell}$ will be complex if there is any inelasticity, which will
occur for example for the reactions
\begin{equation}
\begin{split}
\pi^- + p &\rightarrow \pi^+ \pi^- p \\
          &\rightarrow \pi^0 \pi^0 n \\
          &\rightarrow \pi^- \pi^0 p. \\
\end{split}
\end{equation}
The differential cross-section is written in terms of these phase shifts as
\begin{equation}
\frac{d\sigma}{d\Omega}=\biggl[
\biggl| \frac{1}{k}\sum_{\ell}[(\ell +1) 
f^{(+)}_{\ell} +\ell f^{(-)}_{\ell}] P_{\ell}(\cos\theta)\biggr|^2 +
\biggl| \frac{i}{k}\sum_{\ell}[f^{(+)}_{\ell} -f^{(-)}_{\ell}]\sin\theta 
 \frac{d P_{\ell}(\cos\theta)}{d\cos\theta}\biggr|^2
\biggr].
\label{eqn:piN}
\end{equation}

$\pi N$ phase shifts have been determined experimentally from near threshold up
to several GeV.  The dominant phase shifts for the $L_{2T,2J}$ states:
S$_{11}$, S$_{31}$, P$_{11}$, P$_{13}$, P$_{31}$, P$_{33}$ are very accurately
known for centre of mass momenta up to 350 MeV/c.  The phase shift
parameterizations used in our model are dominated by the $\Delta(1232)$ and
$N^*(1440)$ resonances.  For further information, please see
Ref.~\cite{ericson}.

In our implementation of the FSI process, the Fermi momentum of one of the
recoil protons was chosen according to Sec.~\ref{sec:fermimotion} and collided
with the outgoing $\pi^-$ in their mutual centre of mass frame.  Outgoing
$\pi^-$ $N$ were randomly generated, and the events
weighted according to the differential cross setions of Eqn.~\ref{eqn:piN}.
Events were generated for both the FSI and non-FSI versions of the
model and the results compared, as described in the main text.

%\input{proton_detector.tex}
%\newpage
%\begin{thebibliography}{99}
\clearpage
\begin{thebibliography}{}

\bibitem{solid:e12-10-006} 
  Approved SoLID SIDIS experiment E12-10-006,
  ``Target Single Spin Asymmetry in Semi-Inclusive Deep-Inelastic
  $(e,e'\pi^{\pm})$ Reaction on a Transversely Polarized $\mathrm{^{3}He}$
  Target at 11 GeV'',\\
$https://www.jlab.org/exp\_prog/proposals/14/E12-10-006A.pdf$

\bibitem{solid_pcdr} 
  SoLID Collaboration, ``Solenoial Large Intensity Device Preliminary
  Conceptual Design Report'',\\
  $http://hallaweb.jlab.org/12GeV/SoLID/files/solid\_precdr.pdf$
  
\bibitem{Di05} M. Diehl, S. Sapeta, Eur. Phys. J. C {\bf 41} (2005) 515,
  arXiv:hep-ph/0503023.

\bibitem{hermes10} A. Airapetian, Phys. Lett. {\bf B 682} (2010) 345-350,
  arXiv:0907.2596 [hep-ex].

\bibitem{atpi39} PR12-12-005:
D. Dutta, D. Gaskell, W. Hersman, G.M. Huber, et al., ``The Longitudinal
Photon, Transverse Nucleon, Single-Spin Asymmetry in Exclusive Pion
Production''.

\bibitem{Di00} M. Diehl, Contribution to the eRHIC White Paper,
arXiv:hep-ph/0010200.

\bibitem{Co97} J.C. Collins, L. Frankfurt, M. Strikman, Phys. Rev. D {\bf 56}
  (1997) 2982.

\bibitem{Go01} K. Goeke, M.V. Polyakov, M. Vanderhaeghen,
  Prog. Part. Nucl. Phys. {\bf 47} (2001) 401-515.

\bibitem{Ra00} A.V. Radyushkin, arXiv:hep-ph/0101225.

\bibitem{Th01} A.W. Thomas, W. Weise, ``The Structure of the Nucleon'',
  J. Wiley-VCH, 2001.

\bibitem{Ma69} R.E. Marshak, Riazuddin, C.P. Ryan, ``Theory of Weak
  Interactions in Particle Physics'', J. Wiley, 1969.

\bibitem{Pe00} M. Penttinen, M.V. Polyakov, K. Goeke, Phys. Rev. C {\bf 62}
  (2000) 014024 1-11.

\bibitem{Be01} A.V. Belitsky, D. Mueller, Phys. Lett. {\bf B 513}
  (2001) 349-360.

\bibitem{Go10} S.~V.~Goloskokov and P.~Kroll, Eur.\ Phys.\ J.\ C {\bf 65},
  137 (2010), arXiv:0906.0460 [hep-ph].

\bibitem{Fr99} L.L. Frankfurt, P.V. Pobylitsa, M.V. Polyakov, M. Strikman,
  Phys. Rev. D {\bf 60} (1999) 014010 1-11.

\bibitem{hermes-thesis} Ivana Hristova, ``Transverse-Target Single-Spin
  Azimuthal Asymmetry in Hard Exclusive Electroproduction of Single Pions at
  HERMES'', Ph.D. thesis, Humboldt University of Berlin, December 13, 2007.\\
$http://bib-pubdb1.desy.de/record/288884/files/desy-thesis-07-041.pdf?version=1$

\bibitem{Va99} M. Vanderhaeghen, P.A.M. Guichon, M. Guidal, Phys. Rev. D 
  {\bf 60} (1999) 094017 1-28.

\bibitem{Fr00} L.L. Frankfurt, M.V. Polyakov, M. Strikman, M. Vanderhaeghen,
  Phys. Rev. Lett. {\bf 84} (2000) 2589-2592.

\bibitem{belitsky} A.V. Belitsky, CIPANP 2003 proceedings.  arXiv:
  hep-ph/0307256.

\bibitem{Ma99} L. Mankiewicz, G. Piller, A. Radyushkin, Eur. Phys. J. {\bf C
    10} (1999) 307-312.

\bibitem{Ca90} C.E. Carlson, J. Milana, Phys. Rev. Lett. {\bf 65} (1990) 1717.

\bibitem{12GeV} E12-06-101, ``Measurement of the Charged Pion Form Factor to
  High $Q^2$'', G.M. Huber, D. Gaskell, spokespersons.

%\bibitem{Ba73} A. Bartl, W. Majerotto, Nucl. Phys. {\bf B62} (1973) 267-285.

\bibitem{clas} pCDR for the Science and Experimental Equipment for the 12 GeV
  Upgrade of CEBAF, June, 2004.\\
V. Burkert et al., PAC18 Review of the Science Driving the 12 GeV Upgrade, 
  July, 2000.

\bibitem{GoPC} S.~V.~Goloskokov and P.~Kroll, private communications 2009-2017.

\bibitem{Go11} S.~V.~Goloskokov and P.~Kroll, Eur.\ Phys.\ J.\ A {\bf 47},
  112 (2011), arXiv:1106.4897 [hep-ph].

\bibitem{Fpi2} H.P. Blok, et al., Phys. Rev. C {\bf 78} (2008) 045202.

\bibitem{solid:e12-11-007} 
  Approved SoLID SIDIS experiment E12-11-007,
``Asymmetries in Semi-Inclusive Deep-Inelastic  $(e,e'\pi^{\pm})$ Reactions on a
Longitudinally Polarized $\mathrm{^{3}He}$ Target at 8.8 and 11 GeV'',\\
$https://www.jlab.org/exp_prog/PACpage/PAC37/proposals/Proposals/New\%20Proposals/PR-11-007.pdf$

\bibitem{solid:e12-11-108} 
  Approved SoLID SIDIS experiment E12-11-108,
``Target Single Spin Asymmetry in Semi-Inclusive Deep-Inelastic
$(e,e'\pi^{\pm})$ Reactions on a Transversely Polarized Proton Target'',\\
$https://www.jlab.org/exp\_prog/proposals/11/PR12-11-108.pdf$

\bibitem{solid:e12-12-006} 
  Approved SoLID J/$\mathrm{\psi}$ experiment E12-12-006A,
  ``Near Threshold Electroproduction of J/$\mathrm{\psi}$ at 11 GeV'',\\
$https://www.jlab.org/exp_prog/proposals/12/PR12-12-006.pdf$

\bibitem{solid:e12-10-006A} 
  Approved SoLID SIDIS experiment E12-10-006A,
  ``Dihadron Electroproduction in DIS with  Transversely Polarized
  $\mathrm{^{3}He}$ Target at 11 and 8.8 GeV'',\\
$https://www.jlab.org/exp_prog/proposals/14/E12-10-006A.pdf$

\bibitem{solid:e12-11-008A}
   Approved SoLID SIDIS experiment E12-11-108A,
``Target Single Spin Asymmetry Measurements in the Inclusive Deep-Inelastic
$\vec{N}(e,e')$ Reaction on Transversely Polarized Proton and Neutron
($\mathrm{^{3}He}$) Targets using the SoLID Spectrometer''\\
$https://www.jlab.org/exp_prog/proposals/14/E12-11-108A_E12-10-006A.pdf$

\bibitem{solid:e12-10-007} 
  Approved SoLID PVDIS experiment E12-10-007,
  ``Precision Measurement of Parity-violation in Deep Inelastic Scattering Over
  a Broad Kinematic Range'',\\
  $https://www.jlab.org/exp\_prog/PACpage/PAC37/proposals/Proposals/Previously\%20Approved/E12-10-007.pdf$

\bibitem{3hepol1}J. L. Friar et al., Phys. Rev. C 42, (1990) 2310; C. Ciofi
degli Atti, and S. Scopetta, Phys. Lett. {\bf B404}, (1997) 223; R.W. Schulze
and P.U. Sauer, Phys. Rev. {\bf C56} (1997) 2293; F. Bissey, A.W. Thomas, and
I.R. Afnan, Phys. Rev. {\bf C64}, (2001) 024004.

\bibitem{gmhuber-2} G.~M.~Huber $\textit{et}$ $\textit{al}$.,
Phys.~Rev.~\textbf{C91}, 015202 (2015).

\bibitem{hallc-1} T.~Horn $\textit{et}$ $\textit{al}$.,
Phys.~Rev.~Lett~\textbf{97}, 192001 (2006).

\bibitem{hallc-2} T.~Horn $\textit{et}$ $\textit{al}$.,
Phys.~Rev.~\textbf{C78}, 058201 (2008).

\bibitem{park} K. Park $\textit{et}$ $\textit{al}$.,
Eur. Phys. J. A \textbf{49}, 16 (2013).\\
$http://clas.sinp.msu.ru/cgi-bin/jlab/db.cgi?eid=36;search=on$

\bibitem{hermes} A.~Airapetian $\textit{et}$ $\textit{al}$.,
Phys.~Lett.~B.~\textbf{659}, (2008).

\bibitem{dotto} A. Del Dotto, L.P. Kaptari, E. Pace, G. Salme, S. Scopetta,
arXiv:1704.06182 [nucl-th].

\bibitem{vr} T.~Vrancx and J.~Ryckebusch., Phys.~Rev.~\textbf{C89},
025203 (2014).

\bibitem{fermipaper}    R.~Schivavilla and V.~R.~Pandharipande.,
Nucl.~Phys.~A.~\textbf{449}, 219 (1986).

\bibitem{samc} Hall-A Single-Arm Monte Carlo Simulation Tool,\\
$https://userweb.jlab.org/ \sim yez/Work/SAMC/$

\bibitem{pdg} Particle Data Group, Chinese Physics C, \textbf{40}, 100001
(2016).

\bibitem{rowe} G.~Rowe, M.~Solomon, R.H.~Landau, Phys.~Rev.~C~\textbf{18}, 584
(1978).

\bibitem{ericson} T. Ericson, W. Weise, ``Pions and Nuclei'', Oxford University
Press (1988).

%\bibitem{sft_zye} Z. Ye, A Prototype Project of a new Scintillating Fiber
%Tracker, 2014 JSA Postdoctoral Fellowship,\\
%``$https://userweb.jlab.org/~yez/Work/SFT/SFT\_prop.pdf$''

\end{thebibliography}

%\end{thebibliography}

\end{document}
