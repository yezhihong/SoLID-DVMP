\section{Responses to Items Identified in the 2016 Review}

Our 2016 SoLID Run-Group proposal was deemed to be of high scientific merit,
but there were a number of technical questions that were asked to be studied
before final approval can be given.  The following list is compiled from the
TAC, Theory, and SoLID review reports, reordered according to topic.  Our 
response to each item is also given.

\subsection{SoLID Acceptance Simulations
\label{sec:resp_gemc}}

{\it [TAC]\ The simulations for this measurement may benefit from tracking
  DEMP events through the full SoLID GEANT4 simulation (GEMC), particularly for
  kinematics with the lowest momentum protons (300~MeV/c).}\\[0.2ex]

GEMC is a GEANT4-based simulation framework designed mainly for CLAS12. SoLID has adopted this framework to do some preliminary studies but implementation of the realistic SoLID detector information is still underway.  
To address this comment, we have
used our generator in two different types of simulations.  The projected
results in Secs.~\ref{sec:rates}, \ref{sec:asym} incorporate electron
radiation, multiple scattering and ionization energy loss in the generator.
Then for every detected particle in each event, we smeared the momenta and
angles of the final state particle by the detector resolutions based on tracking study, 
and then used the acceptance profiles
obtained with the SoLID-SIDIS GEANT4 simulation.

In the second study, the event generator generates the LUND-format file for
the three outgoing particles, $e^-$, $\pi^-$, $p$. 
We can store up to eight variables ($Q^2$, $-t$, $W$, etc.) in the header 
lines of the LUND file.  Events are generated with 
bremsstrahlung and multiple scattering for the incoming electron and
Fermi motion of the target neutron. 
Events are generated uniformly over the 40~cm length of
the SIDIS $^3$He target and a $25 mm \times 25 mm$ raster size is
assumed.  Particle acceptance is studied with the SoLID SIDIS 
GEANT4 configuration, incorporating multiple scattering and energy loss, as
well as the individual detector apertures via the GEANT4 framework.
The GEMC flux tree is used to see if a particle hits a specific detector. 

In this second study, a DEMP event is defined as the following `trigger':
\begin{center}
DEMP Event: =  $T^{e'}_{LA+FA}$ and $T^{\pi^{-}}_{FA}$  and $T^{p}_{LA+FA}$,
\end{center}
where
\begin{itemize}
\item  $T^{e'}_{LA+FA}$ is true if an electron hits the virtual planes of (LGC and
HGC and EC-FA) or EC-LA,
\item $T^{\pi^{-}}_{FA}$ is true if a pion hits the virtual planes of HGC and EC-FA and
MRPC,
\item $T^{p}_{LA+FA}$ is true if a proton hits the virtual planes of (MRPC and SPD-FA)
or SPD-LA.
\end{itemize}

\begin{figure}[!ht]
 \begin{center}
      \includegraphics[width=0.65\textwidth]{./figures/demp_Wp-x.png}
      \includegraphics[width=0.65\textwidth]{./figures/demp_W-x.png}
   \caption{\footnotesize{{\em Top:} $W'$ vs. $x$ coverage, {\em Bottom:} $W$
       vs. $x$ coverage of triple coincidence $e$+$\pi^-$+$p$ DEMP events as
       simulated by GEMC.  A $W>$2~GeV cut was applied.}}
  \label{fig:Wp_x}
  \end{center}
\end{figure}

Fig.~\ref{fig:p_theta} shows the acceptances for all three particles when an
event passes these DEMP event triple-coincidence conditions.  As expected, the
results resemble those obtained in the first study, with the low
momentum proton acceptance now being more accurately modeled. 
Because the DEMP process has very low physics rate, and meanwhile, it takes huge amount of computer time  to accumulate enough statistics, some regions may have stronger statistical fluctuation effect. However, the experimental requirement of detecting protons is more relax compared with the detection of electrons and pions, simply because we only reply on the Time-of-Flight information to separate protons from other positive particles. We will perform more carefully study of reconstructing low momentum protons but we believe the SoLID detector systems will have good coverage of detecting protons.

It is important to note that the angles plotted in Fig.~\ref{fig:p_theta} are not in the
detector frame, but correspond to the reaction vertex location within
the simulated target volume for that event.  This causes the gap between the FA
and LA angular acceptances to be largely blurred out, as well as a slight
enlargement of the accepted angular range.  Similarly, the plotted momenta
are the emitted particle values, not the detected values, corresponding to what
would be obtained in the data analysis after correcting for energy loss.  In 
comparison to the first study, this difference in plotting variables causes the 
lower range of the detected proton momentum to appear to have shifted upward
slightly, to 300-350~MeV/c.  However, the actual difference is very small and
the overall triple-coincidence rate and accepted $t$-range are only marginally
affected.

Figs.~\ref{fig:kin_cor}, \ref{fig:Q2_t_bin} also are the result of GEMC
simulations.  Finally, because there were some oral questions about the $W'$
acceptance (invariant mass of the hadronic system without the detected
$\pi^-$), we show both the $W$ and $W'$ coverage versus $x$ in
Fig.~\ref{fig:Wp_x}.  As expected for an exclusive reaction, this quantity is
sharply peaked at the mass of the recoil proton, with a small tail due to
nuclear effects.  In contrast, the SIDIS events are distributed over a broad
range $1.5<W'<3.5$ GeV \cite{solid:e12-10-006}.

We would like to stress that the SoLID GEMC simulation tool is still under massive development and the detector designs are still needed to be finalized. Together with other approved physics programs on SoLID, we plan to perform more careful study when the SoLID detector designs are completed and the full simulation software is available.

\subsection{Experimental Background
\label{sec:resp_bkd}}

{\it [SoLID]\ The committee is convinced that the SIDIS background is likely
  not a major problem.  However, an alternate approach (rather than SIDIS
  fragmentation functions) could be used.  The primary background channel under
  study is $^3$He$(e,e'\pi^-)pp$ with the two undetected protons as spectators.
  The continuum background that can leak under the quasi-exclusive peak can be
  of the form $e+n\rightarrow e'+\pi^-+\Delta^+$ with the $\Delta^+$ decaying
  to $p+\pi^0$.}\\[0.2ex]

{\it [Theory]\ The authors may want, however, to expand on possible
  contamination arising from $\Delta^{++}$ production on bound protons, and
  subsequent decay into $\pi^+$ and $p$.}\\
We take this comment to mean the $e+p\rightarrow e'+\pi^-+\Delta^{++}$
background reaction, as otherwise there is no $\pi^-$ in the final state to
satisfy the offline event finder.\\[0.2ex]

Neither $\pi^-\Delta$ final state is expected to be a substantial source of
contamination.  We have investigated and compared the kinematics of the
$\pi^-n$ and $\pi^-\Delta$ final states.  The missing momentum of the
$\pi^-\Delta$ state is about 500~MeV/c higher than the $\pi^-n$ final state,
very similar to the difference between the DEMP and SIDIS distributions in
Fig.~\ref{fig:missing_mom}.  A cut at 1.2~GeV/c missing momentum cut will 
discriminate against
%eliminate about 50\% of the
$\pi^-\Delta$ events.  In addition, the $\pi^-\Delta$ final state is centered
about 300 MeV higher in missing mass than the $\pi^-n$ state, and will be
further suppressed by a cut of approximately $M_{miss}<$1.05~GeV.  We estimate
the $\pi^-\Delta$ contamination remaining after the application of both cuts to
be very similar to that already shown in Fig.~\ref{fig:missing_mass}.

\subsection{Resolution and Energy Loss}

{\it [SoLID]\ The effects of Fermi-smearing, detector resolution, ionization
  energy loss and bremsstrahlung need to be clarified.  Although they seem to
  all be included in Figs.~15 and 16, it was not clear which curves included
  which effects.}\\[0.2ex]

We have revised the text to make this more clear.  All of the results shown
incorporate all these effects.

\subsection{Projected Uncertainties}

{\it [SoLID]\ The extraction of the term $|\sigma^y_{TT}+2\epsilon\sigma^y_L|$
  in Eqn.~8 from the other $\sin\beta$ and $\cos\beta$ terms requires good
  knowledge of the $\beta$-acceptance in each $t$-bin.  This should be shown,
  in addition to the acceptance plots of Fig.~12.}\\[0.2ex]

We have put much more effort into understanding the $(\phi,\phi_S)$ coverage of
each $t$-bin, and we hope the proposal more clearly addresses both
the goals of the experiment and the potential limitations.  As presented in
Sec.~\ref{sec:asym}, we have used the same unbinned maximum likelihood (UML)
analysis that was used by the HERMES Collaboration in the analysis of their
experimental data.  It is clear that the large azimuthal acceptance and high
luminosity capabilities of the SoLID detector makes it very well suited to this
measurement, and there should be no major obstacles to cleanly identify the
desired $\sin(\phi-\phi_S)$ and $\sin(\phi_S)$ asymmetries.

The two limitations we face are the following:
\begin{itemize}
\item{the effect of $^3$He Fermi momentum.  This is discussed more fully in
  Sec.~\ref{sec:resp_fermi}.}
\item{we are unlikely to extract definitive values for the more rapidly varying
  $\sin(2\phi-\phi_{S})$, $\sin(2\phi+\phi_{S})$, $\sin(3\phi-\phi_{S})$
  asymmetries after all statistical and systematic uncertainties are included.
  However, we have shown that this should not adversely affect the main physics
  goals of our measurement.}
\end{itemize}

We are confident that we can obtain a significant physics result should this
proposal be accepted.  and can help justify funding for the SoLID project as
one of its flagship measurements.
\\[0.2ex]

{\it [SoLID]\ The collaboration should attempt to quantify the projected
  precision of the measured spin-dependent cross section.  Although the
  asymmetry may have a smaller error bar, the spin-dependent cross section
  difference has a simpler interpretation.}\\[0.2ex]

One of the biggest advantages of DEMP measurements with transversely polarized
target is that the asymmetries can suffer many fewer higher twist effects than
the cross sections.  As shown in Fig.~\ref{fig:belitsky_atpi}, higher twist
corrections, which are expected to be significant in Jefferson Lab kinematics,
likely cancel in the formation of the single spin asymmetry, leading to a
precocious scaling of $A_L^{\perp}$ at moderate $Q^2$ \cite{Fr99, belitsky}.
This is one of the reasons for the intense theoretical interest in these
measurements.

Thus, while the extraction of the absolute spin-dependent cross sections will
undoubtedly be helpful for confirming the theoretical interpretation of our
results, they will be dominated by the higher twist effects.  It is difficult
to quantify the expected precision at this early stage, given that SoLID is not
designed for absolute cross section measurements.  To be conservative, this
proposal only relies on the extraction of the azimuthal asymmetry components
from the target up-down polarization difference, where many systematic
uncertainties cancel.  This can be done reliably, even if the systematic
uncertainties are otherwise too large for reliable absolute cross section
measurements.  The quantitative estimate of detector efficiency and acceptance
correction uncertainties, absolute kinematic offsets, etc. requires
dramatically more study and a possible optimization of the SoLID detector
system.

Ultimately, the measurement presented in this proposal is important preparatory
work for future measurements at the EIC, where the higher twist contributions
should be smaller.  The Electron-Ion Collider is optimized for transverse
single spin asymmetry measurements such as these, and the ability to have both
polarized $^3$He and proton beams will allow $A_{UT}^{sin(\phi-\phi_s)}$ to be
directly compared for the $\vec{n}(e,e'\pi^-)p$ and $\vec{p}(e,e'\pi^+)n$
reactions, without target dilution, over a broad kinematic range.  In the
meantime, the proposed measurement with SoLID is our best short-term
opportunity to considerably advance over the pioneering HERMES data.

\begin{figure}[hbt!]
\begin{center}
(a) \includegraphics[height=6cm]{./figures/t_simple_fit5.png}\\
(b)\includegraphics[height=6cm]{./figures/t_fermi_fit5.png}\\
(c)\includegraphics[height=6cm]{./figures/t_mult_nofermi_fit5.png}\\
\end{center}
\caption{\label{fig:asym_nofermi}{Projected values and uncertainties of the two
    dominant single-spin asymmetry modulations, $\sin(\phi-\phi_S)$ (left) and
    $\sin(\phi_S)$ (right) for the a) {\it simple}, b) {\it fermi} and c) {\it
      mult\_nofermi} cases, as discussed in the text.}}
\end{figure}

\subsection{Fermi Momentum Effects
\label{sec:resp_fermi}}

{\it [SoLID]\ Fermi-momentum is not just a kinematic effect.  It also affects
  the DEMP amplitude.  The $^3$He momentum distribution $\rho(p)$ is plotted in
  Fig.~10 (Appendix~A).  The weighted distribution $p^2\rho(p)$ peaks at
  $p_n\approx$60~MeV/c.  This means that the effective $x_B$ is smeared by
  $\approx p_n/M\approx$6\%.  The significance of this effect should be
  discussed.  Also, if the proton momentum resolution is good enough, it will
  be possible to correct for this effct, event-by-event.}\\[0.2ex]

{\it [Theory]\ The authors may want to switch off $^3$He Fermi motion in their
  simulations and determine how large and in which kinematics they see a
  difference.  Having evidence of non-negligible nuclear effects at an early
  stage would encourage theorists to extend now their calculations from
  inclusive to exclusive measurements for a timely and correct utilization of
  the data the authors propose to take.  It would also be helpful to elaborate
  on the possible corrections in addition to Fermi motion, such as from binding
  and nucleon off-shell effects, as well as corrections beyond the impulse
  approximation from rescattering or final state interactions of the detected
  proton.}\\[0.2ex]

We agree that if the proton momentum resolution is sufficiently good, it will
be possible to correct for Fermi momentum on an event-by-event basis.  We are
familiar with this technique from our work with the A2 Collaboration at Mainz,
for example.  For the purposes of this proposal we take the more pessimistic
view that the proton resolution is likely not good enough to make this
correction, even though the removal of the Fermi momentum affect would simplify
the physics interpretation of our data and further increase its likely physics
impact.

To get a better estimate of the impact of Fermi momentum, we have run the
generator in a variety of configurations and repeated the analysis of
Sec.~\ref{sec:asym}.  These configurations were:
\begin{itemize}
\item{{\it simple:} All Fermi momentum, multiple scattering, energy loss effects
  turned off.}
\item{{\it fermi:} Same as {\it simple} except that Fermi momentum is turned
  on.  This allows the Fermi momentum effect to be isolated separate from other
  effects.}
\item{{\it mult:} All Fermi momentum, multiple scattering, energy loss effects
  turned on.  This is the default configuration, presented in
  Fig.~\ref{fig:asym1_t}.}
\item{{\it mult\_nofermi:} Same as {\it simple} except that Fermi momentum is
  turned off.  This allows the anticipated effect of correcting Fermi momentum
  event-by-event should the recoil proton momentum resolution be sufficiently
  good to be modeled.}
\end{itemize}

While all five modulations were fit in each case, only the two of physics
interest are shown in Fig.~\ref{fig:asym_nofermi}.  It is clearly seen in panel
a) that the agreement between input and output values is very good when all
Fermi momentum, scattering and energy loss effects are turned off.  This
validates our UML procedure.  The effect of Fermi momentum can be seen by
comparing panels b) and a), or alternately Fig~\ref{fig:asym1_t} with panel c).
Their effects are similar, giving rise to a deviation to the
$\sin(\phi-\phi_S)$ asymmetry of about -0.02, and a deviation to the
$\sin(\phi_S)$ asymmetry of about -0.04.  Panel c) clearly shows that if we are
able to correct the data for Fermi momentum on an event-by-event basis, it is
in principle possible to remove most of this effect on the asymmetries, even
though there is multiple scattering, scattering resolution, and other smearing
effects.

We hope this estimate of Fermi-momentum effects at an early stage will
encourage theorists to extend their calculations for a timely and correct
utilization of our proposed data, as suggested in the Theory review comment.

\subsection{Dialog with Theorists}

{\it [SoLID]\ There are a number of important theory issues raised by this
  proposal.  These probably cannot be fully resolved before re-submission, but
  it will be important to have a clear dialog with relevant theorists (and
  experimentalists) in place...  Both Goloskokov and Kroll, and Liutti and
  Goldstein, have published estimates of $\sigma_T$, based on transversity GPDs
  and a twist-3 helicity-flip pion distribution amplitude.  One or the other of
  these theory groups should be engaged in a discussion of both the
  $|\sigma^y_L|$ and $|\sigma^y_{TT}|$ terms.}\\[0.2ex]

We have been in communication with Goloskokov and Kroll on the physics
objectives of this proposal for some years, and they provided helpful comments
to our 2016 proposal.  With the additional time we had available for the 2017
proposal, Goloskokov and Kroll generously provided new asymmetry calculations,
based on their best estimates of $\sigma_T$, $\sigma_{TT}$ for SoLID
kinematics.  We are confident that approval of this proposal will raise the
interest of other theorists to the physics potential of our measurements, and
we will have dialog with them.\\[0.2ex]

{\it [SoLID]\ The QCD factorization theorem implies color transparency for the
  final state $\pi^-$ in this proposal.  Thus the $^3$He$(e,e'\pi^-)$ final
  state interactions (FSI) are identical with $^3$He$(e,e'p)$, just with a more
  exotic scattering amplitude.  It is not practical to obtain full FSI
  calculations before resubmission, but a dialog should be started both with
  the groups doing FSI calculations, and the groups doing Deep Virtual
  calculations on light nuclei.  Empirically, it will be useful to determine if
  the FSI `peak' lies within the $^3$He$(e,e'\pi^-p)pp$ acceptance of this
  proposal.}\\[0.2ex]

We have made some estimates of FSI effects, based on the empirical
parameterization of $\pi N$ differential cross sections discussed in
Sec.~\ref{sec:fsi} of the Appendix.  The results of our study indicate that the
FSI `peak' does not lie within the acceptance of this proposal and that FSI
effects are expected to be small.  Based on the $\pi N$ differential cross
sections and the fact that there are only two proton spectators in the final
state, we anticipate only about 1\% of events will suffer FSI interactions.
The FSI fraction is weakly dependent on $Q^2$, rising to about 1.2\% for events
with $Q^2>$5 GeV$^2$ in our study.  Of these, a large fraction of the FSI
events are scattered outside the triple coincidence acceptance, reducing the
FSI fraction to $\sim 0.4\%$.  This wil be further reduced by analysis cuts
such as $P_{miss}<1.2$ GeV/c.

Over the longer term, we will consult with theoretical groups for a more
definitive study of FSI effects.  For example, Del Dotto, Kaptari, Pace,
Salme and Scopetta recently published a study of FSI effects in SIDIS from a
transversely polarized $^3$He target \cite{dotto} in SBS, SoLID and EIC
kinematics.  The SIDIS final state has more outgoing particles than DEMP, so
there are more opportunities for FSI interactions there than the simple $\pi^-
ppp$ final state considered here.  Nontheless, they were able to show that the
extracted Sivers and Collins asymmetries are basically independent of FSI,
evaluated within the generalized eikonal approximation and a realistic
distorted spin-dependent spectral function.  A similar calculation for DEMP,
after this proposal is approved, would be a natural extension of their work.
