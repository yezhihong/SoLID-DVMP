\section{Responses to Items Identified in the 2016 Review}

Our 2016 SoLID Run-Group proposal was deemed to be of high scientific merit,
but there were a number of technical questions that were asked to be studied
before final approval can be given.  The following list is compiled from the
TAC, Theory, and SoLID review reports, reordered according to topic.  Our 
response to each item is also given.

\subsection{SoLID Acceptance Simulations}

{\it The simulations for this measurement may benefit from tracking DEMP
  events through the full SoLID GEANT4 simulation (GEMC), particularly for
  kinematics with the lowest momentum protons (300~MeV/c).}

This has been done.  See Sec.

\subsection{Experimental Background}

{\it The committee is convinced that the SIDIS background is likely not a
  major problem.  However, an alternate approach (rather than SIDIS
  fragmentation functions) could be used.  The primary background channel under
  study is $^3$He$(e,e'\pi^-)pp$ with the two undetected protons as
  spectators.  The continuum background that can leak under the quasi-exclusive
  peak can be of the form $e+n\rightarrow e'+\pi^-+\Delta^+$ with the
  $\Delta^+$ decaying to $p+\pi^0$.}\\[0.2ex]

{\it The authors may want, however, to expand on possible contamination arising
  from $\Delta^{++}$ production on bound protons, and subsequent decay into
  $\pi^+$ and $p$.}\\
We take this comment to mean the $e+p\rightarrow e'+\pi^-+\Delta^{++}$
background reaction, as otherwise there is no $\pi^-$ in the final state to
satisfy the offline event finder.\\[0.2ex]

Neither $\pi^-\Delta$ final state is expected to be a substantial issue.  We
have investigated and compared the kinematics of the $\pi^-n$ and $\pi^-\Delta$
final states.  The missing momentum of the $\pi^-\Delta$ state is about
500~MeV/c higher than the $\pi^-n$ final state, very similar to the difference
between the DEMP and SIDIS distributions in Fig.~\ref{missing_mom}.  The 1
GeV/c missing momentum cut will eliminate about 50\% of the $\pi^-\Delta$
events.  In addition, the $\pi^-\Delta$ final state is centered about 300 MeV
higher in missing mass than the $\pi^-n$ state, and will be further suppressed
by a cut of approximately $M_{miss}<$1.05~GeV.  We estimate the $\pi^-\Delta$
contamination remaining after the application of both cuts to be very similar
to that already shown in Fig.~\ref{missing_mass}.

\subsection{Resolution and Energy Loss}

{\it The effects of Fermi-smearing, detector resolution, ionization energy
  loss and bremsstrahlung need to be clarified.  Although they seem to all be
  included in Figs.~15 and 16, it was not clear which curves included which
  effects.}

This has been clarified.  See Sec.

\subsection{Projected Uncertainties}

{\it The extraction of the term $|\sigma^y_{TT}+2\epsilon\sigma^y_L|$ in
  Eqn.~8 from the other $\sin\beta$ and $\cos\beta$ terms requires good
  knowledge of the $\beta$-acceptance in each $t$-bin.  This should be shown,
  in addition to the acceptance plots of Fig.~12.}

This has been done.  See Sec.
\\[0.2ex]

{\it The collaboration should attempt to quantify the projected precision of
  the measured spin-dependent cross section.  Although the asymmetry may have a
  smaller error bar, the spin-dependent cross section difference has a simpler
  interpretation.  Measuring the spin-dependent cross section is also
  consistent with the opening sentence of Appendix A.}

We agree that the extraction of these absolute cross sections would be very
useful in terms of theoretical interpreation, but this is a very difficult
question to answer at this early stage, given that SoLID is not designed for
absolute cross section measurements.  To be conservative, this proposal only
relies on the extraction of the azimuthal asymmetry components from the target
up-down polarization difference, where many systematic uncertainties cancel.
This can be done reliably, even if the systematic uncertainties are otherwise
too large for reliable absolute cross section measurements.

The quantitative estimate of detector efficiency and acceptance correction
uncertainties, absolute kinematic offsets, etc. requires dramatically more
study and a possible optimization of the SoLID detector system.  After this
proposal is accepted, we can continue our studies to see if this is a viable
option for the SoLID Collaboration.

\subsection{Fermi Momentum Effects}

{\it Fermi-momentum is not just a kinematic effect.  It also affects the DEMP
  amplitude.  The $^3$He momentum distribution $\rho(p)$ is plotted in Fig.~10
  (Appendix~A).  The weighted distribution $p^2\rho(p)$ peaks at
  $p_n\approx$60~MeV/c.  This means that the effective $x_B$ is smeared by
  $\approx p_n/M\approx$6\%.  The significance of this effect should be
  discussed.  Also, if the proton momentum resolution is good enough, it will
  be possible to correct for this effct, event-by-event.}

We agree that if the proton momentum resolution is sufficiently good, it will
be possible to correct for Fermi momentum on an event-by-event basis.  We are
familiar with this technique from our work with the A2 Collaboration at Mainz,
for example.  However, for the purposes of this proposal we take the
less optimistic view that the proton resolution is likely not good enough.
\\[0.2ex]

{\it The authors may want to switch off $^3$He Fermi motion in their
  simulations and determine how large and in which kinematics they see a
  difference.  Having evidence of non-negligible nuclear effects at an early
  stage would encourage theorists to extend now their calculations from
  inclusive to exclusive measurements for a timely and correct utilization of
  the data the authors propose to take.  It would also be helpful to elaborate
  on the possible corrections in addition to Fermi motion, such as from binding
  and nucleon off-shell effects, as well as corrections beyond the impulse
  approximation from rescattering or final state interactions of the detected
  proton.}

This has now been done, see Sec.

\subsection{Dialog with Theorists}

{\it There are a number of important theory issues raised by this proposal.
  These probably cannot be fully resolved before re-submission, but it will be
  important to have a clear dialog with relevant theorists (and
  experimentalists) in place...  Both Goloskokov and Kroll, and Liutti and
  Goldstein, have published estimates of $\sigma_T$, based on transversity GPDs
  and a twist-3 helicity-flip pion distribution amplitude.  One or the other of
  these theory groups should be engaged in a discussion of both the
  $|\sigma^y_L|$ and $|\sigma^y_{TT}|$ terms.}

We have been in communication with Goloskokov and Kroll on the physics
objectives of this proposal for some years, and they provided helpful comments
to our 2016 proposal.  With the additional time we had available for the 2017
proposal, Goloskokov and Kroll generously provided new asymmetry calculations,
based on their best estimates of $\sigma_T$, $\sigma_{TT}$ for SoLID
kinematics.  We are confident that approval of this proposal will raise the
interest of other theorists to the physics potential of our measurements, and
we will have dialog with them.
\\[0.2ex]

{\it The QCD factorization theorem implies color transparency for the final
  state $\pi^-$ in this proposal.  Thus the $^3$He$(e,e'\pi^-)$ final state 
  interactions (FSI) are identical with $^3$He$(e,e'p)$, just with a more exotic
  scattering amplitude.  It is not practical to obtain full FSI calculations
  before resubmission, but a dialog should be started both with the groups
  doing FSI calculations, and the groups doing Deep Virtual calculations on
  light nuclei.  Empirically, it will be useful to determine if the FSI `peak'
  lies within the $^3$He$(e,e'\pi^-p)pp$ acceptance of this proposal.}

As discussed in Sec.~\ref{sec:fsi} of the Appendix, we have made some estimates
of FSI effects, based on an empirical parameterization of $\pi N$ cross
sections.  Our estimates indicate that most FSI events are scattered outside
the triple coincidence acceptance, and are effectively removed by our analysis
cuts.

Elaborate, based on our findings.

Over the longer term, we will consult with theoretical groups for a more
quantitative estimate of FSI effects.  For example, Del Dotto, Kaptari, Pace,
Salme and Scopetta recently published a study of FSI effects in SIDIS from a
transversely polarized $^3$He target \cite{dotto} in SBS, SoLID and EIC
kinematics.  The SIDIS final state has more outgoing particles than DEMP, so
there are more opportunities for FSI interactions there than the simple $\pi^-
ppp$ final state considered here.  Nontheless, they were able to show that the
extracted Sivers and Collins asymmetries are basically independent of FSI,
evaluated within the generalized eikonal approximation and a realistic
distorted spin-dependent spectral function.  A similar calculation for DEMP,
after this proposal is approved, would be a natural extension of their work.
